\subsection{Les scénarios}

\smallbreak Nous allons maintenant présenter deux de ces scénarios, un premier jugé dur à appliquer, et un deuxième considéré comme le plus efficient.

\subsubsection{Premier scénario}    
\paragraph{Mise en place d’un contrôle des États sur l’exploitation et l’utilisation des ressources rares.}
\medbreak
Ce scénario consiste à mettre en place un contrôle de l’extraction des ressources dont les réserves sont considérées faibles. Au préalable une étude des ressources restantes doit être effectuée afin de mettre en place des mesures plus ou moins conséquentes pour réguler l’extraction et repousser la date de l’épuisement des ses ressources.
Il est très compliqué de mettre en place une base légale sur le contrôle de l’extraction, car la majorité des gisements sont détenues par des sociétés privés. 
Il faut pour cela d’important moyen financier or la plupart de ses ressources se trouvent dans des pays en voie de développement.
Les contraintes de temps sont importantes, et un inventaire complet des ressources se compte en années. 
Il ne faut cependant pas de technologies particulières pour faire cette inventaire.

Ce scénario a une efficience de 2.2/4 avec les notes suivantes :
\begin{itemize}
\item Légales : 1/4
\item Financières : 2/4
\item Temporelles : 2/4
\item Technologiques : 4/4
\end{itemize}


\paragraph{Analyse SWOT}

La force de se scénario est de pouvoir contrôler la consommation à sa source, de protéger les ressources rares, et d’avoir un impact international.
La faiblesse de se scénario est le problème de propriété de ces gisements, et l’implication importante des états que demandée. Il faut aussi noter que certains gisements restent à découvrir et il est compliqué de faire un inventaire en tenant compte de ces ressources potentielles.
Il donne également l’opportunité de faire un inventaire fiable et contrôlé, contrairement aux estimations sur lesquels on se base aujourd’hui.
\medbreak
La principale menace dans ce scénario repose dans la totale collaboration des états, dans un contexte géopolitiques ou l’on peut vouloir refuser de communiquer ses ressources pour des raisons stratégiques.


\subsubsection{Deuxième Scénario}

\paragraph{Mise en place de campagnes de sensibilisation des consommateurs.}

\medbreak
Il s’agit d’informer les consommateurs sur ce qu’est l’obsolescence programmée, et quelles sont les conséquences d’une durée de vie faible de leurs produits, afin de mettre en évidence de manière officielle cette problématique, et lui donner de l'intérêt. C’est une campagne de prévention devant être mener par l’état, comme celles visant à informer sur les drogues, la circulation routière et autres problèmes de société.

\medbreak
Il ne faut pas de base légale particulière pour le déploiement d’une campagne de sensibilisation.
Il n’est pas très coûteux de mettre en place une campagne de sensibilisation efficace, il est même possible de le faire gratuitement avec l'expansion des réseaux sociaux.
La création d’une campagne de sensibilisation n’est pas chronophage, et les résultats peuvent apparaître très rapidement.
Et enfin la réalisation d’une telle campagne ne demande pas de technologie particulière.

Ce scénario a une efficience de 4/4 avec les notes suivantes :
\begin{itemize}
\item Légales : 4/4
\item Financières : 4/4
\item Temporelles : 4/4
\item Technologiques : 4/4
\end{itemize}

\paragraph{Analyse SWOT}

\medbreak
La force de ce scénario est qu’il agit sur l’acteur qui a le plus de poids sur cycle de vie d’un produit. Les entreprises adapteraient donc le produit au attentes du consommateur. L’autre force est l’aspect grande échelle d’une campagne de sensibilisation, qui permet de diffuser un message de manière globale à l’ensemble des consommateurs si elle est bien menée.
\medbreak
Même si l’efficacité d’une campagne de sensibilisation n’est plus a prouver, elle ne touche pas forcément les personnes les plus concernées. L'obsolescence programmée, ou la durée de vie faible des produits n’est pas non plus dans les préoccupations actuelles de la plupart des consommateurs
\medbreak
L’utilisation des réseaux sociaux permet une diffusion aisée d’une information, et de manière ciblé et ce à faible coût, ce qui présente une réelle opportunité.
\medbreak
Cependant la question de l’obsolescence programmée n’est pas connue du grand public, et peut être facilement décrédibilisée.