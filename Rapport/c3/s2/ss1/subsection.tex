\subsection{Système de notation}

Les solutions pour lutter contre la durée de vie limitée des produits ou contre l’obsolescence programmée existent, et nous allons essayer d’en expliciter quelques une au long de  cette partie.

Joachim Fuchs, de la \testit{Haute École de Gestion} de Genève, a publié son mémoire en 2012. Celui-ci traite de l'obsolescence programmée et propose des solutions pour en inverser la tendance.

\begin{itemize}
\item Mise en place d’un contrôle des États concernant l’exploitation et l’utilisation des ressources en voie de disparition 2.2/4
\item Imposition de quotas de recyclage aux entreprises qui consomment des ressources rares 2.6/4
\item Mise en place de campagnes de sensibilisation des consommateurs 4/4
\item Mise en place d’une durée minimum de garantie 3.2/4
\item Création d’une définition officielle de l’obsolescence programmée et référencement de toutes les méthodes connues 3.1/4
\end{itemize}

Pour ordonner ses solutions il utilise un système de notation particulièrement intéressant. Chaque scénario possède quatre notes, reposant sur les questions technologiques, légales, de temps et de financement. Il associe bien sûr à chaque note une justification.
La moyenne de ces quatre notes, pondérées par des coefficients, donne la note globale qu’il appelle l’efficience de sa solution. L’aspect temps a pour coefficient 0.3, l’aspect financement 0.3, l’aspect technologique 0.2 et l’aspect légal 0.2. C’est cette note qui quantifie les résultats attendus en cas de mise en place de la solution, et c’est avec cette note qu’il ordonne ses scénarios.

Chaque solution est ensuite soumise à une analyse SWOT (Force, Faiblesse, Opportunités et Menaces), ce qui permet de regrouper les atouts et défauts de chaque aspect du système de notation dans un même schéma d’analyse. 
