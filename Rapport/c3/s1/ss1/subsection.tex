\subsection{Les industriels responsables et la garantie longue durée}


<<<<<<< HEAD
\loremipsum
=======

Ayant une parfaite technique pour combattre l’obsolescence programmée des articles du côté du client, la garantie facultative est pourtant un moyen pour apaiser les éventuels consommateurs. Cette assurance optionnelle offerte par l'entreprise a des  spécificités qui n'existent pas dans les garanties légales, ces particularités, on les trouve dans plusieurs exemples.
 
Quand le commerçant ne offre aucune garantie ou quand cette dernière est limitée dans son contenu,ça se peut souvent que le constructeur attribue une garantie pour son produit. Celle-ci, fréquemment appelée "garantie fabricant" , est aussi optionnelle.
 Or, plusieurs commerçants sur Internet ne présentent pas de garantie pour  garder des prix si bas  possibles et ils comptent sur la garantie de l'entreprise. Sachant que,l'acheteur, normalement ne devra pas rendre le produit acheté au vendeur mais au constructeur . En conséquence, en cas de problème, on doit se rappeler selon la marque du produit, que les formalités à suivre sont à peu près facile .

Cet exemple, on le trouve particulièrement avec la multinationale Apple qui fournit une garantie fabricant d’un an. Cet acte est toute à fais normale car tous les entreprises du domaine électrique et électronique ne offrent que des garanties d’un an, mais Apple laisse croire à ses clients que la seule garantie du produit ne durent qu'une seule année depuis la date d'achat du produit,après ce temps le produit n'est plus garantie. Ceci est juste mais l'entreprise américaine joue sur la différence entre la garantie constructeur qui dure un an et les garanties légales qui durent deux ans et plus, et il joue sur l'ignorance de ses clients des garanties qui existe aujourd'hui.

Aujourd'hui, on commence à voir des entreprises ou des commerçants qui offrent des garanties de longue durée pour leurs produits.
 C'est le cas de la société anglaise d’électroménager Dyson, qui a proposé des aspirateurs avec une longue garantie de cinq ans. Ce qui veut dire que Dyson offre trois ans de garantie de plus, sachant que la garantie constructeur est indiquée en une seule année. Normalement pour bénéficier d'une telle garantie, le client devrait payer le prix normal du produit plus une augmentation liée à la garantie,au contraire de la bonne proposition de la marque Dyson. 


L’électroménager n'est pas le seul, le domaine automobile, lui aussi a osé de proposer  des garanties de longues durées, par l’intermédiaire de la marque KIA. Jusqu’à aujourd'hui, le constructeur automobile sud-coréen avec son modèle CEE’D, reste le seul à proposer une garantie gratuite de sept ans ou 150 000 kilomètres. Ceci est difficile pour les Européens qui conservent leur voiture pour presque sept ans. Malheureusement, pour les automobilistes, pas toutes les marques ont suivi l’initiative de la marque KIA. Ainsi, si le client veux vendre sa voiture pendant la durée de la garantie indiquée dans sept ans, c'est le nouvel acheteur qui profitera du reste de la garantie. Cependant, les constructeurs automobiles européens restent avec des garantie de deux ans.

On trouve aussi,l'entreprise néerlandaise Philips qui a décidé de prendre en compte les conséquences écologiques de l'obsolescence programmée, notamment les produits jetables. Ces dernières années, Philips a dirigé son investissement vers  les produits durables. L'un de ces produits durables est l’ampoule LED (Light Emitting Diode), en français la diode électroluminescente ou la diode qui  diffuse de la lumière , cette ampoule a une durée de vie de 25 ans à l'opposée de l'ampoule à incandescence qui dure 2 000 heures (presque 2 ans), elle consomme presque 3 watts d’électricité contrairement à l'ampoule à incandescence qui utilise 25 watts d’électricité. 

... à suivre ....
… 
>>>>>>> a7c77b9fa4d126cecfe71b2fe7433c22a66d16ef
