\subsection{L'économie circulaire}

L’économie circulaire ou l’économie collaborative est une notion économique qui s'inscrit dans le cadre du développement durable afin de de produire des biens et services en respectant concept de l'économie verte, en luttant contre l'obsolescence programmée et le gaspillage des matières premières.

Aujourd'hui, l'économie circulaire est devenu une solution optimale pour réduire les dégâts de l'obsolescence programmée et pour affronter ce modèle de développement économique linéaire, qui encourage l’hyper-consommation et le gaspillage. Cette notion est très simple, faute de utiliser plus de ressources pour créer des produits obsolètes qu'on finit par jeter, il est préférable de réintroduire dans le cycle de production ce qu'on appelle actuellement des déchets.
En États-Unis, les entreprises gagnent 43 milliards dollars par an en recettes, en récupérant des pièces à partir de produits. Chaque smartphone nous jetons détient 100 dollars des matériaux que nous pourrions facilement utiliser à nouveau.     À l'heure actuelle, seulement 20\% de ce matériel est recyclé à l'échelle mondiale, ce qui signifie que 100 milliards de dollars de matériel est perdu au sol tout juste un an après l'achat. Le coût de recyclage des téléphones mobiles pourrait être réduit de moitié si l'industrie les rendait plus faciles à démonter, afin que les nouveaux téléphones se produisent à partir du recyclage des téléphones obsolètes. Et avec la vitesse à laquelle nous changeons nos téléphones actuellement, vous pouvez facilement imaginer le taux des pertes causées par l'obsolescence programmée.

Plusieurs régions sont beaucoup plus avancés dans le développement de l'économie circulaire. En Europe, prenant l'exemple de Renault qui possède une entreprise de recyclage qui a déjà gagner 300 millions de dollars de revenus, et possède la marge bénéficiaire la plus élevée dans le groupe. Le secteur du recyclage de la Chine est si lucrative que l'un de les leaders du domaine aurait dit qu'il veut acheter le New York Times. 
Les États-Unis traînent loin derrière derrière l'Europe et l'Asie dans la récupération des ressources. Il récupère environ 8\% de tous les plastiques (l'Europe récupère presque deux fois plus), 15\% des textiles (le taux du Royaume-Uni est quatre fois plus élevé) et 20\% de l'aluminium (le Japon récupère 98\% de ses métaux).
Mise en place, il y a presque 20 ans , le tri sélectif est une notion importante avantageuse pour l’environnement, c'est la phase qui vient avant le recyclage des déchets. Même si, le recyclage donne des résultats positifs, l’obsolescence programmée Contribue à la hausse du nombre de des déchets chaque jour. Quelquefois, il est impossible de recycler quelques produits, ce qui crée un problème très grave au niveau sanitaire. 
Actuellement, les déchets des produits électriques et électroniques posent un gros problème lorsqu'il faut les recycler. En 2008, chaque français produisait 543 kilogrammes de déchets, dont 16 à 20 kilogrammes de déchets d’équipements électriques ou électroniques . Le problème que le recyclage de ces déchets est très dure, et par ça peut aller vers le non-recyclage de ces biens. Dans ce cas, on se pose face est à l'impossibilité du recyclage : le recyclage est préférable si on parle des déchets ménagers mais la réalité est totalement différente pour les déchets des produits électriques et électroniques. 
