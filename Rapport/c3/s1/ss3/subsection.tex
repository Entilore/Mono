\subsection{L'économie circulaire}


\bigbreak
L’économie circulaire ou l’économie collaborative est une notion économique qui s'inscrit dans le cadre du développement durable afin de produire des biens et des services en respectant le concept de l'économie verte et en luttant contre l'obsolescence programmée et le gaspillage des matières premières %\cite{wiki_ec}.


\bigbreak
Aujourd'hui, l'économie circulaire est devenue une solution optimale pour réduire les dégâts de l'obsolescence programmée et pour affronter ce modèle de développement économique linéaire, qui encourage l’hyper-consommation et le gaspillage. Cette notion est très simple, faute d'utiliser plus de ressources pour créer des produits obsolètes qu'on finit par jeter, il est préférable de réintroduire dans le cycle de production ce qu'on appelle actuellement des déchets.


\bigbreak
Aux États-Unis, les entreprises gagnent 43 milliards dollars par an en recettes \cite{usatoday}, en récupérant des pièces à partir des produits jetés. Chaque smartphone que nous jetons est ainsi l'équivalent de 100 dollars de matériaux que nous pourrions facilement utiliser à nouveau. À l'heure actuelle, seulement 20\% de ces ressources sont recyclées à l'échelle mondiale, ce qui signifie que 100 milliards de dollars de matériel est perdu au sol un an après l'achat. Le coût de recyclage des téléphones mobiles pourrait être réduit de moitié si l'industrie les rendait plus faciles à démonter, afin que les nouveaux téléphones se produisent à partir du recyclage des téléphones obsolètes. Et avec la vitesse à laquelle nous changeons nos téléphones actuellement, on ne peut qu'imaginer les pertes causées par l'obsolescence programmée.


\bigbreak
Plusieurs régions sont beaucoup plus avancés dans le développement de l'économie circulaire. En Europe, nous avons l'exemple de \textit{Renault} qui possède une entreprise de recyclage qui a déjà gagné 300 millions de dollars de revenus, et possède la marge bénéficiaire la plus élevée dans le groupe. Le secteur du recyclage en Chine est si lucratif que l'un de les leaders du domaine aurait dit qu'il voulait acheter le \textit{New York Times}. Les États-Unis sont loin derrière l'Europe et l'Asie dans la collecte des ressources. Ils récupèrent environ 8\% de tous les plastiques (l'Europe récupère presque deux fois plus), 15\% des textiles (au Royaume-Uni, ce taux est quatre fois plus élevé) et 20\% de l'aluminium (le Japon récupère 98\% de ses métaux).

\bigbreak
Mis en place il y a presque 20 ans , le tri sélectif est une notion avantageuse pour l’environnement, c'est la phase qui vient avant le recyclage des déchets \cite{tri}. Même si le recyclage donne des résultats positifs, l’obsolescence programmée contribue à la hausse du nombre de déchets chaque jour. Quelquefois, il est impossible de recycler des produits. On peut penser aux piles, qui sont particulièrement polluantes ou aux lampes à incandescence.

Actuellement, les produits électriques et électroniques posent un gros problème lorsqu'il faut les recycler. En 2008, chaque français produisait 543 kilogrammes de déchets, dont 16 à 20 kilogrammes de déchets d’équipements électriques ou électroniques \cite{opSsg}. Le problème que le recyclage de ces déchets est très dur, à cause des nombreux composants, électroniques ou non, la plupart du temps mélangé. Ainsi, il devient tellement difficile de séparer les matériaux qu'on peut parfois ne pas du tout recycler. Dans ce cas, on est face à l'une des problématiques du recyclage : il est possible lorsqu'on parle des déchets ménagers mais la réalité est totalement différente pour les produits électriques et électroniques. 


\bigbreak
Certaines entreprises et états sont donc déjà conscients du problème de la quantité trop importante de déchets. Cependant nous avons vu que cela n'était pas suffisant. On pourrait donc se dire que le dernier acteur, le consommateur est capable de réagir et pourrait être sensibiliser à cette problématique.