\subsection{Loi sur la transition énergétique pour une croissance verte}


En octobre 2014 fut votée à l'Assemblé Nationale l’adoption du projet de loi relatif à \textit{la transition énergétique pour une croissance verte}. Le texte de loi contient une série d’articles qui vise le renforcement de l’indépendance énergétique de la France, la lutte contre les gaspillages, la promotion de l’économie circulaire et la lutte contre le changement climatique.


\bigbreak
La politique énergétique nationale possède différents objectifs dont les principaux sont \cite{loi_te} :
\begin{itemize}
  \item la réduction de la consommation énergétique finale de 50 \% en 2050 par rapport à 2012
  \item la réduction des émissions de gaz à effet de serre de 40 \% entre 1990 et 2030
  \item la baisse de la consommation énergétique primaire des énergies fossiles de 30 \% en 2030 par rapport à 2012
  \item l'augmentation de la part des énergies renouvelables de la consommation finale brute de l’énergie à 23 \% en 2020 et à 32 \% en 2030
\end{itemize}


\bigbreak
La loi a donné la priorité au bâtiment, secteur qui représente à lui seul la moitié de la consommation énergétique de la France. Le gouvernement présente une stratégie nationale à l’horizon 2050 pour la rénovation énergétique des bâtiments. Avant 2030, tous les bâtiments privés résidentiels qui consomment plus que 330 kilowattheures par mètre carré et par an devront subir une rénovation énergétique. Ainsi la France prévoit de \textit{« rénover énergétiquement 500~000 logements par an à compter de 2017, dont au moins la moitié est occupée par des ménages modestes. »}


\bigbreak
Le secteur du transport, considéré comme le premier émetteur de gaz à effet de serre (27 \%), est lui aussi visé par la nouvelle loi, qui cherche à \textit{« développer les transports propres pour améliorer la qualité de l’air et protéger la santé »}. L’état vise à renouveler le domaine en y intégrant au moins 50 \% de véhicules électriques ou hybrides rechargeables, et parallèlement l’installation d’au moins sept millions de points de charge installés sur les places de stationnement des ensembles d’habitations et sur les places de stationnements publics.


\bigbreak
Dans ses articles \textit{n\degre 19, n\degre 20, n\degre 21 et n\degre 22}, la loi encourage le développement de l’économie circulaire et la lutte contre le gaspillage. Ainsi, elle donne la priorité à la prévention et la réduction de la quantité des déchets produits par les français. Celle-ci devra être réduite à 10 \%  en 2020 par rapport à 2010, et réduire de 30 \% les déchets non dangereux en 2020 et de 50 \% en 2025.


\bigbreak
Dans \textit{son article 22 ter A}, le texte de loi définit l’obsolescence programmée comme :

\begin{itshape}« I. – L’ensemble des techniques par lesquelles un metteur sur le marché vise, notamment par la conception du produit, à raccourcir délibérément la durée de vie ou d’utilisation potentielle de ce produit afin d’en augmenter le taux de remplacement.

II. – Ces techniques peuvent notamment inclure l’introduction volontaire d’une défectuosité, d’une fragilité, d’un arrêt programmé ou prématuré, d’une limitation technique, d’une impossibilité de réparer ou d’une non-compatibilité. »
\end{itshape}

\bigbreak
L’obsolescence programmée des produits pourra être punie, l’amendement comprend également des sanctions pour ceux qui ne respecteraient pas ces engagements. Le texte prévoit une amende maximale allant jusqu'à 300~000 euros et une peine de prison de deux ans au plus \cite{sanctionloi}. L'affichage de la durée de vie des produits deviendra obligatoire à partir d'une valeur équivalente à 30 \% du \textit{Smic}.


Une partie des politiques est donc consciente de la problématique de l'\op et cherche aussi des solutions viables. D'autres idées semblent également envisageables par les villes ou communautés.
