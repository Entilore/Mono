\subsection{La consommation (ir)responsable}

Combien de temps le consommateur passe-t-il à réfléchir à la question \textit{De quoi ai-je besoin ?}.
La réponse est sans doute \textit{trop peu}. Nous allons donc passez en revue quelques causes probables de ce désengagement du consommateur de la consommation elle même.
% j'aimerais bien trouver des sources mais j'ai rien, donc on reste dans le général

% les différents modes de vente
\medbreak
Les différents modes de ventes 
Par exemple, dans le cas de la vente à distance, le consommateur n'as plus la possibilité d'observer ou de manipuler le produit autrement que par des photos ou des vidéos.
Cela peut être un désavantage dans le cas ou le produit ne correspond pas exactement à ce que cherche le consommateur.
%exemples
\medbreak
On peut prendre pour exemple le cas de la recherche d'un produit compatible avec un produit possédé.
Un second exemple peut être tout simplement l'essayage. Les tailles pour les vêtements varies souvent de fabricants en fabriquant et acheter sans essayer est une loterie risqué.
%quelques solutions
\smallbreak
%protection du consomateur par l'état - dans le cadre des garanties retours
Heureusement, peut-on dire, le consomateur dispose de garantie obligatoire durant laquelle il est possible de rnevoyer le produit si il ne correspond pas à ses attentes. On déplorera le faite que lesdites garanties nécssite souvent que le prioduit n'ai pas été sorti de son embalage.

%la course au prix
\medbreak
Dans ces temps de crise, un autre fcateur décisif de l'achat est le \textit{prix} de 
%crédit qui encourage la dépense immédiate, promotions à tout vas, vente privé ou 'flash'

%le manque de temps

