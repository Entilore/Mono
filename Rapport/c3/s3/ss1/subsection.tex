\subsection{La consommation (ir)responsable}

Combien de temps le consommateur passe-t-il à réfléchir à la question \textit{De quoi ai-je besoin ?}.
La réponse est sans doute \textit{trop peu}. Nous allons donc passez en revue quelques causes probables de ce désengagement du consommateur de la consommation elle même.
% j'aimerais bien trouver des sources mais j'ai rien, donc on reste dans le général

% les différents modes de vente
\medbreak
Les différents modes de ventes  : 

Par exemple, dans le cas de la vente à distance, le consommateur n'as plus la possibilité d'observer ou de manipuler le produit autrement que par des photos ou des vidéos.
Cela peut être un désavantage dans le cas ou le produit ne correspond pas exactement à ce que cherche le consommateur.
%exemples
\medbreak
On peut prendre pour exemple le cas de la recherche d'un produit compatible avec un produit possédé.
Un second exemple peut être tout simplement l'essayage. Les tailles pour les vêtements varient souvent de fabricants en fabricants et acheter sans essayer est une loterie risquée.
%quelques solutions
\smallbreak
%protection du consomateur par l'état - dans le cadre des garanties retours
Heureusement, peut-on dire, le consommateur dispose de garantie obligatoire durant laquelle il est possible de renvoyer le produit si il ne correspond pas à ses attentes. On déplorera le fait que lesdites garanties nécessitent souvent que le produit n'ai pas été sorti de son emballage, ce qui est en opposition avec les cas de mise en œuvre.

%la course au prix
\medbreak
Dans ces temps de crise, un autre facteur décisif de l'achat est le \textit{prix}, tout simplement.
Il est humain de préférer, parmi un panel de produits similaires, celui qui nous parait le moins cher.
L'effet pervers des garanties cité précédemment est que le consommateur perd une partie de son regard critique au profit d'une course au prix le plus bas, course habilement alimentée par certains distributeurs.

\smallbreak
Ces distributeurs encouragent la dépense immédiate et sans réflexion du consommateur en lui présentant des offres qui semblent être de bonnes affaires, mais dans un temps limité. C'est le cas notamment des ventes dites \textit{flash} et dans une moindre mesure, des promotions et des soldes.

\smallbreak
Le consommateur, pensant faire une bonne affaire, se dit \textit{qu'il serait dommage de manquer cette occasion}, même si il n'as pas un réel besoin du produit, car \textit{il pourrait en avoir besoin plus tard}. C'est cet achat, effectué par \textit{prévention}, qui est l'effet consumériste.

%crédit qui encourage la dépense immédiate, promotions à tout vas, vente privé ou 'flash'