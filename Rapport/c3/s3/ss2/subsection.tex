\subsection{Rétablir la démarche mercatique}
%associations de consommateurs pour faire des comparatifs
\medbreak
Nous avons donc vu que le consommateur est largement influencé par les réclames et multiples promotions, qui le poussent à acheter des biens dont il n'a pas réellement le besoin. Nous allons maintenant voir les solutions envisageables pour sortir le consommateur de ce cycle sans fin et rétablir une relation mercatique équilibré.

\medbreak
Les plus grands acteurs de ce ré-équilibrage sont les associations de consommateurs. %need ref en bas de page avec les noms
Ces associations se posent en entitées indépendantes, financées par les consommateurs eux-même via des campagnes de dons et les abonnements aux revus qu'elles diffusent. 
\smallbreak
Elles effectuent des tests indépendants sur certaines gammes de produits, mais jouent aussi un rôle central quand il arrive que le consommateur désire engager une action légale contre une entreprise.

%exemples d'actions d'association ?

\medbreak
Mais souvent, même si le consommateur est supporté par des associations, la machine juridique est souvent lourde à mettre en route, et les frais fixe trop importants. C'est d'autant plus vrai dans la production de masse, où le prix du produit peu parfois diminuer à des niveaux dérisoires en regard des du coup horaire d'un avocat.

\subsubsection{Les recours collectifs, aux États-Unis et en France}

\medbreak
La réponse évidente à cette disymétrie fabriquant/consommateurs est de regrouper ces derniers afin qu'unis, ils forment une seul entité à même de mettre ses ressources en commun et à partager les dédomagements.

\bigbreak
Très connu aux États-Unis sous le nom de \textit{Class Action}, certains sont même très connus, comme celui contres Apple, au sujet de la batterie défectuseuse et non remplaçable de l'iPod, qui obligea la firme de Cupertino à verser 1500\$ à tout les membres de la contestation et à prendre en charge les frais de justice engagé\cite{classApple}. Les recours collectifs sont cependant très critiqué au USA, tant ils profitent aux cabinets d'avocats (certains sont même spécialisés dans ce secteur juridique) et peuvent parfois inverser totalement l'équilibre des forces, en permettant au consommateur de spéculer sur les bénéfices.

\medbreak
Ces \textit{Recours collectifs} ont été introduit dans le droit Français le 13 févier 2014, et on permis à l'UFC-Que Choisir de lancer la première action groupé de consommateurs contre l'agence immobilière \textit{Foncia} pour avoir prélevé, à tord, 2,30\euro par mois durant 5 ans pour un total de 44 millions d'euros\cite{adg_Foncia}.

\subsubsection{Le produit jetable}

Un autre secteur où le consommateur a le choix d'influencer par sa décision l'axe de sa consomation est le domaine du produit jetable.

\smallbreak
En effet, qui utilise encore des mouchoires en tissu de nos jour ?
Et même si les appareils photos jetables on disparu de la valise des touristes au profit des téléphones portables, on trouve maintenant des téléphones dis jetable. % ref needed
