\subsection{Rétablir la démarche mercatique}
%associations de consommateurs pour faire des comparatifs
\medbreak
Nous avons donc vu que le consommateur est largement influencé par les réclames et multiples promotions, qui le poussent à acheter des biens dont il n'a pas réellement le besoin. Nous allons maintenant nous intéressé à comment sortir le consommateur de ce cycle sans fin et rétablir une relation mercatique équilibré.

\smallbreak
Les plus grand acteurs de ce ré-équilibrage sont les associations de consommateurs. %need ref en bas de page avec les noms
Elles effectuent notamment des test indépendants sur certaines gammes de produits, mais jouent aussi un rôle central quand il arrive que le consommateur désire engager une action légal contre une entreprise.
%exemples d'actions ?

Ces actions sont assez peu courantes en France, mais elles sont très utilisé aux USA sous le nom de \textit{Class Action}.