\chapter*{Introduction}

 Sujet d'actualité s'il en est, l'obsolescence programmée est en tout cas apparue récemment sur le devant de la scène médiatique et politique, suite au reportage de Cosima Dannoritzer diffusé sur ARTE en 2011. Celui-ci est devenu rapidement une référence en la matière et les exemples qu'il traite sont ainsi souvent cités


Etant un terme assez obscur, il possède différentes définitions. Pour certains, il signifie : 


Apparu en 1932 dans l'essai de l'agent-immobilier Bernard London, le terme est réellement arrivé devant le grand public suite au reportage de Cosima Dannoritzer diffusé sur ARTE en 2011. Celui-ci est devenu rapidement une référence en la matière et les exemples qu'il traite sont ainsi souvent cités


(Intéressés par le développement durable), nous nous sommes posés de nombreuses questions au sujet de l'obsolescence. L'impression que la durée de vie des produits actuels est plus faible que par le passé nous paraissait plus ou moins fondée, cependant après quelques recherches, nous nous sommes rapidement aperçus que la réalité est moins simple. En effet de nombreuses personnes considèrent l'obsolescence comme un mythe, une théorie du complot visant à accuser les industriels de la baisse de la durée de vie des biens manufacturés. On peut également estimer que le consommateur a un rôle très important dans le renouvellement des produits, étant lui-même un élément central de la société de consommation


Théorie du complot, réalité inévitable de notre société de consommation ou machination orchestrée par les industriels, nous avons tenté de répondre à ces interrogations.


Pour commencer, nous 
