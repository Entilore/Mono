\chapter*{Introduction}

\setcounter{page}{1}

Sujet d'actualité s'il en est, l'\op est en tout cas apparue récemment sur le devant de la scène médiatique et politique. Apparu en 1932 dans l'essai de l'agent-immobilier Bernard London, le terme est réellement connu du grand public suite au reportage de Cosima Dannoritzer diffusé sur ARTE en 2011. Celui-ci est devenu rapidement une référence en la matière et les exemples qu'il traite sont ainsi souvent cités.

\smallbreak Le terme d'\op reste encore assez obscur et on lui attribue différentes définitions. L'obsolescence en soit est le fait, généralement pour un produit, d'être désuet ou dépassé, par la technique l'esthétisme ou la mode.

\smallbreak Plusieurs types d'obsolescence en dérivent ensuite. L’obsolescence fonctionnelle simple, la plus courante est liée au fabricant, qui a prévu une faible durée de vie de son produit, afin d'obliger le consommateur à racheter son produit. On peut y inclure également l'incompatibilité de logiciels, qui ne fonctionnent plus avec de nouveaux appareils, forçant au rachat. Certains y intègrent finalement l'obsolescence esthétique, puisque créée par les designers travaillant pour les industriels.

\smallbreak Nous nous sommes posés de nombreuses questions au sujet de l'obsolescence. L'impression que la durée de vie des produits actuels est plus faible que par le passé nous paraissait plus ou moins fondée, cependant après quelques recherches, nous nous sommes rapidement aperçus que la réalité est moins simple. En effet de nombreuses personnes considèrent l'obsolescence comme un mythe, une théorie du complot visant à accuser les industriels de la baisse de la durée de vie des biens manufacturés. On peut également estimer que le consommateur, étant lui-même un élément central de la société de consommation, a un rôle très important dans le renouvellement des produits, puisque c'est finalement lui qui choisit d'acheter et de racheter de nouveaux produits. Enfin, certains considéraient l'obsolescence comme faisant partie intrinsèquement de notre société et étant ainsi nécessaire.

\bigbreak Nous avons tenté de répondre à l'interrogation : l'\op, mythe, réalité inévitable de notre société de consommation ou machination orchestrée par les industriels ?

\bigbreak Pour commencer, nous tenterons de clarifier quelques points concernant l'obsolescence\\ programmée, puis nous tenterons de déterminer les grandes problématiques de son existence au sein de notre société de consommation. Finalement, nous nous sommes projetés sur l'avenir et les pratiques en cours pour limiter les effets de l'obsolescence.

%\medbreak Pour commencer, nous tenterons de clarifier quelques points concernant l'obsolescence programmée. Premièrement nous montrerons ce qui tend à nous faire penser qu'elle n'existe pas, puis les arguments contestant cette vision des chose. Finalement nous présenterons les différentes obsolescences, tels que décrites par ses détracteurs.

%\medbreak Puis nous tenterons de déterminer les grandes problématiques de son existence au sein de notre société de consommation, puisqu'elle en est tout d'abord un fondement important. Cependant cela conduit à certaines dérives, dangereuses pour l'environnement, que nous développeront. Enfin nous prendrons en considération la gestion des matières premières, dont la réutilisation, que l'on pourrait penser aisée et nécessaire, est en réalité peu facilement mise en œuvre.

%\medbreak Nous nous sommes finalement projetés sur l'avenir et les pratiques en cours pour limiter les effets de l'obsolescence. Nous avons donc pris en compte les différentes actions en cours, mais avons également recherché des solutions afin de sortir de ce mode de consommation. Enfin, nous nous sommes concentrés sur l'importance du consommateur dans le système économique de l'obsolescence et des façons de changer les mentalités pour l'avenir.
