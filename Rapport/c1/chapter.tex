\chapter{L'\OP}

Définir l'\op est une tâche ardue. Dans son livre \textit{Bon Pour la Casse}, Serge Latouche passe en revue les différentes définitions existantes. D'après lui, aucune n'est suffisante pour décrire le phénomène, car le sujet est complexe, et faits observés trop peu nombreux. 

Prouver que l'\op existe est une tâche encore plus ardu. Une énorme controverse existe de nos jours, qui oppose principalement écologistes et association de consommateurs aux industriels. 
\smallbreak
Certains pensent qu'elle est réelle. C'est l'avis des consommateurs, qui se sentent lésés lorsqu'un objet casse quelques mois après la fin de la garantie.

C'est également  l'avis des écologistes, qui tentent  de réduire cette pratique. La fabrication d'un objet est très souvent polluant. Demander aux industriels de réduire la quantité de déchets émis est une solutions qui peut permettre de réduire le réchauffement climatique, mais il serait plus efficace de réduire la fréquence de renouvellement des produits. Ainsi, augmenter  la durée de vie d'un ordinateur portable d'un an pourrait réduire d'un tiers la quantité de déchets émise.  

Dans ce chapitre, nous essai  
