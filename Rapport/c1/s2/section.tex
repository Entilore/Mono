\section{L'\op, une réalité ?}

Nicols Fox, essayiste et romancière, est sûre que l'\op existe, et elle s'y oppose en écrivant des sonnets de complainte : 
\itshape\begin{center}
\begin{verse}
Nonobstant, cette boîte déborde d'innombrables,

Lampes de poche inutilisables,

Irréparables, presque à usage unique,

Leur prompte relégation\\
M’apparaît bien inique !

Est-ce donc trop demander,

Que d'avoir à portée\\
Des lampes qui puissent durer ?
\end{verse}
\end{center}
\normalfont
Comme elle, de nombreux utilisateurs sont exaspérés de la facilité qu'ont les objets à casser. Certains se sentent arnaqués, d'autres y voient plutôt le côté polluant. 

\bigbreak

Aucune preuve parfaite n'a jamais été trouvée pour affirmer que l'\op existe. L'argumentation de ceux qui défendent son existence est cependant solide. Quelques entreprises ont déjà été traînées en justice, et certaines coïncidences sont trop grosses pour être dues au hasard.