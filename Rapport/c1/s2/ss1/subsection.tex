\subsection{Des produits de moins en moins durables ?}

Une impression de diminution de la durabilité des produits du quotidien est souvent ressentie par les utilisateurs. Qui ne s'est jamais étonné que le four de vingt ans des grands-parents soit toujours fonctionnel, alors qu'on vient de changer le sien, hors d'usage après 4 ou 5 années de bons et loyaux services. 

Il semble que la durabilité de vie de certains produits ait baissé depuis le milieu du dernier siècle. 

\bigbreak
La Ford T, une voiture produite par l'entreprise du même nom, est l'une des premières victimes de l'\op. Sortie en 1908, elle était une voiture durable, conçue pour durer ; le taux de pannes était donc très faible.   


Seulement, cette voiture était un modèle unique : elle était majoritairement produite en noir. Son esthétisme ne plaisait pas à tout le monde. De plus, elle était peu confortable, et pas forcément pratique à utiliser. Il fallait par exemple la démarrer à la manivelle. 

\label{GM}
General Motors (\textit{GM}), souhaitant également avoir sa part de marché dans l'automobile, décide de créer sa propre voiture. Devant la qualité de la Ford T, il semblait impossible à \textit{GM} d'inventer un modèle capable de surpasser la vedette. Il décidèrent ainsi de construire une automobile moins fiable, mais plus esthétique, et plus pratique. Certains effets de mode furent ainsi introduits : de nouveaux types étaient créés tous les ans, différents par leur design et par leur capacités.


Cependant, les créateurs de ces voitures décidèrent de sacrifier la longévité, et les voitures tombaient plus souvent en panne. Mais le consommateur en profitait pour en acheter une nouvelle.


La volonté de Alfred Sloan, le dirigeant de \textit{GM}, était justement que chaque voiture dure juste assez longtemps pour que le client rembourse le prêt nécessaire à l'achat initial.
Il peut ainsi en contracter un nouveau pour acheter un véhicule plus récent. Le but du directeur de l'entreprise automobile était de réduire encore plus la durée de vie de ses produits.

Serge Latouche écrit dans \textit{Bon Pour la Casse}\cite{bpc}, que l'ambition finale du président de \textit{General Motor} était d'arriver à une longévité d'un an.

\bigbreak

L'exemple des voitures de \textit{GM} est l'un des premiers cas de réduction volontaire de la durée de vie d'un produit. L'entreprise remportera la bataille du marché automobile, en privilégiant le design à la fiabilité.

Ce nouveau modèle économique attira de nombreuses autres entreprises. D'après Serge Latouche \cite{bpc}, ce système aurait connu beaucoup de succès auprès d'autres industriels. A. Sloan présente en 1950 sa méthode à d'autres entreprises. De nombreux produits de très courte durée de vie apparaîtront (bungalows, tente en papier, ... ). C'est également à ce moment là qu'apparaissent les produits jetables.
 
 
\bigbreak
Un autre exemple, peut-être plus flagrant est celui des bas en nylon.
L'entreprise de chimie de \textit{Du Pont de Nemours} a créé dans les années 1940 des bas fait de nylon.
Ces bas étaient de très bonne qualité, et réputés pour ne pas filer. La société \textit{Du Pont de Nemours} remarqua cependant que les bas étaient trop solides. 
Ainsi, leur durée de vie ne permettaient pas à la compagnie de survivre.
Selon le site web \url{obsolescence-programmee.fr},  la qualité de leur produit aurait été volontairement détériorés de manière à améliorer la marge de l'entreprise, au détriment des consommateurs. 

\bigbreak
Il est donc possible que certains industriels altèrent la durabilité de produits afin d'augmenter leur chiffre d'affaire. Les exemples donnés datent de la moitié du XIXe siècle, cependant ils sont toujours d'actualité. Une voiture ne dure en général que dix ans, et les bas en nylons demeurent aussi facilement déchirable. 

De plus, le modèle économique créé par A. Sloan n'a pas été abandonné. Ainsi, la durabilité de certains produits quotidiens a clairement baissé depuis les 50 dernières années. 
Une étude \cite{opSsg}, réalisée en septembre 2010, montre que la durée de vie d'objets de la vie de tous les jours est belle et bien plus faible qu'auparavant.
Ainsi, un appareil électroménager aurait perdu quatre années d'espérance de vie (de 11 ans à 7 ans), une télévision près de 10 (dix à quinze ans pour une TV à tube cathodique, cinq ans pour un écran plat). 

\bigbreak

Il semble donc que la durée de vie des produits diminuent. Ceci pourrait être dû à la volonté des industriels d'augmenter leur chiffre d'affaire. Une solution pour arriver à cette finalité serait de détériorer la qualité des produits. Ce cadre correspond  bien à une définition de l'obsolescence programmée : la réduction de la durée vie volontaire d'un objet, afin de pouvoir le revendre plus fréquemment. 
