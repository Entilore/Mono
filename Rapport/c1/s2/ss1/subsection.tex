\subsection{Les difficultés de la définition}

De part sa nature discrète, si ce n'est pas secrète, l'obsolescence programmée n'a pas d'existence légale. Ce vide juridique, qui dure depuis bientôt cinquante ans, est lié au faite que des lois existantes condamnent déjà les entreprises dans le cas d'une durée de vie anormalement courte de leur produit. Ces lois sont celles relatives à la protection du consomateur en cas de vice caché du produit. Il faut savoir que la définition de vice caché est très floue, et très compliqué à prouver devant un tribunal.

Car le grand problème ici est de déterminer si le concepteur est de bonne fois. Ce défault, qui raccourci la durée de vie du produit à un ans, est-il un choix délibéré ou une simple négligeance du fabriquant ? Et dans le cas où le fabriquant aurait dimentionné son produit afin qu'il dépasse pas la garantie légale (qui est de 365 jours au moment où j'écrit ces lignes), est-ce là une faute aux yeux de la loi ?

Le bon sens nous apporte une réponse évidente : un fabriquant ne peut pas, pour tout ses produits, obtenir une durée de vie illimité. Il est donc obligé de définir une durée de vie moyenne. Celle-ci doit-elle dépasser la garantie légale de fonctionement ? Assurément ! Mais au delà ?
\medbreack
On peut donc associer cette réticence à légiférer à la construction du système capitaliste, qui laisse le champs libre aux entreprises dans la conception de leurs produits. 
Il est donc très difficile dans ce contexte légal de trouver une définition claire et concise de l'\OP.
