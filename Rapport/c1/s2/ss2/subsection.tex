\subsection{L'innovation, moteur de l'obsolescence}
Recherchée, acclammée même, dans la société moderne, l'innovation n'en est pas moins une cause majeur et un des vecteurs de l'obsolescence dans le sens le plus général du terme. 
\smallbreak
En effet, de même que le réfrigirateur mis fin au commerce de la glace et qu'internet causa la mort du minitel, les innovations techniques font une concurence féroce à l'usure naturelle des biens dans le registre de l'obsolescence. 
\medbreak
Cepandant le terme d'\textbf{innovation} peut sembler incompatible avec le terme \textbf{programmé}, car comment programmer l'innovation, qui n'existe pas encore ?
\smallbreak
Malgré tout, parmis les nombreux moteurs de l'\op, dont quelques-un seront détaillé par la suite, l'innovation est l'un des plus insidieux.
En effet, qui ira reprocher à une entreprise d'innover ? Pourtant cette même inovation est au cœur de la problématique de l'obsolescence, car elle crée de nouveaux usages et par sa simple existence rend obsolete certains objets.
