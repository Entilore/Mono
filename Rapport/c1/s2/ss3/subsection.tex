\subsection{Un épouvantail politique}
L'\op est depuis longtemps déjà un sujet brulant en politique. Régulièrement remise sur le tapis de l'Assemblée, comme récemment avec la loi Hamon, elle est notament défendu par les Verts.
\smallbreak L'accroc vient du faite que le code de la consomation contient déjà une cause qui peu s'apparenter à de l'anti-\op.
Cette clause est celle qui vise à protéger le consomateur en cas de vice caché du produit. Elle a l'avantage de ne pas blamer spécifiquement le fabriquant, car on ne condamne que la durée de vie anormalement courte du produit, et non spécifiquement un comportement particulier et/ou une volonté particulière de la part du concepteur du produit.

\smallbreak C'est cette volonté qu'il est très difficile de prouver. De façon à éviter cet écueil, il a souvent été proposé de remonter la durée de garantie minimal des produits. Cette garantie, qui est actuellement d'un an pour la grande majorité des produits, est le seul moyen pour l'état d'agir efficacement sur la qualitée réel des biens de consommation fabriqués.



[insérer ici un tableau des garanties réglementaires en fonction des types de biens de consommation]
