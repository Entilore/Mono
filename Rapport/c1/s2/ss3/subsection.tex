\subsection{la pression des acteurs prônant le respect de l'environnement }

Depuis la diffusion du reportage \textit{Prêt à Jeter} sur Arte, la notion d'\op est connue de tous. 

Le nombre de victimes potentielles de l'\op doit avoisiner les 100\% de la France. En effet, à partir du moment où une population est en contact avec des objets issus de la mondialisation, il y a une possibilité désuétude planifiée. 
Parmi les fervents dénonciateurs de l'\op, l'un des acteurs les plus virulent est le groupe des écologistes. 

\begin{wrapfigure}{r}{0.3\textwidth}
\vspace{0.1cm}\og  \textit{Les produits que nous utilisons dans la vie quotidienne sont trop souvent programmés par le producteur pour ne plus fonctionner après un certain nombre d'utilisations. Ces pratiques sont néfastes pour l'environnement et pèsent sur le pouvoir d'achat des ménages} \fg{} \caption{Eric Alauzet, Denis Baupin et Cécile Duflot, Septembre 2014}
\end{wrapfigure}

Dans ce groupement, on peut bien sûr inclure les partis politiques écologistes  comme \textit{Europe Ecologie Les Verts}, mais aussi les associations environnementales. Ainsi, l'association \textit{Les Amis de la Terre} s'est emparée du sujet depuis 2010. 

Ils ont obtenu que l'interdiction de l'\op soit ajouté dans le projet de loi sur la transition énergétique, qui a été adoptée par l'assemblée le 14 octobre 2014 en première lecture. Si le projet est accepté par le sénat en février 2014, un créateur de désuétude programmée pourra être condamné à deux ans de prison, et jusqu'à 300~000\euro~d'amende. 

Le but principal de ce groupe est de réduire les déchets. En effet, l'\op produit un énorme gaspillage de matières premières, et contribue à la pollution de la Terre. Nous reviendrons sur ce point dans le chapitre \ref{chapter::pb_duree_vie_reduite}.

\medbreak
Le client est également représentatif parmi les opposants à l'\op. Il est  représenté par les associations de consommateur comme \textit{UFC Que Choisir} qui ont publié un article \cite{ufc_OP} sur le sujet. Même si leurs tests \og \textit{n'ont jamais révélé d’obsolescence vraiment "programmée" }  \fg{}  , le groupe affirme dans l'article \textit{Obsolescence programmée : Trop de produits à durée de vie limitée } que les produits semblent être plus fragiles qu'avant.


L'association de consommateurs du \textit{Centre Européen de la Consommation} (CEC) est allée plus loin dans ses recherches sur l'\op. Ils ont publié en 2013 un rapport intitulé \textit{L'Obsolescence Programmée ou les Dérives de la Société de Consommation} où ils affirment clairement que l'\op existe, et que le consommateur en est l'une des victimes, tandis que l'environnement en est l'autre. 

\medbreak

De nombreux exemples d'appareils tombant en panne quelque temps après la fin de la garantie regorgent. Devant le nombre de cas suspectés par le consommateur et les nombreux acteurs qui dénoncent cette pratique, l'hypothèse de l'existence de l'\op semble fondée. 

\bigbreak
Cependant, comme M. Delaigue nous l'expliquait dans l'interview qu'il nous a accordée, l'\op peut simplement être une illusion. Ainsi les vieux fours d'il y a 20 ans ne sont pas forcément tous immortels, mais c'est l'idée que s'en ferait le consommateur : les vieux congélateurs qui sont observables de nos jours ne sont que les \textit{survivants} de la gamme de produit de l'époque, et pas une généralité comme il est facile de penser. 


Voyons maintenant les arguments qui mènent à penser que l'\op n'est qu'une illusion. 
