\subsection{Psychologique ou technologique}

l’obsolescence, non intentionnelle ou programmée peut-être ensuite psychologique ou technologique. 

\bigbreak
L’obsolescence psychologique est souvent la conséquence d’effet de mode. Le produit est alors remplacé alors qu’il est encore totalement fonctionnel. Autre dimension de l’obsolescence psychologique : pousser le consommateur à remplacer un produit par un autre plus respectueux de l’environnement.
Ce que le consommateur ignore, c’est que les déchets de production sont souvent bien plus important que ceux générés durant sa durée de vie. Encore une fois, il y remplacement d’un produit jugé non respectueux de l’environnement pour un autre qui ne l’est souvent pas si l’on inclut son cycle de production. Le consommateur achète un produit dont il pense avoir besoin alors que ce n'est pas le cas.

\medbreak
L’obsolescence technologique englobe quant à elle toutes les obsolescences d’ordre logiciel ou matériel, ainsi que la politique de support du vendeur. 

On peut distinguer :
\begin{itemize}
  \item les défauts de fonctionnement, par l’utilisation de composants peu robustes
  \item l’obsolescence indirecte, de par l'impossibilité de remplacer un accessoire ou par l’impossibilité de remplacer ou réparer l’une de ses pièces
  \item l’obsolescence d’incompatibilité entre générations de produits.
\end{itemize}


\bigbreak
Dans le cas du support utilisateur, si le vendeur décide de ne plus assurer le service après vente d’un produit de génération n, pour pousser à l’achat du produit de génération n+1, on peut évidement parler d'obsolescence programmée fonctionnelle.