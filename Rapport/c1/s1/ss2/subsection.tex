\subsection{Les preuves : cartel et actions judiciaires}


Nous disions plus tôt que l'\op est très difficile à prouver. En effet, aucun laboratoire de recherche spécialement appliqué à l'obsolescence programmée n'a jamais été trouvé, aucune preuve parfaite n'existe. Cependant, quelques exemples semblent prouver l'existence de la désuétude planifiée. S'ils est impossible de savoir s'il s'agit d'une norme ou d'une marginalité, les cas que nous allons développer prouve que certains constructeurs ont effectivement réduit la qualité de leur produit dans l'espoir d'engendrer plus de profit. 

\paragraph*{Le cartel Phœbus : } L' exemple du cartel Phœbus est un exemple fondateur : c'est le plus vieil exemple connu d'obsolescence programmée durant l'époque moderne. Son histoire est donc bien connue. 

Le cartel Phœbus désigne le rassemblement des producteurs d'ampoules, qui se sont arrangés en 1924 pour réduire la durée de vie des lampes à filament. 

Avant cette date, les producteurs étaient capable de produire des ampoules qui tenaient 2300 heures en moyenne \cite{opes_PHOEBUS}. La célèbre ampoule de Livermore\footnote{http://www.centennialbulb.org/cam.htm} témoigne de cette longévité : il brille depuis plus de 100 ans. 

Cette durée de vie n'était pas une bonne solution pour les fabricants d'ampoules. Afin d'augmenter leur vente,  les principaux fabricants (General Electric, Osram, La Compagnie des Lampes, \dots) se réunirent en une réunion secrète, lors de la fête de Noël de 1924, pour organiser la  réduction de la durée de vie des ampoules. L'objectif fût atteint  en 1940, soit 15 ans après la première réunion. 

Ce cartel a été poursuivi en justice en 1942, et condamné 11 ans plus tard. Cependant, comme le souligne \textit{Bon pour la casse}, la règle des 1000 heures ne fût pas abrogée. La société est-allemande \textit{Narva}, qui créait sous l'URSS des ampoules longues durées, n'arrivera jamais à les commercialiser après la chute du mur de Berlin. 

\paragraph*{l'iPod et sa batterie : } Les iPods de première, deuxième et troisième génération sont tout trois soumis à l'obsolescence programmée. En effet, si l'appareil dans l'ensemble est solide, la batterie ne pourrait pas tenir plus de 18 mois\cite{cec-zevRapportObsProg}. De plus, la firme ne propose aucune solution de remplacement de cette batterie, mais plutôt de remplacer l'appareil en entier. 

\smallskip
Beaucoup de critiques ont été faites vis à vis de la politique de réparation d'Apple. La firme  rendraient la réparation de leur produits quasiment impossible. Par exemple, la batterie est collée au reste de l'appareil dans l'iPhone 3. Pour changer la batterie, il faut donc réussir à décoller l'ancienne sans abimer le reste de l'appareil.  

Certains clients ont essayé d'aller faire réparer leurs appareil dans un centre officiel. A chaque fois, il leur a été répondu qu'ils devront en acheter un autre. Cette réponse montre bien que l'entreprise souhaite clairement obliger les usagers à remplacer leur produit. 
\smallskip


En 2003, Elizabeth Pritzker, une avocate américaine, réuni les victimes de la politique d'Apple, et porte plainte contre la multinationale. L'affaire portera le nom d'un des principaux plaignants : Andrew Westley. Le groupe obtiendra gain de cause assez rapidement. L'affaire fût réglée à l'amiable, chacun des plaignants a reçu 50\$ en bon d'achat pour un nouvel appareil, et Apple a étendu la durée de garantie à deux ans. 

\paragraph*{Les imprimantes : cartouches et compteur d'impression}

Le fil rouge du reportage \textit{Prêt à jeter} représente un informaticien donc l'imprimante connait une panne. La machine, de marque Epson, indique qu'elle ne fonctionne plus. Il décide donc d'essayer de la réparer. Pour ce faire, il rencontre plusieurs réparateurs ; tous lui conseillent d'acheter une nouvelle imprimante. En effet, le prix de la réparation est plus élevé que celui d'une imprimante neuve. 

Cependant, l'informaticien refuse de céder à la solution de facilité, et parcourt internet pour trouver une solution à son problème. Il se rend alors compte que l'erreur serait due à un compteur d'impression intégré dans l'engin. 

Après une longue recherche, il finit par trouver un logiciel, développé par un russe, Vitaliy Kiselev , qui réinitialise le compteur. Cette solution est efficace, et l'imprimante fonctionne à nouveau.


Le cas des compteurs d'impression est le cas le plus flagrant d'obsolescence programmée. La panne de la machine vient d'une puce, ajouté par le fabricant, qui semble limiter la durée dans le temps de la vie du produit ; sa ré-initialisation répare l'imprimante. La panne est bien créé par le fabricant, pour limiter la longévité de la vie des imprimantes.

Cet exemple est pointé du doigt par le parlement européen depuis 2002, et interdit par la directive \textit{2002/95/CE}. 
\smallbreak

Les compteurs d'impressions ne sont pas les seuls exemples d'\op au niveau des imprimantes. D'après \textit{GreenIT}\cite{greenit_cartouche_encre}, une association s'intéressant au développement durable  dans le domaine des  technologies de l'information, les imprimantes demanderaient à l'utilisateur de remplacer les cartouches bien avant la fin réelle du réservoir ( de 9\% à 45\% selon les marques).


\medbreak

