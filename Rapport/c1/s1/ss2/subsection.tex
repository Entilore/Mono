\subsection{cartels}



Comme déjà dit plus haut, l'obsolescence programmée est très difficile à prouver. Il n'existe que quelques rare cas d'obsolescence prouvée. 

\paragraph*{Le cartel Phoebus : } L' exemple du cartel Phoebus est un exemple fondateur : c'est le plus vieil exemple connu d'obsolescence programmée dans l'époque moderne. Son histoire est donc bien connue. 

Le cartel Phoebus désigne le rassemblement des producteurs d'ampoules, qui se sont arrangés en 1924 pour que réduire la durée de vie des ampoules. 

Avant cette date, les producteurs étaient capable de produire des ampoules qui tenaient 2300 heures en moyenne \cite{opes_PHOEBUS}. La célèbre ampoule de Livermore\footnote{http://www.centennialbulb.org/cam.htm} témoigne de sa longévité : il brille depuis plus de 100 ans. 

Cette longévité n'était pas une bonne solution pour les fabricants d'ampoules. Affin d'augmenter leur chiffre d'affaire,  les principaux fabricants (General Electric, Osram, La Compagnie des Lampes, \dots) se réunissent à Noël 1924 pour réduire la durée de vie d'une ampoule. L'objectif sera atteint  en 1940, soit 15 ans après la première réunion. 

Ce cartel fût mené en procès en 1942, et condamné 11 ans plus tard. Cependant, comme le souligne \textit{Bon pour la casse}, la règle des 1000 heures ne fût pas abrogée. La société allemande Narva, créant sous l'URSS des ampoules longues durées, n'arrivera jamais à les commercialiser hors du domaine bolchevique. 

