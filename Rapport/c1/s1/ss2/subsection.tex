\subsection{cartels}



Comme déjà dit plus haut, l'obsolescence programmée est très difficile à prouver. Il n'existe que quelques rare cas d'obsolescence prouvée. 

\paragraph*{Le cartel Phoebus : } L' exemple du cartel Phoebus est un exemple fondateur : c'est le plus vieil exemple connu d'obsolescence programmée dans l'époque moderne. Son histoire est donc bien connue. 

Le cartel Phoebus désigne le rassemblement des producteurs d'ampoules, qui se sont arrangés en 1924 pour réduire la durée de vie des ampoules. 

Avant cette date, les producteurs étaient capable de produire des ampoules qui tenaient 2300 heures en moyenne \cite{opes_PHOEBUS}. La célèbre ampoule de Livermore\footnote{http://www.centennialbulb.org/cam.htm} témoigne de sa longévité : il brille depuis plus de 100 ans. 

Cette longévité n'était pas une bonne solution pour les fabricants d'ampoules. Afin d'augmenter la vente des ampoules,  les principaux fabricants (General Electric, Osram, La Compagnie des Lampes, \dots) se réunissent à Noël 1924 pour réduire la durée de vie d'une ampoule. L'objectif sera atteint  en 1940, soit 15 ans après la première réunion. 

Ce cartel fût poursuivi en justice en 1942, et condamné 11 ans plus tard. Cependant, comme le souligne \textit{Bon pour la casse}, la règle des 1000 heures ne fût pas abrogée. La société est-allemande \textit{Narva}, créant sous l'URSS des ampoules longues durées, n'arrivera jamais à les commercialiser hors du domaine bolchevique. 

\paragraph*{l'iPod et sa batterie : } Les iPods de première, deuxième et troisième génération sont tout trois soumis à l'obsolescence programmée. En effet, si l'appareil dans l'ensemble est solide, la batterie ne pourrait pas tenir plus de 18 mois. De plus, la firme ne propose aucune solution de remplacement de cette batterie, mais plutôt de remplacer l'appareil en entier. 

\smallskip
Beaucoup de critiques ont été faite vis à vis de la politique de réparation d'Apple. Il rendraient la réparation de leur produits quasi-impossible. Par exemple, la batterie est collée au reste de l'appareil dans l'iPhone 3. Pour changer la batterie, il faut donc réussir à décoller l'ancienne sans abimer le reste de l'appareil.  

Certains clients ont essayé d'aller faire réparer leurs appareil dans un centre officiel. A chaque fois, on leur a répondu qu'ils devront en acheter un autre. Cette réponse montre bien que l'entreprise souhaite clairement obliger les usagers à remplacer leur produit. 
\smallskip


En 2003, Elizabeth Pritzker, une avocate américaine, réuni les victimes de la politique d'Apple, et porte plainte contre la multinationale. L'affaire portera le nom d'un des principaux plaignants : Andrew Westley. Le groupe obtiendra gain de cause assez rapidement. L'affaire fût réglée à l'amiable, chacun des plaignants a pu recevoir 50\$ en bon d'achat pour un nouvel appareil. 

\paragraph*{Les compteurs d'impression}

Le fil rouge du reportage \textit{Prêt à jeter} représente un informaticien donc l'imprimante ne fonctionne plus. La machine, de marque HP, indique qu'elle ne fonctionne plus. Il décide donc d'essayer de la réparer. Pour ce faire, il rencontre plusieurs réparateurs ; tous lui conseillent d'acheter une nouvelle imprimante. En effet, le prix de la réparation est plus élevé que celui d'une imprimante neuve. 

Cependant, l'informaticien refuse de céder à la solution de facilité, et recherche une solution sur internet. Il se rend alors compte que l'erreur serait due à un compteur d'impression intégré dans l'engin. 

Après une longue recherche, il finit par trouver un logiciel, développé par un russe, qui réinitialise le compteur. Cette solution est efficace, et l'imprimante fonctionne à nouveau.

\medbreak

Autre