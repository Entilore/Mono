\subsection{cartels}



Comme déjà dit plus haut, l'obsolescence programmée est très difficile à prouver. Il n'existe que quelques rare cas d'obsolescence prouvée. 

\paragraph*{Le cartel Phoebus : } L' exemple du cartel Phoebus est un exemple fondateur : c'est le plus vieil exemple connu d'obsolescence programmée dans l'époque moderne. Son histoire est donc bien connue. 

Le cartel Phoebus désigne le rassemblement des producteurs d'ampoules, qui se sont arrangés en 1924 pour que réduire la durée de vie des ampoules. 

Avant cette date, les producteurs étaient capable de produire des ampoules qui tenaient 2300 heures en moyenne \cite{opes_PHOEBUS}. La célèbre ampoule de Livermore\footnote{http://www.centennialbulb.org/cam.htm} témoigne de sa longévité : il brille depuis plus de 100 ans. 

Cette longévité n'était pas une bonne solution pour les fabricants d'ampoules. Affin d'augmenter leur chiffre d'affaire,  les principaux fabricants (General Electric, Osram, La Compagnie des Lampes, \dots) se réunissent à Noël 1924 pour réduire la durée de vie d'une ampoule. L'objectif sera atteint  en 1940, soit 15 ans après la première réunion. 

Ce cartel fût mené en procès en 1942, et condamné 11 ans plus tard. Cependant, comme le souligne \textit{Bon pour la casse}, la règle des 1000 heures ne fût pas abrogée. La société allemande Narva, créant sous l'URSS des ampoules longues durées, n'arrivera jamais à les commercialiser hors du domaine bolchevique. 

\paragraph*{l'iPod et sa batterie : } Les iPods de première, deuxième et troisième génération sont tout trois soumis à l'obsolescence programmée. En effet, si l'appareil dans l'ensemble est solide, la batterie ne pourrait pas tenir plus de 18 mois. De plus, la firme ne propose aucune solution de remplacement de cette batterie, mais plutôt de remplacer l'appareil en entier. 

\smallskip
Beaucoup de critiques ont été faite vis à vis de la politique de réparation d'Apple. Il rendraient la réparation de leur produits quasi-impossible. Par exemple, la batterie est collée au reste de l'appareil dans l'iPhone 3. Cela rend le remplacement impossible d'une batterie. 

Certains clients ont essayé d'aller faire réparer leurs appareil dans un centre officiel. A chaque fois, on leur a répondu qu'ils devront en acheter un autre. Cette réponse montre bien que l'entreprise souhaite clairement obliger les usagers à remplacer leur produit. 
\smallskip


En 2003, Elizabeth Pritzker, une avocate américaine, réuni les victimes de la politique d'Apple, et porte plainte contre la multinationale. L'affaire portera le nom d'un des principaux plaignants : Andrew Westley. 

Le groupe obtiendra gain de cause assez rapidement. L'affaire fût réglée à l'amiable, chacun des plaignants a pu recevoir 50\$ en bon d'achat pour un nouvel appareil. 