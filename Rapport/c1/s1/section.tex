\section{L'\op, une réalité ?}

L'idée que l'\op existe est facilement soutenue. C'est le parti pris par(voir tous) de nombreux (écologistes par exemple.
\smallbreak
L'exemple le plus flagrant est sans doute celui révélé par l'émission  \textit{Prêt à jeter}, diffusée sur \textit{Arte} : une puce électronique implantée dans certaines imprimantes qui permet de bloquer la machine après un certain nombre d'impression. Pour dépanner l'appareil, les journalistes utilisent un logiciel permettant de réinitialiser la puce. 

Cet exemple est bien un cas d'obsolescence programmée : en effet c'est bien le constructeur qui a implémenté une fonction permettant de durer la durée de vie du produit. 

Dans cette partie, nous aborderons différents points qui confortent l'idée de l’existence de l'\op.