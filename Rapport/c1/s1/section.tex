\section{L'\op, une réalité ?}

L'idée que l'\op existe est facilement soutenue. C'est le parti pris de nombreux groupements écologistes par exemple.
\smallbreak
L'exemple le plus flagrant est sans doute celui révélé par l'émission  \textit{Prêt à jeter}, diffusée sur \textit{Arte} en 2011 et rediffusé plusieurs fois depuis : une puce électronique implantée dans certaines imprimantes qui permet de bloquer la machine après un certain nombre d'impression. Pour dépanner l'appareil, le journaliste utilise un logiciel trouvé sur internet permettant de réinitialiser la puce, et donc de relancer l'impression. 

Cet exemple est bien un cas d'obsolescence programmée : en effet c'est bien le constructeur qui a implémenté une fonction permettant de contrôler la durée de vie du produit. 

C'est pourquoi qdns cette partie, nous aborderons différents points qui confortent l'idée de l’existence de l'\op.
