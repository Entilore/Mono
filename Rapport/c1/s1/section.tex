\section{L'\op, une réalité ?}

Nicols Fox, essayiste et romancière, est sûr que l'\op existe, et elle s'y oppose en écrivant des sonnets de complainte : 
\itshape\begin{center}
\begin{verse}
Nonobstant, cette boîte déborde d'innombrables

Lampes de poche inutilisables

Irréparables, presque à usage unique

Leur prompte relégation\\
M'apparait bien inique

Est-ce donc trop demander

Que d'avoir à portée\\
Des lampes qui puissent durer ?
\end{verse}
\end{center}
\normalfont
Comme elle, de nombreux utilisateurs sont exaspérés de la facilité qu'ont les objets à casser. Certains en ont pour leur argent, d'autre y voit plutôt le côté polluant. 

\smallbreak

Aucune preuve parfaite n'a jamais été trouvé pour affirmer que l'\op existe. L'argumentation des défenseur de l'existence de l'\op est cependant solide. Quelques entreprises ont été trainées en justice, quelques coïncidences sont trop grosse pour être du hasard. 