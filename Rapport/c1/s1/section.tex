\section{Ce qui conforte l’idée de son existence}

L'idée que l'\op existe est facilement soutenue. C'est la thèse soutenue par de nombreux écologistes, pour ne citer que ce groupe politique. 
\smallbreak
L'exemple le plus flagrant est sans doute celui révélé par l'émission  \textit{Prêt à jeter}, diffusée sur \textit{Arte} : une puce électronique implantée dans certaines imprimantes qui permettent de bloquer la machine après un certain nombre d'impression. Pour dépanner l'appareil, les journalistes utilisent un logiciel, qui va réinitialiser la puce. 

Cet exemple est bien un cas d'obsolescence programmée : en effet c'est bien le constructeur qui a implémenté une fonction permettant de durer la durée de vie du produit. 

\smallbreak

Dans cette partie, nous aborderons différents points qui confortent l'idée de l’existence de l'\op 