\subsection{Quelques exemples troublants}


\paragraph*{Obsolescences technologiques}

\subparagraph*{Obsolescence d’incompatibilité logiciel :}

On observe sur le graphe \ref{compSamsIph}, qui représente les recherches sur Google "Mon téléphone de marque X est lent", que les utilisateurs d'iPhone voient leur téléphone ralentir juste avant la sortie de la génération suivante (ce qui correspond aux pics sur le graphique). Ce qui n'est pas le cas chez Samsung qui ne gère pas directement les mises à jour logicielles d'Android.

\begin{figure}[h]

\begin{minipage}{0.5\linewidth}
\includegraphics[scale=0.25]{Rsc/searchForIphoneSlow.png} 
\end{minipage}
\begin{minipage}{0.5\linewidth}
\includegraphics[scale=0.25]{Rsc/searchForSamsungSlow.png} 
\end{minipage}
\caption{Comparaison entre les téléphones Samsung et iPhone}
\label{compSamsIph}
\end{figure}

\subparagraph*{Obsolescence indirecte :}

Malgré la norme USB imposée aux fabricants de smartphones par l’Union Européenne, Apple a toujours utilisé ses propres connecteurs, et ne manque pas de changer la forme du connecteur d’une génération à une autre, exigeant l'achat d'adaptateurs et brisant la compatibilité avec de nombreux produits dérivés.

\subparagraph*{Obsolescence de fonctionnement :}

Les batteries d’iPod étaient irremplaçables pour les 1ere, 2eme et 3eme générations. Un procès a été intenté par Elizabeth Pritzyker, ce qui a conduit Apple à mettre en place une politique de remplacement des batteries usées.

\subparagraph*{Obsolescence de service après-vente :}

Le 8 avril 2014, Microsoft annonce la fin du service après-vente du système d’exploitation Windows XP, encore très présent en entreprise. Que ce soit pour pousser au rachat d’un système de génération plus récente ou pour se décharger de la responsabilité de certaines failles, cette décision met en péril la sécurité de plus d'un terminal sur 5 dans le monde. Aux États-Unis 95\% des distributeurs de billets du pays fonctionnaient encore sous XP en janvier 2014.


\paragraph*{Obsolescence psychologique :}

Grâce à l’extrait d’analyse du cycle de vie fourni à GreenIT.fr par Fujitsu, on apprend que le bilan carbone de production peut être jusqu’à 70 fois supérieur au bilan carbone d’utilisation pour un ordinateur fabriqué en Asie. Il faut prendre en compte le fait que la Chine utilise abondamment le charbon dans sa production d'électricité. Pour cette raison il est important de mesurer les arguments écologiques des produits mis en vente, avant de remplacer un appareil qui fonctionne encore. De même les effets de mode poussent à remplacer des produits totalement fonctionnels.

\newpage
