\subsection{la pression des acteurs prônant le respect de l'environnement }
Depuis la diffusion du reportage \textit{Prêt à Jeter} sur Arte, la notion d'\op est connue de tous. 
Les victimes potentielles de l'\op doivent avoisiner les 100\% des français. En effet, à partir du moment où une population est en contact avec des objets issus de la mondialisation, il y a une possibilité d'\op. 
Parmi ces acteurs, l'un dont les actions sont les plus visibles, du moins actuellement, est les écologistes. 
\begin{wrapfigure}{r}{0.3\textwidth}
\textit{Les produits que nous utilisons dans la vie quotidienne sont trop souvent programmés par le producteur pour ne plus fonctionner après un certain nombre d'utilisations. Ces pratiques sont néfastes pour l'environnement et pèsent sur le pouvoir d'achat des ménages}, Eric Alauzet, Denis Baupin et Cécile Duflot, Septembre 2014.
\end{wrapfigure}
Ils ont obtenu cette année que l'\op soit interdite par la loi. Ainsi depuis le 14 octobre 2014, un créateur de désuétude programmée peut être condamné à deux ans de prison, et jusqu'à 300~000\euro d'amende. 

Dans ce groupement, on peut bien sûr inclure les partis politiques écologistes  comme \textit{Europe Ecologie Les Verts}, mais aussi les associations environnementales. Ainsi, l'association \textit{Les Amis de la Terre} s'est emparé du sujet depuis 2010. 

Le but principal de ce groupe est de réduire les déchets. En effet, l'\op produit un énorme gaspillage de matière première, et contribue à la pollution de la Terre. Nous reviendrons sur ce point dans le chapitre 2.

\medbreak

Un autre groupe d'acteur important est les consommateurs, représenté par les associations comme \textit{UFC Que Choisir} qui, même si leurs tests \textit{n'ont jamais révélé d’obsolescence vraiment «programmée »},reconnait dans l'article \textit{Obsolescence programmée : Trop de produit à durée de vie limitée } que les produits semblent être plus fragile qu'avant. 

L'association de consommateur du \textit{Centre Européen de la Consommation} (CEC) est allé plus loin dans ses recherche sur l'\op. Ils ont publié en 2013 un rapport  intitulé \textit{L'Obsolescence Programmée ou les Dérives de la Société de Consommation} où ils affirment clairement que l'\op existe, et que le consommateur en est l'une des victime, tandis que l'environnement en est l'autre. 

\medbreak

Ainsi, de nombreux acteurs affirment que l'\op existe bel et bien. De plus, de nombreux exemples d'appareils tombant en panne quelque temps après la fin de la garantie regorgent. Devant le nombre de cas suspectés par le consommateur et les nombreux acteurs qui dénoncent cette pratique, il semble fondé de supposer que l'\op existe. 

