\subsection{Des produits de moins en moins durable ?}

Une impression de diminution de la durabilité des produits du quotidien est souvent ressentie par les utilisateurs.  Qui ne s'est jamais étonné que le four de vingt ans des grands-parents soit toujours fonctionnel, alors qu'on vient de changer le sien, mort après 4 ou 5 années de bon et loyaux services. 

\smallbreak
Voici un exemple plus concret : la Ford T
Il s'agit d'une voiture produite par Ford et accessible au grand public dès 1908.
Elle est l'une des premiers victimes de l'\op.
C'était une voiture durable, conçue pour durer, le taux de pannes était donc très faible.   

Seulement, cette voiture était un modèle unique : elle était majoritairement produite en noir. Son esthétisme ne plaisait pas à tout le monde. De plus, elle était peu confortable, et pas forcément pratique à utiliser. Il fallait par exemple la démarrer à la manivelle. 

General Motors (\textit{GM}), souhaitant lui aussi avoir sa part de marché dans l'automobile, décide de créer sa voiture.
Devant la qualité de la Ford T, il semblait impossible à \textit{GM} d'inventer un modèle capable de surpasser ce modèle, phare de l'époque.
Il décidèrent ainsi de construire une automobile moins fiable, mais plus esthétique, et plus pratique.
Certains effets de mode furent ainsi introduits : de nouveaux modèles étaient créés tous les ans, différents par leur apparence et par leur capacités.

Cependant, les créateur de ces voitures décidèrent de sacrifier la longévité. Ainsi, les voitures tombaient plus souvent en panne, cependant le consommateur en profitait pour en acheter une nouvelle.

Il semble que la volonté de Alfred Sloan, le dirigeant de \textit{GM}, était justement que chaque voiture dure juste assez longtemps pour que le client rembourse le prêt nécéssaire à l'achat initial.
Il peut ainsi en contracter un nouveau pour acheter la dernière auto sortie. Le but du directeur de l'entreprise automobile était de réduire encore plus la durée de vie de ses produits.

D'après le livre écrit par Serge Latouche, \textit{Bon Pour la Casse}\ref{bpc}, l'ambition finale serait une longévité d'un an.

\smallbreak
L'exemple des voiture de \textit{GM}est l'un des premiers cas de réduction volontaire de la durée de vie d'un produit. L'entreprise remportera la bataille du marché automobile, \testit{en privilégiant le design à la fiabilité}.
Ce nouveau modèle économique attira de nombreuses autres entreprises. 

D'après Serge Latouche \cite{bpc}, ce système aurait connu beaucoup de succès auprès d'autres industriels. A. Sloan présente en 1950 sa méthode à d'autres entreprises. On verra de nombreux produits de très courte durée de vie apparaître (bungalows, tente en papier, ... ). C'est également à ce moment là qu'apparaissent les produits \textit{jetables}.

Un autre exemple, peut être plus flagrant est celui des bas en nylon.
L'entreprise de chimie de \textit{Du Pont de Nemours} a créé dans les années 1940 des bas fait de nylon.
Ces bas étaient de très bonne qualité, et réputés pour ne pas filer. La société \textit{Du Pont de Nemours} remarqua cependant que les bas étaient trop solide. 
Ainsi, leur durée de vie ne permettaient pas à la compagnie de survivre.
D'après le site web \url{obsolescence-programmee.fr}, ils auraient volontairement détérioré la qualité de leur produit, de manière à améliorer leur marge, au détriment des consommateurs. 

\smallbreak

Il est donc possible que certains industriels altèrent la durabilité de produits afin d'augmenter leur chiffre d'affaire. Les exemples donnés datent de la moitié du XIXe siècle, cependant, ils sont toujours d'actualité. Une voiture ne dure en général que dix ans, et les bas en nylons sont toujours aussi facilement abimable. 

De plus, le modèle économique créé par A. Sloan n'a pas été abandonné. Ainsi, la durabilité de certains produits quotidien a clairement baissée depuis les 50 dernières années. 
Une étude \cite{opSsg}, réalisée en septembre 2010, montre que la durée de vie d'objets de la vie de tout les jours est belle et bein plus faible qu'auparavant.
Ainsi, un appareil électroménager aurait perdu quatre années d'espérance de vie(de 11 ans à 7 ans), une télévision près de 10 (dix à quinze ans pour une TV à tube cathodique, cinq ans pour un écran plat). 

Il semble donc, du moins du point de vue du consommateur, que la durée de vie des produits diminuent. Ceci pourrait être dû à la volonté des industriels d'augmenter leur chiffre d'affaire. Ceci peut être effectué par eux en détériorant la qualité des objets. Ceci est bien la définition de l'obsolescence programmée : la réduction volontaire de la durée de vie du produit. 
%\cite{etudedureedevie}
