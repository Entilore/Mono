\subsection{Les difficultés de la définition}

De part sa nature discrète, si ce n'est pas secrète, l'obsolescence proggrammé n'as pas d'existence légale. Ce vide juridique, qui dure depuis bientôt ciquante ans, est lié au faite que des lois existantes condamnent déjà les entreprises dans le cas d'une durée de vie anormalement courte de leur produit. Ces loi sont celles relatives à la protection du consomateur en cas de vice caché du produit. Il faut savoir que la définition de vice caché est très floue, et très compliqué à prouver devant un tribunal.

Un autre argument puvant expliquer cette réticence à légiférer est la construction du système capitaliste, qui laisse le champs libre aux entreprises dans la conception des produits.
