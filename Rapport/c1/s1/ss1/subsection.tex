\subsection{La baisse de la durabilité d'un produit en augmentation}

Une impression de diminution de la durabilité des produits est souvent ressentie par les utilisateurs.  Il n'est pas rare d'être surpris que le four de vingt années des grands-parents est toujours fonctionnel, alors que l'on vient de changer le notre, mort à l'aube de ses 8 ans. 

Une étude \cite{opSsg}, réalisée en septembre 2010, montre que la durée de vie de produits de la vie de tout les jours est plus faible qu'auparavant. Ainsi, un appareil électroménager aurait perdu quatre années d'espérance de vie(de 11 ans à 7 ans), une télévision près de 10 (dix à quinze ans pour une TV à tube cathodique, cinq année pour un écran plat). 
\smallbreak
La Ford T, une voiture produite par Ford accessible au grand publique dès 1908, est l'un des premiers victime de l'\op était une voiture durable, conçue pour durer.  C'est le premier modèle automobile à avoir été produit en série. Cette voiture était très fiable : elle tombait rarement en panne, durait longtemps. 

Seulement, cette voiture était un modèle unique : elle était majoritairement produite en noir. Son esthétisme ne plaisait pas à tout le monde. 

General Motors (\textit{GM}), souhaitant aussi avoir sa part de marché dans l'automobile, décide de créer sa voiture. Devant la qualité de la Ford T, il semblait impossible à \textit{GM} de créer un modèle capable de surpasser le modèle phare de l'époque. Il décidèrent ainsi de creer une automobile moins fiable, mais plus esthétique. De nouveaux types étaient créé tout les ans. Les voitures tombaient plus souvent en panne, cependant le consommateur en profitait pour en acheter une nouvelle, différente en design. 

Cet exemple est l'un des premiers cas de réduction volontaire de la durée de vie d'un produit. \textit{GM} remporta la bataille du marché automobile, en privilégiant design à fiabilité. Ce nouveau modèle économique attira de nombreuses autres entreprises. 


%\cite{etudedureedevie}
