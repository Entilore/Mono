\subsection{Programmée ou non intentionnelle}

L'obsolescence naturelle est donc un fait, une suite logique de l'innovation de nos technologies. Elle est la résultante de l'usure d'un ancien produit ayant été conçu pour durer.
Il faut bien distinguer l’obsolescence de l’usure. Un appareil obsolète n’est pas forcément défectueux. L’apparition d’internet a par exemple rendu complètement obsolète le Minitel, appareil qui n'était pas forcément défectueux lors de son remplacement par les premier ordinateurs connectés au web. 

\bigbreak
Cependant l'hypothèse d'obsolescence intentionnelle, programmée, subsiste. 
En effet l'acte de remplacement, voire de réparation si celui-ci est payant, permettrait à l'entreprise de multiplier ses prestations et donc ses bénéfices. 
On peut ainsi distinguer deux types d'obsolescence, l'obsolescence programmée, ou volonté de réduction de la durée de vie d’un produit et l'obsolescence non intentionnelle, comme dit plus haut, suite logique et justifiée d’une innovation.

\bigbreak
Inversement, si une entreprise déploie intentionnellement des fonctionnalités toutes créées à une date \textit{T}, sur différentes générations de produits pour pousser le consommateur à remplacer son produit à chaque génération, on peut considérer cette pratique comme étant de l'obsolescence programmée. Dans chaque suspicion d’obsolescence intentionnelle, il faut toutefois mesurer le propos dans le sens où cette stratégie peut-être indispensable à la survie de l’entreprise, par exemple pour amortir les coûts de recherche et développement ayant mené à ces différentes innovations. Il ne faut également pas écarter l’hypothèse d’une erreur de conception ou d'ingénierie.

\medbreak
L'\textit{obsolescence programmée} résulte donc de la création d'un "besoin" auprès des utilisateurs, à des fins économiques. Le fond du problème de la pratique de l'obsolescence programmée, outre les impacts environnementaux est la manipulation du consommateur, le dol lors de l'acte d'achat. Le consommateur achète un produit neuf pour remplacer le précédent, rendu intentionnellement obsolète par le vendeur, que ce soit par une phénomène de mode ou d'usure voulue de l’ancien appareil.

\bigbreak
On peut se demander pourquoi la pratique de l’obsolescence programmée au sein d’une entreprise ne la précipite pas vers la faillite, par la perte des clients mécontents. L’hypothèse est que le nombre d’entreprises pratiquant l’obsolescence programmée est important, ce qui pallie cette perte dans un premier temps. La durée de vie courte des produits deviendrait, dans un second temps, la nouvelle norme.
