\subsection{Programmé ou non intentionnelle}

L'obsolescence est donc un fait, une suite logique de l'innovation de nos technologies. Cependant l'hypothèse d'obsolescence intentionnelle, programmée, subsiste. En effet l'acte de remplacement, voire de réparation si celui-ci est payant, permet à l'entreprise de multiplier ses prestations et donc ses bénéfices. On peut distinguer donc deux types d'obsolescence, l'obsolescence programmée et l'obsolescence non intentionnelle, suite logique et justifié d’une innovation. Il faut bien distingué l’obsolescence de l’usure. Un appareil obsolète n’est pas forcément défectueux.

L'obsolescence non intentionnelle, ou "naturelle"est la résultante de l'usure d'un ancien produit ayant été conçu pour durer.  Une innovation peut aussi être une cause d'obsolescence, l’apparition d’internet à par exemple rend complètement obsolète le minitel, l’appareil n’est pas encore défectueux lors de son remplacement.

Cependant si une entreprise déploient intentionnellement des fonctionnalités toutes crées à une date t, sur différentes générations de produits pour pousser le consommateur à remplacer son produit à chaque génération, on peut considérer cette pratique comme étant de l'obsolescence programmée. Dans chaque suspicion d’obsolescence intentionnelle, il faut toutefois mesurer le propos dans le sens ou cette stratégie peut-être indispensable pour la survie de l’entreprise, par exemple pour amortir les coûts de recherche et développement. Il ne faut également pas écarter l’hypothèse d’une erreur d'ingénierie.

L'obsolescence programmée résulte donc de la création d'un “besoin” auprès des utilisateurs, à des fins économiques. Le fond du problème de la pratique de l'obsolescence programmée, outre les impacts environnementaux est la manipulation du consommateur et le dol lors de l'acte d'achat. le consommateur achète un produit neuf pour remplacer un appareil rendu obsolète par l'entreprise qui lui avait vendu le produit. Le consommateur se tournerait alors vers un concurrent.

L’hypothèse est que le nombre d’entreprises pratiquant l’obsolescence programmée est important, ce qui pallie la perte des clients mécontents dans un premier temps. La durée de vie courte des produits deviendrait, dans un second temps, la nouvelle norme.
