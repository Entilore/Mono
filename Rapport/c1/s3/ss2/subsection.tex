\subsection{L'innovation, moteur de l'obsolescence}
Recherchée, acclamée dans la société moderne, l'innovation n'en est pas moins un des vecteurs majeurs de l'existence même de l'obsolescence. Le terme d'\textbf{innovation} peut sembler incompatible avec le terme \textbf{programmé}, car comment programmer ce qui n'existe pas encore ?
\medbreak
Parmi les nombreux moteurs de l'\op, dont quelques-uns seront détaillés par la suite, l'innovation est l'un des plus insidieux. En effet, qui ira reprocher à une entreprise d'innover ?
Pourtant cette même innovation est au cœur de la problématique de l'obsolescence, car elle crée de nouveaux usages et par sa simple existence rend obsolète certains objets.
