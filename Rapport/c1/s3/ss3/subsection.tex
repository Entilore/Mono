\subsection{Un épouvantail politique}
L'\op est depuis longtemps déjà un sujet brûlant en politique. Régulièrement remise sur le tapis de l'Assemblée, comme récemment avec la loi Hamon, elle est notamment défendue par les groupement écologiques (Europe Écologie/ Les Verts) ainsi que par la gauche.
Pourtant, l'état légifère très peu dans ce domaine.

\smallbreak L'accroc vient du faite que le code de la consommation contient déjà une cause qui peut s'apparenter à de l'anti-\op.
Cette clause est celle qui vise à protéger le consommateur en cas de vice caché du produit. 
Elle a l'avantage de ne pas blâmer spécifiquement le fabriquant, car elle ne condamne que la durée de vie anormalement courte du produit, et non spécifiquement un comportement particulier et/ou une volonté particulière de la part du concepteur du produit.

% section à mettre en encart CLASS ACTION ?
\smallbreak Car c'est cette volonté qu'il est très difficile de prouver. En effet, on n'a encore jamais retrouvé de "bureau noir" chargé de saboter de quelque manière que ce soit le travail des ingénieurs. Et dans la majorité des cas, c'est à la personne qui a constaté le défaut de produire la preuve de ce qu'elle avance.
Au États-Unis, cette démarche est facilité par les \textit{Class Action}, de grands groupements de consommateurs qui se rassemblent de manière à attaquer une entreprise sur un contentieux précis. Cette opération juridique n'est cependant pas sans compromis, car elle peut apparaître comme trop facile.
Le coût, divisé entre un grand nombre de consommateur, est très bas et peut entraîner des abus.
De plus elle est très longue à aboutir, même en temps jurique. (JURIQUE ??????????????????????????????????????????????????????????????????)
Cette procédure n'existe pas en France.

\smallbreak De façon à éviter cette situation ou le consommateur est incapable de prouver quoi que ce soit, il a été proposé de remonter la durée de garantie minimal des produits. Cette garantie, qui est actuellement d'un an pour la grande majorité des produits, est une garantie légal qui est imposé aux fabricants par la loi. 
En voici quelques une;

\begin{center}
\begin{tabular}{|l|l|p{5cm}|}
  \hline
  \multicolumn{3}{|c|}{Tableau des garanties réglementaires gratuites pour le consommateur, décembre 2014} \\
	\hline Garantie commerciale & Biens de consommation & De 6 mois à 1 ou 2 ans ou même plus\\
	       Garantie légale de conformité & Tout biens & 2 ans \\
	       Garantie légale des vices cachés & Tout bien, neuf ou occasion & 2 ans \\
	\hline
\end{tabular}
\end{center}

% http://vosdroits.service-public.fr/particuliers/F11093.xhtml le 8/12/14
% http://www.economie.gouv.fr/dgccrf/Publications/Vie-pratique/Fiches-pratiques/Les-Garanties

\smallbreak Le risque posé par l'augmentation de cette garantie est le poids qu'elle fait peser sur le pouvoir d'achat de la population.
En période de crise économique, cette augmentation du prix des bien est évidemment inconcevable au niveau politique, et est probablement la raison du désintérêt législatif dans ce secteur.
