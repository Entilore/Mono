\subsection{Un épouvantail politique}
L'\op est depuis longtemps déjà un sujet brûlant en politique. Réguliè\-rement mentionné à l'Assemblée Nationale, comme récemment avec la loi Hamon, elle est notamment défendue par les groupements écologiques (Europe Écologie/ Les Verts) ainsi que par la gauche.
Pourtant, l'État légifère très peu dans ce domaine.
\medbreak 
L'accroc vient du fait que le code de la consommation contient déjà une clause qui peut s'apparenter à de l'anti-\op.
Cette clause est celle qui vise à protéger le consommateur en cas de vice caché du produit. 
Elle a l'avantage de ne pas blâmer spécifiquement le fabricant, car elle ne condamne que la durée de vie anormalement courte du produit, et non spécifiquement un comportement particulier et/ou une volonté particulière de la part du concepteur du produit.

% section à mettre en encart CLASS ACTION ?
\bigbreak Car c'est cette volonté qu'il est très difficile de prouver. En effet, on n'a encore jamais retrouvé de "bureau noir" chargé de saboter de quelque manière que ce soit le travail des ingénieurs. Et dans la majorité des cas, c'est à la personne qui a constaté le défaut de produire la preuve de ce qu'elle avance.
\medbreak
Au États-Unis, cette démarche est facilitée par les \textit{Class Action}, de grands groupements de consommateurs qui se rassemblent de manière à attaquer une entreprise sur un contentieux précis. Cette opération juridique n'est cependant pas sans compromis, car elle peut apparaître comme trop facile.
Le coût, divisé entre un grand nombre de consommateurs, est très bas et peut entraîner des abus.
De plus elle se compte généralement en dixaine de mois, alors que le cycle de vie des produits se raccourcis de plus en plus.
Cette procédure n'existe pas en France.

\medbreak De façon à éviter cette situation dans laquelle le consommateur est incapable de prouver quoi que ce soit, il a été proposé de remonter la durée de garantie minimale des produits. Cette garantie, qui est actuellement d'un an pour la grande majorité des produits, est une garantie légale qui est imposée aux fabricants par la loi. 
En voici quelques unes;

\begin{center}
\begin{tabular}{|l|l|p{5cm}|}
  \hline
  \multicolumn{3}{|c|}{Tableau des garanties réglementaires gratuites pour le consommateur, décembre 2014} \\
	\hline Garantie commerciale & Biens de consommation & De 6 mois à 2 ans plus\\
	       Garantie légale de conformité & Tout biens & 2 ans \\
	       Garantie légale des vices cachés & Neuf ou occasion & 2 ans \\
	\hline
\end{tabular}
\end{center}

% http://vosdroits.service-public.fr/particuliers/F11093.xhtml le 8/12/14
% http://www.economie.gouv.fr/dgccrf/Publications/Vie-pratique/Fiches-pratiques/Les-Garanties

\smallbreak Le risque posé par l'augmentation de cette garantie est le poids qu'elle fait peser sur le pouvoir d'achat de la population.
En période de crise économique, cette augmentation du prix des biens est évidemment inconcevable au niveau politique, et est probablement la raison du désintérêt législatif dans ce secteur.
