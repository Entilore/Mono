\subsection{Les difficultés de la définition}

De part sa nature discrète, si ce n'est pas secrète, l'obsolescence programmée n'a pas d'existence légale. Ce vide juridique, qui dure depuis bientôt cinquante ans, est lié au fait que des lois existantes condamnent déjà les entreprises dans le cas d'une durée de vie anormalement courte de leur produit. Ces lois sont celles relatives à la protection du consommateur en cas de vice caché du produit. Il faut savoir que la définition de vice caché est très floue, et très compliqué à prouver devant un tribunal.
\medbreak
Car le grand problème ici est de déterminer si le concepteur est de bonne foi. Ce défaut, qui raccourcit la durée de vie du produit à un an, est-il un choix délibéré ou une simple négligence du fabriquant ? Et dans le cas où le fabriquant aurait dimensionné son produit afin qu'il ne dépasse pas la garantie légale (qui est de 365 jours au moment où j'écris ces lignes), est-ce là une faute aux yeux de la loi ?

Le bon sens nous apporte une réponse évidente : un fabriquant ne peut pas, pour tous ses produits, obtenir une durée de vie illimitée. Il est donc obligé de définir une durée de vie moyenne. Celle-ci doit-elle dépasser la garantie légale de fonctionnement ? Assurément ! Mais au delà ?
\bigbreak
On peut donc résumer cette réticence à légiférer comme étant la construction du système capitaliste, qui laisse le champ libre aux entreprises dans la conception des produits.
