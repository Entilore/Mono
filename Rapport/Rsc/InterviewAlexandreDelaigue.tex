\section{Alexandre Delaigue}
\label{InterviewADelaigue}

\vspace{2\baselineskip}

\begin{small}
\smallbreak\textbf{Commençons par votre définition de l'obsolescence programmée. Vous niez son existence, on suppose donc que c'est peut-être une question de définition qui serait différente que celle des autres.  Pourriez-vous donc nous définir ce concept ?
}\smallbreak

L'obsolescence programmée désigne une pratique consistant, de la part d'une entreprise, à réduire sciemment la durée de vie de ses produits pour inciter les gens à en acheter plus, puisqu'ils doivent les renouveler. C'est de cela dont on parle, et ce n'est pas autre chose. 

Pour bien préciser, cela ne veut pas dire que de temps en temps les entreprises ne font pas des produits de mauvaises qualités. Cela ne veut pas dire que de temps en temps les entreprises ne sont pas amenées à chercher à tromper le consommateur. C'est-à-dire que ce n'est pas ici l'idée que les entreprises font tout pour le bien de tout le monde, \textit{etc.} C'est que cette pratique bien spécifique entendue comme \og \textit{Je fais exprès que mon produit tombe en panne plus tôt, comme ça les clients  vont le racheter }\fg{}   est quelque chose qui n'existe pas pour une raison très simple : c'est que ça n'est pas quelque chose qui est rentable. On détaillera après,  mais voici une définition de l'\op.

\textbf{
Donc pour vous, vous ne prenez pas en compte l'obsolescence programmée psychologique. c'est-à-dire  : créer des phénomènes de mode pour inciter le consommateur à remplacer le produit ne rentrerait pas dans le cas de l'obsolescence programmée ? 
}\smallbreak

Ça c'est quelque chose effectivement qui participe à l'obsolescence des produits. C'est-à-dire que, quand des nouvelles choses apparaissent, qui ont des caractéristiques plus plaisantes, quelle qu'en soit la raison, pour les utilisateurs, effectivement ça génère de l'obsolescence mais on est face à quelque chose où l'on propose au client un nouveau produit. 

Ce que je veux dire c'est que si vous avez votre téléphone d'il y a trois ans, il marche toujours maintenant.  Même s'il y a une nouvelle gamme, il marche toujours maintenant. Si vous avez une voiture d'il y a dix ans, elle marche toujours maintenant. Ce qu'est l'obsolescence programmée, c'est : vous avez acheté une voiture il y a dix ans et au bout de dix ans elle ne marche plus. Pour moi le fait que depuis dix ans, les nouvelles voitures sont plus performantes, plus économes en carburant, \dots, ça, c'est quelque chose qui est tout à fait différent, même si c'est quelque chose qui effectivement est le facteur principal qui conduit en pratique les gens à renouveler leurs matériels.

\textbf{D'accord donc si nous avons bien compris l'obsolescence programmée fonctionnelle n'existe pas.}\smallbreak

Oui c'est ça : l'obsolescence  fonctionnelle, c'est-à-dire celle qui consiste à faire que quelque chose cesse de fonctionner. C'est celle-là qui est une stratégie qui n'est pas rentable, que les entreprises ne mettent pas en œuvre, tout simplement.

\smallbreak\textbf{D'accord. Est-ce que vous pourriez nous donner clairement votre avis sur cet aspect de la question : pourquoi l'obsolescence programmée  n'existe pas, et en quoi ce n'est pas rentable pour les entreprises, puisque vous nous en parliez ?
}\smallbreak

Et bien je vais vous donner un exemple très simple, qui permet d'illustrer la raison pour laquelle ce n'est pas rentable pour les entreprises. Je vais prendre un produit mais l'exemple peut se généraliser à tous les produits. Supposez que j'ai le choix, je peux, soit vous faire des chaussettes qui durent six mois, soit des chaussettes qui durent un ans. En tant que fabricant, qu'est-ce que j'ai intérêt à faire ?  

Imaginez que vous changez vos chaussettes tous les six mois. Vous achetez une paire de chaussettes à 10 euros et au bout de six mois elle est ruinée : il y a des trous. Vous la changez, vous en rachetez une autre. Je suis une entreprise, qu'est-ce que j'ai intérêt à faire, est-ce que j'ai l'intérêt sur l'année à vous vendre deux pairs de chaussettes ou vous vendre une paire de chaussette à 20 euros qui va vous durer l'année ?

\smallbreak\textbf{C'est sûr que le consommateur va préférer celle qui dure le plus longtemps.}\smallbreak

Mais maintenant raisonnez du point de vue de l'entreprise. Si je vous fabrique deux paires de chaussettes, c'est deux fois des coûts de production, deux fois des coûts de distribution. Donc, de deux choses l'une : soit faire quelque chose de peu durable coûte vraiment moins cher et satisfait un usage un peu différent. 

C'est l'exemple du mouchoir en papier par rapport au mouchoir en tissu. Je pense qu'on est d'accord pour dire que le mouchoir en papier par rapport au mouchoir en tissu n'est pas un exemple d'obsolescence programmée. C'est deux produits différents. Il y a des gens qui préfèrent avec des mouchoirs en tissu, d'autres des mouchoirs en papier et ce que vous recherchez dans le mouchoir en papier c'est précisément son caractère jetable. Je pense qu'on est d'accord pour dire qu'il y a des produits pour lesquels on cherche un caractère jetable car on veut précisément quelque chose qui ne dure pas longtemps et qui soit d'usage unique. On ne parle pas d'une situation d'obsolescence programmée. 

Retournons à cet exemple de la paire de chaussette. Le producteur, lui, ce qui l'intéresse, c'est de vous vendre un produit au prix plus élevé, pas de vous en vendre deux à un prix deux fois moins cher. Vous voyez, c'est ça le raisonnement. 

\smallbreak 
Imaginez que vous êtes un client, jeunes. Vous aimez bien les produits high-tech, et vous changez de téléphone tous les ans. Vous payez  donc 500 euros par an pour changer de téléphone. 

Je suis producteur je me dis : je pourrais fabriquer  pour deux fois plus cher un téléphone qui dure 5 ans. Je vous dis ensuite, je vous garanti que ce téléphone dure 5 ans, et je vous le vend 1500 euros, trois fois le prix que vous payez maintenant. C'est bon pour moi producteur : je fais une marge beaucoup plus élevée. Et je le répète : pourquoi je fais une marge beaucoup plus élevé ? Parce que je ne vais pas m'embêter à relancer des processus de production, je ne vais pas m'embêter à consommer trois fois des ressources, consommer plein de frais de  personnel, consommer des coûts de distribution, \textit{etc.}, \textit{etc.} Donc en tant que producteur mon intérêt c'est plutôt de vous \og forcer \fg{}  à payer cher quelque chose avec lequel vous rester pour longtemps. Et je le répète c'est mon intérêt de producteur. Vous ne trouvez pas d'alternative à ce problème là. 

Je répète, je ne raisonne que par rapport à l'intérêt du producteur, je ne pense pas que le producteur cherche le bien du consommateur particulièrement. Je pense que le producteur cherche à maximiser son profit. Ce que je veux dire c'est que la stratégie qui maximise le profit, ce n'est pas de vous vendre deux paires de chaussettes qui vous durent six mois, c'est de vous vendre au double prix une paire de chaussette qui dure un an, parce que vous vous l'achèterez et moi je ferai un plus grosse marge. 

Voila,  c'est l'argument central, et c'est à partir de cet argument central que lorsqu'on prend des cas particuliers. On trouve un certain nombre de pratique d'entreprise, où on se demande ce qu'il se passe. On va a partir de la pour trouver  des explications qui sont de nature toutes autres.  

Parfois ces anomalies ont des explications techniques. Vous constatez qu'un produit tombe en panne à un certain moment. Il peut s'agir tout simplement de  choix techniques à faire : on décide que certaines choses doivent durer plus longtemps. La solidité et la durabilité c'est une qualité mais il y a d'autres qualités pour un produit, et il faut arbitrer entre les différentes options que l'on veut incorporer au produit. Vous vous êtes en école d'ingénieurs et vous voyez bien qu'il faut optimiser. 

Il y a un problème classique des marines de guerres, si vous voulez faire des bateaux rapides vous ne pouvez pas mettre beaucoup de blindage, ni beaucoup d'armement dedans. Mais par contre si vous mettez beaucoup de blindage et beaucoup d'armement et bien vos bateaux ne sont pas manœuvrant. Il faut toujours optimiser, donc on fait des choix. 

Ça c'est la première raison. La deuxième raison qu'on va rencontrer parfois, et là je pense qu'on aura l'occasion d'en discuter sur certains cas, c'est qu'il y a  des entreprises qui ont des pratiques anticoncurrentielles. Et ces pratiques anti-concurentielles, vont consister à jouer sur les produits pour limiter la concurrence. C'est mal. Dans le sens où c'est détestable du point de vue du consommateur, c'est déplaisant à tout un tas de niveaux, mais ce n'est pas de l'obsolescence programmée ; c'est autre chose.

\smallbreak\textbf{Est-ce que vous auriez un exemple de stratégie anticoncurrentielle ?
}\smallbreak

 Je vais vous en donner un exemple que personne ne cite, alors que c'est le cas qui devrait être le plus flagrant. Moi je suis professeur d'Université comme vous le savez. Je pourrais très bien écrire un manuel, et sortir une édition par an de mon manuel, en changeant un peu à chaque fois le texte de mes exercices, puis dire aux étudiants à chaque nouvelle année  : voila vous êtes obligés d'acheter le nouveau manuel. 


C'est une pratique anticoncurrentielle. Pourquoi ? Qu'est-ce que je veux faire ?  Je veux me prémunir contre le marché de l'occasion. Pour éviter que les étudiants des années précédentes revendent mon manuel aux étudiants des années futurs, ce qui, à moi, ne me rapporterais pas d'argent. Vous voyez, si je fais quelque chose comme ça, ce n'est pas de l'obsolescence programmée, même si dans la pratique j'ai fait en sorte que mon ancien manuel ne soit plus utilisable. Je n'ai pas mis une sorte de bombe à retardement dans mon manuel pour le rendre inutilisable au bout d'un certain temps. Je répète, l'exemple que je vous donne là, c'est probablement l'exemple le plus proche de ce qu'on pourrait appeler l'obsolescence programmée. On est bien d'accord c'est un comportement détestable, et rassurez vous je ne le fais pas. Ici on est dans un cas de nature différente. On va avoir l'occasion de voir d'autres exemples, car je crois vous vouliez parler d'un certain nombre de cas.

\smallbreak\textbf{
Au sein de notre monographie, ce que l'on définie \textit{\OP}, on le divise en deux parties : toute la partie fonctionnelle que vous avancez comme inexistante, et toute une partie psychologique, avec tout ce qui est phénomène de mode, arguments écologiques et ce genre de choses. Quand le consommateur remplace son produit, l'ancien produit est encore fonctionnel, et donc ça on le rentrerai dans l'obsolescence programmée. Pour vous est-ce que l'obsolescence programmée psychologique est elle utile pour notre modèle économique ? Est-ce que c'est seulement appliqué pour quelques cas isolés ou alors est-ce que c'est un mythe de consommateur ?
}\smallbreak

Alors, non ce n'est pas du tout quelque chose qui est isolé. La question, ici, c'est que le goût des gens changent, il y a plusieurs éléments à ce niveau là. Je vais vous donner différents exemples pour qu'on voit qu'il y a différents facteurs qui rentrent en jeu. 

Un premier domaine dans lequel c'est le cas, c'est l'habillement. Tout simplement la mode en matière de vêtements c'est un mécanisme par lequel on renouvelle sa garde robe plus que pour de stricts aspects fonctionnels.  Vous allez vouloir acheter la nouvelle paire de basket, ce qui relève d'un phénomène de mode. La mode est quelque chose qui relève d'un fonctionnement social. Pour les entreprises du textile, on peut parler d'un exemple, vous avez peut-être entendu parler du film l'homme au complet blanc. 


C'est un cas célèbre d'un scientifique  qui fabrique un fil inusable pour créer son costume. Conséquence malheureuse, les ouvriers du textile décident de le tuer, parce qu'ils partent du principe qu'il les mets à l’arrêt. 
\smallbreak
Mais notez bien, qu'il y a cet aspect fonctionnel, mais aussi qu'il y a un grand nombre d'entreprises dans le secteur de l'habillement qui gagnent très  bien leur vie avec des produits éternels. 

C'est le cas d'un certain nombre de marques classiques. On peut penser à la marque de chaussure Church's qui font des produits, avec comme objectif la garantie à vie. Quel est la caractéristique de ces produits garanti à vie sur le plan fonctionnel, pour reprendre votre terminologie ?  Et bien c'est qu'ils sont hors mode. Si vous achetez une paire de chaussure Church's, c'est la même qu'il y a 20 ans. Mon père en achetait dans les années 60, les modèles sont exactement les mêmes. Et effectivement, celles qu'il a acheté dans les années 60 n'ont pas bougées. 

Mais ça veut dire qu'il faut choisir de ne pas s'habiller à la mode. Le phénomène de la mode est quelque chose qui n'est pas une nouveauté de notre société de consommatio. C'est quelque chose qui a toujours existé, y compris dans les société les plus primitives. 

Vous voyez que les individus mettent en œuvre des mécanismes pour se différencier. Pour essayer de montrer qu'ils sont plus beaux, plus riches, plus intelligents que les autres, et pour séduire. Cela souligne que, dans notre société de consommation, ce soit la consommation par le fait que je vais acheter les vêtements qui me mettent à la mode. C'est le système économique qui satisfait une demande. 

Il y aura toujours une demande pour une sorte de fantaisie. Je ne sais pas si certains d'entre vous ont été dans des établissement scolaires où l'on porte l'uniforme. On voit bien  qu'en pratique tout le monde essaie de se différencier un petit peu, de mettre pour les filles une bague, ou de mettre des chaussures de cette manière. Tout le monde cherche à se différencier. 

De ce point de vue là, si vous voulez, vous avez le renouvellement des produits par le biais de la mode qui correspond réellement à une demande de la part des gens. Est-ce que c'est indispensable pour le système économique ? Je répète non. Si les gens d'un seul coup disaient \og   moi j'aimerai avoir des vêtements que j’achète une fois et qui durent éternellement\fg{}, ils en trouveraient. D'ailleurs ils en trouvent déjà. Si vous cherchez un bleu de travail, vous pouvez en acheter de très bonne qualité,  qui va vous durer une vie. Objectivement il ne changera pas d'apparence. Simplement je pense que vous n'allez pas aller à vos soirées étudiantes en bleu de travail.

\smallbreak\textbf{Non, sans doute pas.}\smallbreak

%########################################## ########################################### ########################################### ########################################### ########################################### ########################################### ########################################### ########################################### ########################################### ########################################### ########################################### ########################################### ########################################### ########################################### ########################################### ###########################################

Voilà donc  cela,  c'est un premier élément sur des produits où je dirai qu'il n'y a pas véritablement de changements. Le fait que cette année la mode soit aux chemises à carreaux et l'année dernière aux chemises à rayures, ce n'est pas un changement de nature technique. Ça, c'est un premier point.  En deuxième point vous avez des vrais changements de nature technique. Cela correspond à une période ou l'évolution technique est extrêmement rapide. Je vais vous donner un exemple. Peut-être que vous vous êtes un peu jeune... Depuis combien de temps est-ce que vous utilisez un ordinateur ?

\smallbreak\textbf{J'ai connu Windows 95}
\smallbreak

Ah c'est parfait ! Est-ce que vous avez joué aux jeux vidéos à cette époque ? 


\smallbreak\textbf{Oui.}\smallbreak

Je pense que vous avez constaté quelque chose. Moi je peux vous le dire car j'étais joueur à cette époque là.  À chaque nouvelle sortie de jeu, il fallait acheter énormément de matériel pour son PC. Il fallait, pendant la longue période où j'étais joueur, une carte graphique tous les ans, des barrettes de RAM à racheter assez régulièrement, et une carte mère tous les deux ans et demi ou trois ans. C'était à peu près la cadence à suivre, étant donné le rythme d'évolution. Pourquoi ? Et bien lorsque vous aviez une nouveau jeu qui sortait, il avait des performances graphiques meilleures, mais qui étaient encore meilleures si vous aviez la dernière carte \textit{etc.}.  
Si vous vouliez  jouer plus rapidement, en particulier dans tout ce qui était LAN, il vous fallait des bêtes de course. Il y avait une sorte de course à la technologie. 

Je ne sais pas si vous avez remarqué mais depuis quelques temps, le rythme a vraiment ralenti. Si vous acheté un des jeux les plus récents d'aujourd'hui, vous pourrez sans problème et avec de très très belles caractéristiques y joueur, même avec un PC qui a 4 ans. 
Là par exemple l'ordinateur portable avec lequel on discute en ce moment, je l'ai acheté en 2008. Il y a une dizaine d'année, un ordinateur portable ça ne durait pas aussi longtemps. Qu'est-ce qu'il s'est passé ? Et bien les cartes graphiques ont  vraiment eu des performances en augmentation très rapide pendant un certain temps,  et il y avait des jeux qui nécessitaient cette performance.

 Mais maintenant, où est-ce qu'on en est ? Une fois que vous avez obtenu des graphismes qui sont réalistes, sur des écrans très grand, quelles performances voulez-vous rajouter ? Il n'y en a pas. Vous ne pouvez plus rajouter grand chose, donc automatiquement une fois un certain niveau de qualité technique indépassable atteint, il n'y a plus de progrès techniques et, à partir de ce moment, le risque de renouvellement se ralenti. 
 \smallbreak
 
 Vous avez un deuxième facteur de cet obsolescence. Vous voyez, là, on n'est pas tout à fait sur le coté les goûts du consommateur. On est sur le fait que sur certains produits, il y a des périodes de très fort progrès techniques, pendant lesquelles on constate que les gens doivent renouveler régulièrement leurs produits. 
 
 On constate aussi que c'est une période pendant laquelle, bien souvent, les produits ne sont pas très durables. Quel intérêt de faire une carte graphique qui dure très très longtemps si de toute façon les gens les changent tous les ans ? Donc on va avoir cet aspect. 
 
 Par contre à l'inverse, à partir du moment ou le rythme du changement technique va  ralentir, les consommateur vont rechercher la durée de vie. Si je vous proposais un smartphone garanti 10 ans, est-ce que vous l'achèteriez ? Je ne pense pas. Regardez un téléphone d'il y a 10 ans . Acheter un téléphone durant 10 ans servirait à rien, cas dans une décennie,  il sera dépassé technologiquement. 
 
 Par contre si je vous propose une voiture qui est garantie 10 ans, là cela peut vous intéresser. Par contre, si je vous propose une voiture garanti un million de kilomètres, c'est-à-dire probablement ce que vous ferez dans votre vie, je pense que vous l'achèterez pas. Pourquoi ne l'achèterez-vous pas ? Peut-être que, dans 20 ans, les voitures rouleront toutes seules, ou même les voitures actuelles seront interdite ; pour réduire le taux d'accidents, on aura décidé qu'il est mieux que les voitures roulent automatiquement plutôt qu'elles soient pilotées par des gens qui peuvent être ivres \textit{etc.} Donc il vous est inutile, maintenant, d'acheter une voiture qui dure \og trop longtemps \fg{}. 
 
 Ce que je veux donc dire, c'est que le rythme de l'évolution à la fois de la société, des techniques, et de l'environnement fait que les objets doivent s'adapter. Et que de ce point de vue là, il est naturel que les objets changent. Est-ce que c'est un mécanisme qui est produit directement par les producteurs pour pousser les gens à racheter ? Dans une certaine mesure oui. Mais  ce que font les producteurs est plus sous l'effet de la concurrence : ils essaient de vous dissuader d'acheter les produits des autres, en faisant tout de suite quelque chose d'un peu plus sophistiqué que les premiers. Dans le cas par exemple des cartes graphiques c'était tout-à-fait ça : vous aviez deux marques de cartes graphiques concurrentes , qui essayaient sans arrêt d'être celle qui ferait la carte avec les meilleures performances.  Est-ce que cela répond à votre question ou vous voulez d'autres éléments.

\smallbreak\textbf{Oui ça répond tout à fait à la question.
\medbreak


Je propose de passer à quelques exemples. Commençons par le vieil exemple  connu : le cas de  l'ampoule et du cartel de Phoebus. Je rappelle que le cartel de Phoebus c'est une organisation de fabricants d'ampoules qui se serait arrangé en 1924 si je me rappelle bien pour réduire volontairement la durée de vie d'une ampoule. Le fait est qu'ils ont été condamné par la loi..
}\smallbreak


Anti-trust dans les années 30, je ne saurai plus exactement quelle année mais début des années 30 oui. 


\smallbreak\textbf{A priori le but de ce cartel est de réduire la durée de vie des ampoules qui était de 2500 heures à 1000 heures pour gagner en rentabilité, est-ce que vous confirmez cette théorie ? est-ce que pour vous c'est de l'obsolescence ?}\smallbreak

Alors, c'est une histoire très intéressante que celle de ce cartel. Comment est apparut ce cartel ? D'abord il est apparut d'un besoin des sociétés de standardisation, c'est-à-dire que les ampoules électriques, qui ont été inventées par Edison, existaient en différents types, différentes caractéristiques, \textit{etc.}\dots


Or pour le consommateur on a un besoin de standardisation quand on est sur un produit de ce type. C'est pour cela par exemple que toute les essences sont utilisable dans toutes les voitures.
De fait Elf, Total, Shell, se sont tous mis d'accord sur ce que l'on met dans l'essence. 

Pour les ampoules c'est la même chose. Quand vous avez une lampe, il est dans l'intérêt du consommateur que les culots soient standardisés, de manière à pouvoir visser des ampoules de différentes marques sur votre lustre par exemple.

\smallbreak

Donc c'est parti d'un besoin de standardisation, qui a fait que les grosses entreprises de matériel électrique qui fabriquaient des ampoules,  dont Phillips et d'autres marques, qui existent encore d'ailleurs, ont commencé à établir un certain nombre de standards.
C'est quelque chose de tout à fait normal et dont on a besoin, il faut que les produits soient interopérables.

\smallbreak

Ici, le problème est que pour l'établissement des standards des ampoules, il y a une contrainte physique, c'est-à-dire que la durée de vie d'une ampoule est un élément d'une équation qui contient aussi la luminosité, la consommation d'énergie, le degré d'échauffement\dots
Vous avez toute une série de caractéristiques qui font que, si vous voulez gagner en durée de vie, vous allez perdre dans d'autres caractéristiques, en particulier, en luminosité ou en consommation. D'ailleurs vous connaissez l'histoire de cette ampoule dans une caserne de pompier [ndt. l'ampoule de Livermore] est allumé depuis plus d'un siècle, cette ampoule là, si on la regarde, n'éclaire pas du tout.

Pourquoi ? Car c'est une des toute premières ampoules qui a été faite, et les deux raison de sa longévité sont qu'elle éclaire très peu, le filament s'échauffe donc peu, ce qui joue dans sa dégradation. Deuxièmement, c'est une ampoule qui n'est jamais éteinte, donc n'a ne subit pas de dilatation-contraction du filament.

Donc je répète, il y a un besoin de standard, car si certains choisissent un certain niveau de luminosité, cela aura des conséquences sur la durée de vie.
Les ampoules vont toutes être différentes, et vous, en tant que consommateur, vous voulez des ampoules standardisées.

\medbreak

C'est  donc le point de départ du cartel, et c'est tout à fait normal. Mais ils sont allé beaucoup plus loin.

Mais ensuite ils se sont réparti le marché, c'est-à-dire qu'ils se sont dit : \og  
Tel entreprise au le monopole sur tel marché, tel sur tel autre\dots  \fg{} et ont établi des prix pour dire  \og Tel doit faire tel prix à tel endroit  \fg{} pour que le marché soit entièrement fermé, entièrement verrouillé par eux.

 Et ensuite, le problème devient économique. Tout le monde a envie de tricher : Si tout le monde s'est mis d'accord sur un prix élevé, si j'arrive à vendre en douce des produits moins chère, je vais gagner de la part de marché. C'est un problème classique, et la raison pour laquelle l'OPEP[l'organisation des pays producteurs de pétrole] ne fonctionne pas. Si vous êtes un petit pays producteur de pétrole, vous pouvez profiter des prix élevé pour vendre un peu plus, pour gagner plus d'argent.  C'est donc pour ça que tout les cartels ont tendance à contrôler énormément tout leurs membres.
 
\smallbreak 
 
Un élément intéressante c'est que, à peu près à la même époque que le cartel Phœbus, est apparut en Suisse un autre cartel, celui du fromage, organisé par l'état. Pour avoir une idée, ce cartel limitait à 6 le nombre de fromages produit, alors qu'il en existait auparavant plus d'une centaine. Pourquoi ? Car cela permettait de contrôler l'activité.

\smallbreak

Et chez Phoebus, ils ont fait la même chose. Pour éviter que certains ne viennent contourner le cartel, ils ont imposé des standards extrêmement strict sur tout.
Ils ont décidé que la durée de vie d'une ampoule serait 1000h, tel caractéristiques, point. Ce qui, je répète, est un calcul comme un autre ! 1000h, pourquoi pas ?

Je veux dire, ils auraient pu faire des ampoules de 2000h deux fois plus chère, ça n'aurait rien changé. Mais ils ont préférer dire à leur producteur :  \og il est interdit de construire des ampoules qui durent plus longtemps \fg{}. Je reviendrais là dessus plus tard.


Pourquoi l'interdire ? C'est logique. Si vous voulez contourner un cartel, où on a dis que tout le monde vend des ampoules à 1\$, et vous n'avez pas le droit de baisser vos prix, et bien vous dites :  \og Mes ampoules sont à 1\$, mais elles durent plus longtemps ! \fg{}. Vous contournez ainsi le cartel.

\smallbreak

Donc automatiquement le cartel interdira cette pratique, non pas par rapport au consommateur, mais pour maintenir son monopôle. C'est pour cette raison que je dis que l'on est dans un exemple de pratique anticoncurrentielle. Imaginez que le cartel, par un coup de baguette magique, puisse inventer des ampoules qui éclairent de la même façon mais qui durent 2000h. Il est alors dans une position idéale pour vendre ces nouvelles ampoules deux fois plus chère ! Il peut d'autant plus le faire qu'il est tout seul. Et je répète, repensez au raisonnement que j'ai donné tout à l'heure, ça aurait été très rentable pour eux. 

Pourquoi  ne l'ont-ils pas fait ? Parce que étant une entreprise membre du carte allez-vous faire de la recherche pour augmenter la durée de vie des ampoules ? Non ! Puisque que si vous inventez quelque chose, vous êtes obligé d'en faire profiter tout les membres du cartel. Vous ne pouvez pas l'utiliser pour vous.


Donc aucun fabriquant n'a été incité durant cette période là à faire de la recherche pour augmenter la durée de vie des ampoules et la qualité des produits.
\smallbreak 

Ce que l'on donc voit ici, c'est que, quand vous faite face à un monopole,  les produits sont mauvais, et ne s'améliorent pas beaucoup. Ce n'est pas quelque chose de très nouveau et vous voyez qu'ici, le consommateur est absent. Vous voyez ici, il aurait sans doute été intéressant pour les entreprises de choisir un autre standard, mais elles ne pouvaient pas modifier le standard en question. Il aurait fallu se lancer dans la recherche et il aurait fallu le faire collectivement, ce qui est difficile à coordonner.

\medbreak

Par rapport au jugement: comme vous l'avez dis, le cartel est tombé dans un procès antitrust, mais ce qui est intéressant c'est que quand vous lisez les accusations, il est écrit noir sur blanc que les magistrats ont condamné le cartel à une très lourde amende. Il n'y a donc aucune notion d'indulgence. Ils ont explicitement dit :  \textit{Pour ce qui concerne la durée de vie, elle ne constitue pas une pratique anticonsommateurs, mais un choix de standard technique}. Cette phrase est directement extraite du jugement, qui lui même condamne le cartel. Se partager le marché est un scandale, mais avoir décidé d'une durée de vie standard, ça n'en est pas un.

%########################################## ########################################### ########################################### ########################################### ########################################### ########################################### ########################################### ########################################### ########################################### ########################################### ########################################### ########################################### ########################################### ########################################### ########################################### ########################################### %olivier


\textbf{Ce qui m'intriguait c'est la  durée de vie annoncé des ampoules. A l'époque, on annonçait 2.500 h. Vous ne pensez pas qu'ils airaient pu faire\dots plus ?
}\smallbreak

N'oublier pas que la durée de vie d'une ampoule n'est pas si facile que ça à calculer. L'ampoule d'un frigo par exemple, qui est éteinte et allumé sans arrêt, suis des contraintes tout à fait différent de l'ampoule d'une pièce qui reste tout le temps allumé.
Des ampoules identiques dans des contextes différents produiront des résultats différents. Donc cette durée de vie affichée et indiqué, quel est sa validité. On est à une époque où les entreprises mentent comme des arracheurs de dent, on vous vend des cigarettes en vous disant "les docteurs recommandent Camel". J'ai une grande méfiance par rapport à ce qu'affiche les entreprises. Il est très probable que vous trouverez "Jusqu'à 2.500h d'éclairage", comme mon téléphone qui a 20h d'autonomie\dots quand je ne m'en sert pas !
Moi je serait méfiant, et je pense qu'il a eu un vrai choix technique par rapport aux contraintes d'utilisation, pour avoir quelque chose de vraiment standardisé.
Notez bien que maintenant, les ampoules ont des durées de vie différentes. Si vous allez au supermarché, vous verrez des durées de vie différentes\dots et des prix différentes ! Et cela en fonction des gammes et des technologies misent en œuvre.
De ce point de vue là, dans un contexte qui est très différent des années 30, les entreprises trouvent un intérêt à vous proposer des ampoules qui durent plus longtemps, quitte à vous les proposer 3x plus chère.

\textbf{Alors justement, nous avons un exemple, cité par Serge Latouche. je suppose que vous le connaissez ?}
\smallbreak


Oui

\textbf{Il cite le cas de "Navra", une entreprise de l'Allemagne de l'est qui proposait des ampoules avec une durée de vie de 100.000h. Après la chute du mur de Berlin, ils ne réussirent pas à s'implanter sur le marché, et disparurent en 1995. Est-ce que cela veux dire que le cartel se serait prolongé jusqu'en 1995 ?}
\smallbreak


Non, premier point : le cartel a disparu dans les années 30, après sa condamnation, et aussi après la seconde guerre mondiale, à cause de la pression germano-américaine. Il y a eu de nombreux nouveaux acteurs sur ce marché, notamment les japonais. Le marché a complètement changé.
Quand à cette histoire qui est cité dans le documentaire, moi, tout les mythes lié à la qualité des produits de l'Allemagne de l'est  je pense qu'il faut quand même se méfier de ce genre de choses.
Si on vous dis qu'on faisait des ampoules à 100.000h, moi je répond "comment ?". Même en regard de la durée de vie des ampoules aujourd’hui, ça parait peu plausible.
Et je répète, si c'est vraiment le cas, quel intérêt a une entreprise à refuser ça ?
Mettons que la durée de vie actuelle d'une ampoule soit 2000h. Si on se remet à la chute du mur de Berlin, on va dire que ça valait 0.50\euro, pour 2000h.
Si je suis capable de faire une ampoule à 100.00h, je vais dire au gens "Regardez, je sais faire une ampoule qui éclaire tout le temps ! Garantie à vie !". Vous affichez votre garantie, les consommateurs sont intéressé : l'ampoule que vous ne changez jamais ! Vous voyez la réclame !
Et avec ça vous raflez tout le marché.

Donc c'est ça tout mon problème. Moi je veux bien qu'on me raconte des histoire, mais\dots vous savez comment on double la valeur d'une voiture d'Allemagne de l'Est ?

\textbf{Non}
\smallbreak


On fait le plein

\textbf{(rire) }
\smallbreak


C'était les blagues qu'eux même répétaient sur leurs produits. Et ce n'est pas un sombre complot qui a fait que la production Est-Allemande a disparu. C'est qu'objectivement, la qualité était plus que médiocre pour un prix élevé. Donc non, ce n'est pas la cause. Et si vous êtes un constructeur Ouest-Allemand et que vous récupérez le brevet d'une ammpoule qui fait 100 000h, vous faites fortune !
Donc vous partez du principe que les entreprises on sciemment choisi de renoncer à une fortune, moi je ne vous crois pas. Je ne vois pas les entreprises voir une pile de billet sur une table et les jeter par terre.

\textbf{Je propose de passer au prochaine exemple, non pensiont comparer la Ford T et la Chevrolet de General Motors. Là aussi, millieu du 20e sciècle, tout le monde a la Ford T, qui a deux problèmes : le style  est identique et pas très confortable, avec notamment un démarrage à la main. Quelques années après, GM essaye de s'implanter sur le marché. Pour ce faire, il n'ont pas choisi la fiabilité, mais plus l'esthétisme et le confort. Selon vous ce ne serait pas de l'obsolescence car ce n'est pas technologique ?}
\smallbreak


Ici, je pense que l'on est sur un choix qui concerne plus l'évolution de la société.
La Ford T était un evoiture pour les paysans et les population rurales. Elle a été conçu comme tel par Ford d'ailleurs, elle était vendu avec une boite à outils sous la banquette qui permettait de réparer sois même sa voiture et était conçu pour supporter les mauvaises routes et les intempéries.
A partir du moment où le contexte change, d'autres véhicules qui ont des caractéristiques plus adaptés à une population plus urbaine satisfait mieux une autre classe de la société (cadres, etc).
Et il ne faut pas oublié que cette époque a été synonyme pour l'automobile de fort changements techniques. General Motors s'est développé par la personalisation des véhicules, passant du tracteur amélioré de la Ford T à quelque chose qui était plus du domaine de l'habillement. c'est-à-dire assujetir la voiture à la mode et en faire un objet de distinction.
Cela correspond à l'évolution de la société et à un choix, ici de dire "On vendra plus de voiture en satisfaisant le besoin de différentiation des gens". Ce qui implique un degré de renouvellement.

\textbf{Et les bas en nylon ?}
\smallbreak


C'est la même chose. Les premier bas nylon étaient fait pour les gens qui habitaient à la campagne. Comme ma grand-mère, qui avait besoin de bas qui résistaient au ronces et aux traveaux dans les champs.
Quand vous passez des campagnes aux femmes des séries TV des années 50, le caractère esthétique prime sur la solidité. Autre chose, si vous voyez cette publicité d'un bas nylon qui tire un tracteur, il s'agit de publicité mensongère de l'époque. Les réglementations sur les réclames étaient beucoup plus légère que la réglementation actuelle. On avait le droit à l'époque de raconter absolument n'importe-quoi.

C'est un cycle pour éviter d'être limité par la donne technologique ?

Le smartphone, c'est un produit sur lequel il y a beaucoup de changements. Si les téléphones étaient resté ce qu'il étaient, les Nokia basique [NdA: 3310] que l'on utilisait juste pour téléphoner seraient légion. Sauf que les usages on changé. Même si vous aviez acheté un téléphone plus chère qui dure 20 ans au lieux de 3 ans, vous auriez fait un mauvais calcul, et vous l'auriez jeté bien avant, car les usages on changé. Mais on sent bien que ça se tasse, que l'on est sur la fin. Un Ipad 2, second seulement de la série, est capable de faire tout ce que fait un Ipad 4. On a fait le tour des usages.

Pour les smatphones, ça ne vas pas tarder à se stabiliser en faveur d'un modèle ou la durée de vie est provolégiée. Regardez le secteur des montres, qui est extrèmement prospère alors même que les produits vendu sont transmis de génération en génération.

Pour les smartphones, je pense qu'on est vraiment sur quelque chose de transitoire et que ça fera comme pour les PC comme pour les cartes graphiques. c'est-à-dire qu'il y aura un moment où on aura stabilisé la technique, on en est proche à mon avis et où à ce moment-là, les gens commenceront à acheter des trucs qui coûtent plus cher. Là aussi vous pourrez prendre l'exemple des Macs: vous voulez acheter les gros ordinateurs Apple, vous regardez, ils coûtent 3000 \euro. C'est énorme, mais d'un autre côté, c'est justifié. Regardez Apple vend très cher des ordinateurs et ils ont intérêt à le faire, et vous en tant qu'acheteur vous avez intérêt à le faire : on se dit je peux me payer ça puisque je vais le garder pendant longtemps. Ça n'aurait pas été concevable de vendre des ordinateurs à ce prix-là il y a 5 ans. Puisque il y a 5 ans les gens se disaient encore : "Ah non si je dois en changer dans deux ans, ça ne vaut pas le coup de payer autant." Maintenant je pourrais très bien me dire que je vais 3 000 \euro pour un ordinateur, enfin, je pourrais l'envisager, en sachant maintenant que je pourrais le garder très longtemps. Dès lors que vous savez que vos produits ne vont pas être rendus obsolètes par l'évolution technique, vous vous attachez à la durée de vie et vous commencez à être prêt à payer beaucoup plus cher pour ça. 
Je pourrais vous donnez d'autres exemples, comme la durée de vie des voitures est beaucoup plus longue que dans les années 80. Pourquoi, et bien parce que la technologie est à peu près stabilisée. Bien sûr on rajoute de temps en temps des tas de choses, mais, prenez les voitures des années 80, vous allez sûrement voir derrière un autocollant "Garanti 100 000 km" et au bout de 100 000 km, les voitures vont tomber en morceau, à cause de la corrosion. aujourd'hui avec les nouveaux aciers, avec les nouveaux revêtements, est-ce que vous voyez des voitures qui rouillent ? Jamais ! Et on va vous les vendre très cher pour le coup, et c'est une opération très rentable que de vous vendre cher une voiture qui est vendue avec l'aspect solidité.
Voilà donc je vous laisse continuer



%########################################## ########################################### ########################################### ########################################### ########################################### ########################################### ########################################### ########################################### ########################################### ########################################### ########################################### ########################################### ########################################### ########################################### ########################################### ########################################### %jean


\textbf{Alors on passe au cas d'Apple, qui est société très critiquée sur la question. Alors le 1er exemple chez Apple c'est le cas des batteries, qui étaient irremplaçables pour les baladeurs, au moins jusqu’à l'iPhone 3G où elle était collée à la coque.}
\smallbreak


Je crois que si vous cherchez sur Internet, vous pouvez trouvez des batteries que vous pourrez remplacer ensuite. Le fait qu'il n'y ait pas de batterie amovible ne signifie pas que l'on ne puisse pas changer la batterie. 

\textbf{ça complique sans doute énormément le remplacement de la batterie, au point qu'on propose de changer l'appareil.}
\smallbreak


Oui, je me répète mais je ne veux pas me lancer là-dessus, mais il y a quand même eu une période où Apple étaient les 1ers, en particulier sur leurs ordinateurs portables, à faire des produits dans lesquels effectivement il n'y avait pas la batterie amovible, comme il pouvait y avoir avant sur les ordinateurs portables, ce qui fait que parfois les gens en emportaient deux d'ailleurs. Et Apple ont effectivement été les premiers a faire des batteries qui sont changeables, peuvent être changées, peuvent être remplacées, mais il faut complètement ouvrir la machine pour le faire. Je me souviens que maintenant tous les constructeurs l'ont fait. Que si les consommateurs avaient envie d'une batterie amovible, ils pourraient en avoir. Je pense aux ordinateurs portables

\textbf{Concrètement je pourrais vous sortir la batterie de mon ordinateur portable. La batterie est changeable alors que mon ordinateur a seulement un an, donc j'ai l'impression que}
\smallbreak


Oui donc ça va avec, si vous voulez. Vous pouvez choisir. L'argument ici c'est : il n'y a a pas de problèmes par rapport à ça. D'accord ? Personne ne vous oblige et vous avez un choix large. Alors après maintenant sur les cas Apple, je pense qu'il y a quand même un point dont on a peu parlé qui est un point quand même important, c'est que chez Apple vous avez plusieurs facteurs. D'abord vous avez un facteur important qui est 
D'abord un premier point qu'on peut noter c'est que chez Apple en 1998 sur les MacPro, Apple ils étaient les pionniers de boîtiers avec vous savez une portière que vous ouvriez. Une des premières Keynote de Steve Jobs quand il est revenu chez Apple, où il montre la nouvelle gamme Apple, en disant regardez : contrairement à nos concurrent où tous les ordinateurs sont dans des blocs de métal très très serrées, nous une porte, on ouvre et vous accédez à tout. Donc ils étaient à l'époque plutôt dans les premiers à faire des choses où on pouvait remplacer facilement. Ils avaient plutôt tendance à le faire.
Après le problème ici c'est que faire des pièces amovibles, c'est quelque chose qui a des coûts pour un avantage qui n'est pas forcément très grand. Je répète mais vous voyez, il y a toujours un aspect coût-avantage à prendre en compte ici. 
Il y a un élément important chez Apple qui est le design. C'est vrai que faire une batterie qui soit amovible, du point de vue de la conception de l'ensemble, ça pose des problème de design. Si vous voulez faire quelque chose qui soit très fin, très allongé, \textit{etc.} Oui effectivement ça va vous posez quelques problème de design. Donc ils ont eu tendance à faire des produits avec des élément fermés.
Maintenant il y a un autre facteur qui est important, dont on a peu parlé, qui est le fait que les entreprises parfois se plantent. Elles font des trucs POURRI. Ça, les malfaçon, c'est fréquent. Et il y a beaucoup de cas qu'on appelle op qui sont en fait des malfaçons.

\textbf{Pour vous ça pourrait être le cas d'Apple ?}
\smallbreak


Je vais vous donner plusieurs exemples, pour comprendre cette question de la malfaçon. La Malfaçon, d'abord on sait que ça existe, regardez par exemple les constructeurs automobiles qui doivent faire revenir les voitures au garage pour les faire réparer parce que il y a un problème dans le système informatique, ou parce que il y a un problème dans le freinage ou que sais-je. Vous savez ça existe tout le temps. Je crois qu'en ce moment General Motors ils ont plus de voiture qui sont en retour que de voitures qu'ils ont vendu l'année dernière. 
On pourrait citer un autre exemple, pensez à Boeing, à son nouvel avion, les batteries se sont mises à brûler spontanément et ils ont dû toutes les ramener à l'atelier pour tout changer. Je pense qu'on est bien d'accord que des batteries qui s'enflamment ce n'est pas de l'obsolescence programmée, c'est clairement de la malfaçon. C'est compliqué', c'est naturel que de temps en temps il y ait des produits qui ne marchent pas. Sauf que les marques refusent et détestent le reconnaître. Pourquoi, parce que un ça peut leur faire des procès, et puis deux, pour l'image de marque, c'est extrêmement compliqué. Donc ils préfèrent de beaucoup quand ils le peuvent, interrompre le produit, enlever la gamme, enfin mettre le truc sous le tapis pourrait-on dire. 
Et il y a eu des cas comme cela, en particulier sur les Iphone, il y avait eu tout une gamme ou les batteries avaient une durée de vie affichée qui était largement supérieure à la durée de vie réelle, enfin des problème de fiabilité extrêmement importants. Chez Apple, ça fait mauvais genre. Des problèmes des fiabilité dans une marque qui vend cher et qui se placent quand même plutôt dans le côté haut de gamme, ça fait mauvais genre. Alors très clairement ici, ils ont pas du tout eu envie d'admettre qu'ils avaient fait une malfaçon.

\textbf{Ils ont préféré se faire accuser d'obsolescence programmée, que \dots}
\smallbreak


Je le répète, que les entreprises fassent de temps en temps des produits avec des défauts, et qu'ensuite elles soient très malhonnêtes pour le reconnaître, ça j'ai aucun souci pour dire que ça existe et que c'est très fréquent. Ça c'est pour le coup vraiment très fréquent. Je vous le dit, moi je ne suis pas l'avocat des entreprises sur toutes ces histoires-là.
Je pourrais vous citer un autre exemple qui était dans un documentaire qui était dans le Cash Investigation qui était passé sur l'obsolescence programmée, qui était le cas d'une télé Samsung, où en fait les condensateurs avaient été placés juste à côté d'une pièce qui chauffait énormément. Alors dans le reportage ils vous montraient des gens qui avaient leur télévision qui disaient qu'au bout d'un elle est tombé en panne, j'ai été obligé de la changer, \textit{etc.} Et après, les modèles suivant de la marque, les condensateurs étaient placés beaucoup plus loin de cette pièce chauffante. Ici on est très probablement plutôt face à une malfaçon qu'à un vice de conception. 
Vous voyez sur des produits où les gammes changent très vite, de temps en temps, on fait un truc pourri. Ça cause des défauts et évidemment les entreprises s'en rendent compte quand les client commencent à revenir et à râler. Sauf que à ce moment là on ne va pas sortir un grand communiqué en disant : tous nos produits de tant à tant ont tendance à tomber en panne rapidement, venez, on va vous le changer. Vous allez pas faire ça parce que vous allez, un, passer pour des peintres, deux, avoir des procès en particulier aux États-Unis, donc qu'est-ce que vous faites, vous mettez ça sous le manteau, et le jour ou quelqu'un vient avec le truc, vous êtes malhonnêtes et vous dîtes : non, non, il n'y a jamais de problème, nos clients ne se plaignent pas, \textit{etc.} Donc ça ce sont des cas qui sont assez fréquents, et effectivement Apple sur cette histoire-là, ils ont été fourbes comme tout le monde. Mais je répète, c'est une erreur de leur part. Un autre point qui me semble très important sur cet aspect-là aussi, c'est le fait que aujourd'hui il y ait une grande différence pour les produits électroniques, c'est que les distributeurs vont proposent une extension de garantie. Et ils gagnent beaucoup d'argent sur l'extension de garantie. Et ça ça a deux conséquences. La première conséquence c'est que ils veulent pouvoir vendre l'extension de garantie, donc quand on dit, mon produit dure plus longtemps que la réglementation, ça les ennuie. On sait que ça les ennuie. Darty et autres, ont déréférencé Dyson, puisque les aspirateurs qui ont 5 ans de garantie, eh bien au Royaume-Unis c'est trois ans de garantie, et en France c'est un an de garantie. Donc Dyson a dit, écoutez, nos produits, ce sont les mêmes, donc on offre trois ans de garantie en France aussi, puisqu'on respecte la réglementation d'un côté, on ne va pas faire des produits moins solides pour le marché français. Les distributeurs français ont menacé des les déréférencer, et même l'ont fait pendant un temps à cause de ça. Parce que ça allait les empêcher de vendre de l'extension de garantie. Et maintenant, je ne sais pas si vous avez remarqué mais quoi que ce soit que vous achetez, c'est un vrai scandale d'ailleurs. Vous allez avec le vendeur, vous voulez acheter quoi que ce soit, une machine à laver, un smartphone, n'importe quoi, il vous dit : "Celui-là est super solide, super produit, vraiment robuste, vous payez plus cher, mais vous allez voir vous ne regretterez pas, etc" Une fois que vous avez signé le bon de commande, vous vous asseyez et on vous dit : "Vous savez ce produit est vraiment nul, je vous conseille vraiment de prendre une extension de garantie, ce sera plus prudent". Ce discours-là vous l'entendrez à chaque coup. Je vous conseille vraiment de faire un test. La prochaine fois que vous allez acheter quelque chose, je peux vous le dire, je l'ai fait il n'y a pas longtemps, j'ai acheté un lave-vaisselle plutôt haut de gamme parce que précisément moi j'aime bien payer plus cher pour avoir des trucs qui durent plus longtemps, ce qui est mon choix. Quand j'étais étudiant je faisais le choix inverse, mais je déménageais tout le temps, donc aucune raison d'avoir un lave-linge éternel. Vous voyez, ça aussi ça fait parti des choses qui expliquent le phénomène. On a parfois besoin d'un truc pas cher qui ne dure pas longtemps. Et quand on vieillit, on aime bien avoir des trucs qui durent longtemps, et bien on change sa consommation. 
Bref, j'ai donc pris un truc haut de gamme avec vraiment la marque qui dure très longtemps, avec la marque garantie constructeur très développée, enfin tout. Je m'assieds pour prendre mon bon de commande, et il me dit : "Vous devriez quand même prendre l'extension de garantie". Je lui ai dit explicitement que je voulais le produit qui dure le plus longtemps, et il me proposait quand même l'extension de garantie. Donc ça c'est quelque chose qui, je pense, entretient le mythe, entretien la croyance. Parce que à chaque fois que vous achetez le moindre produit, on vous dit, mais vous savez, de nos jours ça dure moins longtemps qu'avant. Ça entretient la croyance. Il y a d'autres choses qui entretiennent la croyance, je ne sais pas si on aura le temps d'en parler, mais indépendamment de la vraie durée de vie des produits, on nous ment beaucoup sur leur durée de vie, mais sur le sens qu'on pourrait croire.

\textbf{Donc pour vous, qu'est-ce qui entretient la croyance de l'obsolescence programmée ?}
\smallbreak


Alors je pense qu'il y a plusieurs facteurs. Il y a d'abord un premier facteur qui est un facteur psychologique bien connu qu'on appelle le biais de survie. Le biais de survie c'est que vous ne voyez par définition que ce qui survit. Vous voyez un frigo des années 50, vous voyez un frigo des années 50 qui a survécu. Résultat vous vous dites : "punaise tout les frigos des années 50 que j'ai vu ils avaient l'air solide !" Vous voyez, le problème. Et c'est un peu comme si vous me disiez, fumer ne donne pas le cancer parce que tout ceux que je connais sont vivants. Vous voyez, c'est ça le problème du biais de survie. Donc je pense qu'on a toujours un biais, on a toujours l'impression que le vieux était plus solide, parce que par définition, le seul vieux qu'on voit c'est celui qui a résisté. On ne voit pas le vieux détérioré d'avant. Et je peux vous dire que des objets nuls d'avant, il y en avait vraiment beaucoup. On ne voit pas tout ce qui est passé par la poubelle auparavant
Donc ça c'est un premier point. Le biais de survie qui nous donne l'impression que le truc était plus solide avant. Avec ce biais vient l'idée qu'on a toujours tendance à idéaliser sa jeunesse son jeune temps, celui d'avant, \textit{etc.} Peut-être que quand vous allez revoir les choses, je ne sais pas, peut-être que vous allez retourner dans la cour de votre école quand vous étiez petits, vous allez la trouver beaucoup plus petite que quand vous y étiez. Simplement parce que vous étiez haut comme ça et que tout vous paraissait beaucoup plus grand. Donc on a tendance comme ça a idéaliser un peu le passé.

Je pense qu'il y a un autre facteur aussi qui est le fait qu'on est entouré de produits électroniques qu'on ne maîtrise absolument pas. On a aucune maîtrise sur ce qui se passe à l'intérieur. Personne ne sait pourquoi un truc tombe en panne. Reconnaissez qu'on passe notre temps à vivre avec du matériel et on ne sait pas pourquoi ça tombe en panne, et c'est énervant parce qu'on ne comprend pas. Et ça c'est quelque chose qui est un vrai désagrément, on est entouré de trucs dont on dépend, et qui peuvent d'un seul coup nous laisser en rade et on ne sait pas pourquoi. 
La Ford T, avant, vous aviez la boite à outils en dessous et les compétences mécaniques pour le faire. ajd même si on vous donnait la boite à outil et même avec les compétences, même vous qui êtes ingénieur vous ne pouvez pas réparer, parce que vous n'avez pas les compétence simultanées en  électronique, en langage informatique, en mécanique, etc, pour réparer une voiture moderne. Vous ne pouvez pas réparer la totalité d'une voiture moderne vous maintenant, c'est trop compliqué, nos produits sont trop compliqués par cela. Résultat quand il tombe en panne, on est vraiment complètement démunis. Face à ce côté démuni, face à un gros désagrement, je veux dire, vous arrivez devant votre ordinateur, il ne marche pas, votre smartphone, vous l'allumez il ne marche pas. C'est horrible d'être désarmé à ce point-là. Quand il y a un lavabo qui est bouché, vous donnez des coups dedans et il se débouche quoi. Là on ne le maîtrise pas. Et je pense que ça ça crée réellement un sentiment d'être démuni, et c'est quelque chose qui a été bien étudié en particulier par un sociologue qui s'appelle Gérald Bronner : quand on ne comprend pas ce qui nous entoure, on cherche des explications un peu du type : quelqu'un maitrise, quelqu'un est responsable. C'est un peu la même chose que des théories du complot. Quand je ne comprend pas ce qui se passe, je préfère me dire que quelqu'un de très méchant est responsable parce que d'un certain point de vue c'est rassurant. Et je pense qu'il y a un peu ça avec les produits électroniques. Avec un effet caisse de résonance qui est un peu la même chose que les forums Doctisimmo. Vous avez un smartphone de tel modèle qui tombe en panne. Avant, il tombait en panne un point c'est tout. Vous n'y pensiez plus, et ça ne laissait pas de traces. Maintenant vous allez sur le forum Objets-MACHIN-CHOUETTE qui tombe en panne, vous faites une recherche Google, vous tombez sur 10 personnes qui ont eu le même problème, même si ce problème est inexistant, c'est-à-dire qu'il s'agit de 10 personnes sur 50 millions, vous allez vous retrouver avec dix personnes, et pour vous 10 personnes c'est beaucoup. Je pense que notre époque pousse un petit peu à ce genre d'interprétation et de pensée. 

\textbf{Alors si j'ai bien compris, finalement ce sont les entreprises qui innovent le plus qui sont le plus sujettes à être accusées d'obsolescence programmée. Puisque quand on est amenés à faire quelque chose de nouveau, placer par exemple un condensateur au mauvaise endroit, on pourrait donc expliquer ainsi, le fait que Samsung ou Apple soient très critiquées puisque ce sont après tout des entreprises qui produisent énormément de brevets. }
\smallbreak


D'abord effectivement, quand vous êtes très innovants, forcément vous rendez les choses obsolètes, donc forcément vous êtes critiqués. ca je suis tout à fait d'accord avec vous, plus vous êtes innovants plus vous allez vous exposer à ce soupçon. 

\textbf{Autant pour les anciens produits que pour les nouveaux finalement, parce que ceux qu'on crée on ne sait pas encore à quoi s'attendre parce que ce ne sont pas des  techniques  encore très connues et il peux y avoir des malfaçons.}
\smallbreak


Je pense beaucoup plus à des erreurs de malfaçon quand on fait des choses nouvelles que quand on reste sur des sentier battus. Mais les malfaçons, même si elles peuvent donner le sentiment que c'est fait exprès pour que je rachète, c'est pas un calcul tout de même. Si vous vous avez un smartphone d'une certaine marque et il tombe en panne, ce n'est pas un argument pour que vous en rachetiez un de la même marque. Au contraire vous allez avoir tendance à dire, cette marque est nulle. Ce qui est d'ailleurs souvent excessif. Mais c'est parce que votre expérience c'est celle-là. Je répète, la marque aurait plutôt tendance à vous garantir sa solidité, et il y en a beaucoup qui le font. Et vous avez raison, dans les secteurs et pour les entreprises qui sont les plus innovantes, c'est là que le soupçon va être le plus fréquent. Et qui va être étayé effectivement par le fait qu'on a en plus comme vous l'avez dit des malfaçons plus fréquentes. 

\textbf{Je pense qu'on va s'arrêter là. Merci beaucoup de nous avoir répondu, ça nous a apporté beaucoup de choses je pense}
\smallbreak


Je vous en prie.

\end{small}