\label{InterviewBClémentin}

Donc nous aurions quelques questions : 
On voulait savoir votre position exacte, concernant l’obsolescence programmée, pour vous elle existe ?
Les différents scénarios possibles dans le futur, si on continue de cette façon ?
Quelles mesures pourraient être utilisées pour contrer l’obsolescence programmée ou ses effets ?
Et enfin on voulait savoir ce que vous pensiez du fait que certains économistes définissent l’obsolescence programmée comme un mythe, si vous avez des contre-arguments à nous proposer ?	

Vous avez vu dans le livre de Serge Latouche, il donne trois champs de l'obsolescence programmée.
 
La première serait purement technique si l'on peut dire, par la production, ce qui est le sens commun que l'on donne à obsolescence programmée. Et vous verrez que les exemples qu'il donne sont souvent d'ailleurs les exemples que vous retrouverez sur Internet. Ils sont d'une part très peu nombreux, ils concernent des champs d'application qui ne sont pas très importants.
 
Les deux principaux, c'est les lampes à incandescence et les bas nylons. C'est les deux grands exemples récurrents, donc effectivement, avec l'effet qu'on pourrait appeler l'effet Wikipédia, qui est l'effet actuel, ce sont les principaux exemples, que l'on retrouve tout le temps.
Les lampes à incandescences seraient dues à un accord de ce qu'on pourrait appeler un cartel des fabricants, dans les années 1920-1930.
Et les bas nylons c'est à peu près pareil, soit issus des technologies DuPont de Nemours sur le nylon, soit ensuite les fabricants proprement dits. Donc on aurait découvert des bas nylons indéchirables, et évidemment on ne les a pas fabriqués.

Le deuxième champ de l'obsolescence programmée, c'est ce qu'on peut appeler le taux de renouvellement dans une société, d'une technologie ou d'une technique. Le mot technologie est un peu ambigu, souvent il ne correspond pas.

\vspace{1\baselineskip}

M : D'un savoir-faire peut être ?

\vspace{1\baselineskip}

Plutôt d'un savoir-produire. Le fait qu'un produit, quel qu'il soit, change

\vspace{1\baselineskip}

M : Ça c'est un cycle qui est normal on va dire. Rien n'est éternel.

\vspace{1\baselineskip}

C'est différent de la mode, attention. C'est le troisième champ, qu'on pourrait qualifier de social ou psychologique, ou autrement, mais qui est autant du point de vue du fabricant que du point de vue du consommateur. C'est-à-dire que tous les trois, quatre, cinq ans, alors ça peut être selon les époques ou autre, on a quelque chose qui fonctionne, qui remplit ce pour quoi il a été fabriqué et acheté, et puis le consommateur n'en veut plus. Il n'y a pas de raison basique.
Par contre, il y a un champ qui est maintenant commun à tous ce qui se produit et tous ce qui s'achète, c'est ce que l'on appelle le marketing et notamment la publicité, la publicité commerciale et qui entraîne, presque par nature, un renouvellement constant de ce qui existe. Ce que l'on pourrait appeler la mode. Donc la mode avec ou sans guillemets, la mode au sens commun, c'est-à-dire une tendance, ...


\vspace{1\baselineskip}

M : Une tendance à changer quelque chose pas forcément pour un besoin technique on va dire


\vspace{1\baselineskip}

Non, on fait passer souvent un besoin technique, qui n'y est pas forcément, vous avez une très bonne explication, qui est forte : on continue à utiliser le vaisseau spatial Soyouz, qui a quarante ou cinquante ans, qui est certainement un peu mis à jour, mais pas beaucoup plus que ça, notamment pour s'arrimer. Ils utilisent encore un télescope.

\vspace{1\baselineskip}

M : Oui c'est la procédure de base sur l'ISS, mais c'est vrai que c'est lié au fait que l'ISS a été mise en place il y a un certain temps déjà.

\vspace{1\baselineskip}

Ce que les cosmonautes ou astronautes en disent, c'est que, il remplit sa mission. Il fait ce pour quoi il a été construit. Il le fait précisément, et il n'y a besoin de rien d'autre. Donc cinquante ans après, il est toujours là. 
On peut aussi utiliser quelque chose qui est plus terre à terre, qui est les rails de chemin de fer. Pour l'instant il y a des tentatives, des trains magnétiques, il y a des tas de systèmes qui sont proposés, mais pour l'instant, sauf, vraiment de petites exploitation, en général qui ne dépassent pas quelques dizaines de kilomètres, La plupart des trains, y compris les derniers modèles de la SNCF qui roulent tout de même à plus de cinq cent kilomètres par heure, se font sur du rail de chemin de fer quasiment tel qu'il a été. L'alliage est peut-être un peu différent, mais, tel qu'il a été conçu au départ, avec un morceau de cuivre, qui frotte sur du métal, et le tout va à  cinq cent kilomètres par heure. 
Alors pour revenir aux choses communes sur lesquelles on base l'obsolescence programmée. Alors je reprendrais plutôt sur la dernière question : à la fois oui c'est un mythe du point de vue technique. Un mythe peut être une fable ou une narration qui nous permet de vivre ensemble. Je ne sais pas à quel endroit de l'histoire on peut se placer, mais nous n'avons toujours pas trouvé un ingénieur repenti, alors vous en trouverez peut-être un, et il n'est pas impossible que cela existe dans différentes lignes de production spécifiques. Je ne dis pas que ça n'existe pas. Mais nous n'avons pas trouvé un ingénieur d'un process, qui vienne nous dire, à tel endroit nous avons fait des essais sur une éprouvette d'acier, et on a fait différents mélanges. Il y avait une éprouvette qui durait vingt ans, une qui durait dix ans, une qui durait cinq ans et on a choisi celle de cinq ans. Nous n'avons pas cette  preuve. Et quand on cherche un peu, non seulement on ne la trouve pas, mais elle est contraire à tous les systèmes de production parce que, on comprend bien l’intérêt de dire, on va faire quelque chose qui casse tous les tant, simplement si dans tous les processus de fabrication, les gens étaient certains sur un objet précis, qu'il soit fabriqué suffisamment, que à tel moment il va casser, là, encore une fois on n'a pas l'information.

\vspace{1\baselineskip}

J: Pour l'exemple de l'imprimante avec la puce, qui s'arrêtait à un certain nombre d'impression

\vspace{1\baselineskip}

Voilà, ça c'est un exemple qui est tout à fait possible, mais vous conviendrez qu'il est parfaitement anecdotique. C'est possible, mais là je dirais ça devient facile, parce que là, on a un compteur. On pourrait très bien le faire sur les roues de voiture, on pourrait le faire en fait partout où on a un capteur, soit de l'ordre de la dynamométrie, soit de l'ordre du nombre de tour, soit du temps. Effectivement, partout où on peut mettre un capteur qui ait une puissance de pilotage, ou une puissance de coupure, qu'on intègre dans un programme, il n'y a aucun soucis. Donc on comprend bien que cette chose-là, on ne risque pas de la voir ni sur une voiture ni sur un avion, ni sur un train, ni sur un système à oxygène, enfin tous les systèmes de santé que l'on pourrait retrouver dans les hôpitaux. Enfin tout ce qui est important, on comprend bien que ça n'existe pas. Et on peut se demander, si dans l'imprimante, je grossis un peu le trait, parlons au nom de la décroissance, ce qui peut paraître curieux, mais si dans l'imprimante, ce n'est pas pour être certain, que l'imprimante rende toujours le meilleur des services. C'est-à-dire qu'on se dit, on a fait cette imprimante qui franchement n'est pas chère, on se demande même comment ça fonctionne. Finalement elle arrive à faire quatre-vingt mille ou cent vingt mille copies, pour ne pas ruiner la réputation de la boite, à cent vingt mille copies on va lui dire d'arrêter. Alors je ne sais pas pourquoi. Il est possible que ça existe, il n'en reste pas moins que pour l'instant, aucun ingénieur des systèmes informatiques ou des techniciens n'ont sortis la puce pour la montrer et la tester. Donc on en reste à ces histoires qui sont peut-être vraies, qui ne sont pas très importante. 

Par contre ce que dit Latouche et c'est ça qui est bien dans son petit livre, alors évidemment il ne le met pas trop en avant parce que ce n’est pas très vendeur dans le sens des altermondialistes. Alors, je résume bien sûr, je grossis un peu le trait, il est bien pour les altermondialistes qu'il existe une obsolescence programmée, c'est-à-dire que dans une théorie du complot, même si elle est un peu molle, et systématiquement dans chaque circuit, chaque process de production, qu'il y ait quelque part un bureau noir qui introduise la petite chose qui fait qu'on va gruger le client, maximiser les profits, ... 
Sauf que, il n'y a aucune preuve de cela, aucun repenti, par contre, la pression de la publicité, la pression du marketing, la pression de la mode est telle que de toute façon, on change presque tout, tous les deux ou trois ans , et dans le monde informatique, qui est un monde jeune, qui est un monde très, très jeune, qui ne coûte pas cher, même pour une entreprise. Bon on va imaginer une entreprise qui a trois cents postes, pour les changer, ça peut être un souci. Mais, si on regarde le coût d'une machine individuelle, si on regarde, les taux d'amortissement, en comptabilité d'entreprise, changer quelque chose tous les trois ans, c'est pas un coût en fait. C'est plutôt une bonne gestion du matériel.

Donc là je vous parle depuis un MacBook de 2008. Monsieur Apple n'est toujours pas venu chez moi avec un fusil pour m'obliger à en changer. Et j'ai téléchargé Yosemite, le dernier OS, qui fonctionne très bien, et surtout qui fonctionne sur cette machine de 2008. Donc il y a quand même déjà six ans. Ce qui en terme d'informatique ...

\vspace{1\baselineskip}

M : Oui c'est relativement vieux, on est bien d'accord.

\vspace{1\baselineskip}

Bon certes, Mac fait des machines assez robustes.

\vspace{1\baselineskip}

M : Assez chères.

\vspace{1\baselineskip}

Oui, je regarde le coût. Donc je l'aurais achetée, c'est le modèle de base, il est dans les mille euros, en euros constants. J'aurais acheté un ordinateur d'à peu près la même valeur dans le monde de Windows, je l'aurais payé au moins cinq ou six cents euros. Est-ce qu'il aurait duré six ans, peut être ! On voit que sur six ans, le coût de base n'est absolument pas comparé au coût de base du logiciel, même si on les trouve maintenant gratuits ou très peu onéreux, enfin les logiciels importants notamment pour Casseurs de pubs, la Décroissance. On utilise Quart ?? Express et Photoshop, donc on met à jour la licence. Chaque année, cela nous revient à, je dirais environ cinq ou six cents euros tous les ans d'achats ou de renouvellement de licence. On voit qu'au bout d'un moment, le soft coûte plus cher que le hard. Mais je pense que l'obsolescence programmée, en ce qui concerne l'informatique, il y a beaucoup plus une pression de mode, et d'utilisation de nouvelles choses, qu'une petite puce qui dirait quelque part à l'ordinateur de s'arrêter. Cela dit, il va peut-être s'arrêter pendant la conversation, on verra bien [Rires].

\vspace{1\baselineskip}

M : On va espérer que si vous avez branché la batterie, ce n'est pas le cas [Rires].

\vspace{1\baselineskip}

Alors, dans les mesures pour la contrer, et oui, puisque là dans l'obsolescence programmée, au sens vraiment de la pression marketing, de la publicité on va dire, c'est notre cœur de sujet. C'est-à-dire qu'on a bien repéré qu'il y avait une pression pour le renouvellement de ..., de tout ! De tout ce qui peut s'acheter aujourd'hui, et c'est notre principal objectif de Casseurs de pubs, comme son nom l'indique, de l'ordre intellectuel, pas de l'ordre physique. On ne prétend pas casser les choses, mais vraiment casser cette idée de la pression qui fait que on est dans l'obligation, non seulement d'acheter presque tout ce qui existe, mais de le renouveler, de le renouveler en fonction d'éléments qui nous font comprendre que on n’est pas à jour, on est vieux, on est obsolète, qu'il faut renouveler quel que soit l'état.
Et je pense que Serge Latouche en parle très bien même si il le met un peu à la fin de son ouvrage. Il le démontre très bien aussi, enfin, ce n'est pas trop complexe. Puisque, qu'on soit d'accord ou pas, on voit bien que ce qui s'appelle la mode, et la pression médiatique ou la pression de communication et de marketing, elle est très, très forte. On parle maintenant des enfants prescripteurs, les enfants sont devenus un champ qui dispose de moyens financiers par la pression qu'ils peuvent mettre sur leurs parents. Il y a quarante ou cinquante ans, ça existait un peu l'argent de poche chez les enfants, mais c'était quand même beaucoup plus faible, chez la majeure partie des personnes, et surtout intellectuellement, il n'était vraiment pas facile de faire pression sur ses parents. Plutôt socialement d'ailleurs.

\vspace{1\baselineskip}

M : Le rapport d'autorité n'était pas le même on va dire.

\vspace{1\baselineskip}

La pression dans les magasins, l'affichage publicitaire, la possession soi-même pour un enfant, même à partir du collège, sinon avant d'un soit d'un téléphone portable, soit d'un ordinateur, qui permet d'avoir la vitrine du monde chez soi, ça n'existait pas. Il y avait les magasins, on faisait du lèche-vitrine. Mais, si on allait avec ses parents, une fois par semaine dans les rues du centre de sa ville, où qu'on soit, on suivait ses parents. Et les parents ne suivaient pas les enfants, pour acheter telle ou telle chose. Il y avait réellement des périodes pour les cadeaux. Or actuellement, nous ne sommes plus dans une situation de stock, ce qui était le cas, il y a quarante ou cinquante ans, lorsque j'étais enfant, nous sommes dans un état de flux, dans une situation de flux. Et tout le monde est dans cette situation de flux. 
J'ai été très étonné, il y a sept-huit ans lorsque ma fille a acheté un vêtement, chez H\&M, et elle a dit, « Ah non, il n'est pas bien, je le rapporte ». J'ai été vraiment surpris qu'on puisse rapporter un vêtement. Donc j'ai un peu regardé, un peu suivi, que le modèle économique de H\&M n'était ni le stock, ni la fabrication, mais le flux. C'est-à-dire qu'il faut qu'il y ait un flux constant, donc je lui ai demandé combien de fois elle rapportait un vêtement, combien de fois elle savait que ses amis rapportaient un vêtement, et en fait, si on prend sur une série un peu longue de quatre ou cinq mois, la personne, qui pense rapporter un vêtement, il y a une chance sur deux qu'elle en rachète un autre. C'est-à-dire qu'elle a acheté un premier vêtement, qui coûte ce qu'il coûte, donc qui paie la chaîne de production, elle revient, elle rend ce vêtement, et donc soit elle en prend un autre, mais la plupart du temps, souvent, elle en rachète encore un autre, donc au final, il y a eu deux vêtements produits, un qui n'est pas porté, mais deux qui ont été achetés. 

Et c'est un flux, et le flux est très important, parce qu’au sens de la ressource financière, là il est bien réel. Et on constate, c'est assez général maintenant, j'ai vu dans presque toutes les productions, il reste encore quelques voitures chez les concessionnaires, mais relativement peu, dès que vous avez, non pas les moyens, mais le temps vous allez commander votre propre voiture, y compris avec sa couleur et autres, sauf à prendre vraiment ce qui se trouve là, du coup vous avez encore un petit avantage, mais presque tout est en flux tendu. Et ça, évidemment, ça pousse à l'obsolescence au sens où, vous ne pourrez, au moment où vous faites l'acte d'achat, acheter que ce qui vous est proposé. Si ça vous est proposé dans un magasin et que vous voyez l'objet, là on est dans une situation normale. Mais la plupart du temps, si l'achat se fait sur internet, ou autre, vous ne saurez même pas exactement l'objet que vous allez acheter. Vous allez acheter l'image de quelque chose, dont vous n'aurez la réalité que quand vous le recevrez. Alors pour l'instant c'est faible en tant que proportion d'achat, mais ça monte, c'est aux alentours de dix ou quinze points, ce n'est pas tout à fait négligeable. Et puis, il y a tous les effets de commande, qui sont basés là-dessus, pour, au fur-et-à-mesure, tout ce que vous achetez, est forcément nouveau, et est forcément produit spécialement. 

Alors on conçoit qu'il y ait des choses qui soient assez lourdes, si on prend, la voiture, qui est très emblématique. Bon, à la Décroissance, on n'a pas de voiture, ça reste l'objet qui est le plus emblématique, puisque c'est l'objet le plus lourd, qu'une personne individuelle peut acheter, et qui dure à peu près jusqu'à dix ans. L'individu en France n'achète pas des avions, ni des trains ou des bateaux. On va "moyenniser" tout, mais le couple moyen de salariés qui est à peu près certain de garder ses revenus, qui a réussi à avoir une maison, qui peut payer à peu près les études des enfants, qui décide d'acheter un bateau, ce n’est pas si fréquent. Les ventes de bateaux d'occasion ne sont si importantes que ça, et les bateaux sont stockés. Ils ne servent pas mais ils sont stockés. Donc ça, ça se renouvelle très peu. Les bicyclettes, il y a eu la mode des VTT. Personne n'avait de vélo, il y a eu beaucoup d'achat de VTT, pour les enfants, donc il y a eu un fort taux de renouvellement. Mais au bout d'un moment, il n'y a plus tellement d'améliorations qu'on puisse apporter à un vélo. Une fois que tous les vélos seront en aluminium ou en fibre de carbone, on retournera peut être au vélo en bambou, mais bon. 

Donc on a cet objet qui tourne dans les débats sociaux de l'obsolescence programmée. Le parti politique les Verts, EELV a même réussi à faire voter une loi. Là je vous engage à la lire. Et on se demande à qui ils s'adressent. Alors on peut comprendre que les Verts ont eu un intérêt à faire ça, qui est évidemment politique. La plupart des lois sont des objets sociaux avant d'être des réalités, puisqu'il en est voté deux fois plus qu'il en est appliqué et celles qui sont appliqués le sont par un système, en tout cas en France, qui s'appelle des décrets. Et souvent derrière les décrets, il y a des arrêtés, des arrêtés qui fixent, soit les exonérations, soit les applications précises. Et là l'obsolescence programmée, on ne comprend pas comment la loi va s'appliquer. Parce que, pour que cette loi s'applique, il faudrait donc trouver ce cabinet noir d'ingénieur, dans chaque système de production, chaque process industriel, quelqu'un qui sort précipitamment en cachant sa serviette ou son ordinateur, où il a fait les plans du machin qui doit casser au bout de cent dix-huit jours et trois heures, et ne pas durer dix ans.

\vspace{1\baselineskip}

M : C'est ça le problème, l'état ne peut pas vérifier auprès de chaque entreprises que ça a pas été fait sciemment. Et dans le cas de l'examen du produit après coup, l'entreprise peut toujours dire que c'était une erreur de conception et qu’ils n’en sont pas responsables. Et personne ne pourra aller les voir pour cela. On ne peut pas vraiment les contredire.

\vspace{1\baselineskip}

Il y a déjà la loi sur ce que l'on appelle les vices cachés qui est très puissante.

\vspace{1\baselineskip}

M : C'est ce qu'ils avaient l'air de dire, lorsqu'ils ont passée la loi Hamon, ils avaient l'air de dire que cela faisait doublon avec la loi des vices cachés, et que du coup, ce n'était pas intéressant.

\vspace{1\baselineskip}

Tout à fait ! Mais c'est un geste politique. On connaît l'accident, on s'y est intéressé puisqu'on fait des manifestations avec Casseurs de pubs, pour l'abolition du Grand Prix de Formule 1. Et du coup, on était en pleine préparation de manifestation, lors de la mort d'Ayrton Senna. Et il semblerait que les ingénieurs de chez Renault ont diminué la taille de la colonne de direction pour gagner du poids. C'est un pur geste, bien sûr pas d'obsolescence programmée, ils n'envisageaient pas l'accident. Mais on peut comprendre que des ingénieurs, à un moment donné, cherchent à réduire systématiquement la quantité de matière qui doit être mis dans quelque chose qui doit être fabriqué. Bon là, c'était un objet unique, deux-trois modèles. Mais on peut imaginer une production de X millions de choses, si on réduit, ne serait-ce que d'un gramme, ou de dix grammes sur chaque partie, il y a un gain extraordinaire à la fin. Après, dire que c'est de l'obsolescence programmée, la maximalisation de la ressource minière ou autre, c'est vraiment très compliqué.

On voit mal par exemple les fabricants d'autocuiseurs jouer avec ça. Parce que, au bout du troisième autocuiseur qui exploserait, la marque peut plier. Donc c'est là où ça pèche. C'est là où ça pèche puisque, on a peu d'exemple. On a les exemples des bas nylons et des lampes à incandescence mais c'est un peu faible.

\vspace{1\baselineskip}

J : Donc pour vous, la loi ne sert pas à grand-chose, il faudrait plutôt attaquer l'obsolescence  psychologique. 
Comment on pourrait faire ? Il faudrait sensibiliser les gens ?

\vspace{1\baselineskip}

Quand on a commencé en 1999 avec Casseurs de pubs, on n'avait pas encore internet, on n'avait pas encore basculé dans la sphère internet. Et la publicité était encore en débat. Mais maintenant elle n'est plus en débat, parce qu’elle se télescope avec une autre donnée qui serait la gratuité, la gratuité d'internet. Et donc maintenant la publicité a quasiment tout racheté. Donc on a un vrai sujet, c'est comment invalider la sphère ou le champ publicitaire. Et donc il n'est pas forcément que produit, il est demandé du côté de la consommation. Parce que, ça serait la contrepartie d'une certaine gratuité. Mais effectivement c'est la pression publicitaire, la publicité commerciale qui est la plus importante, le champ le plus important dans l'obsolescence programmée. C'est-à-dire que c'est le côté psychologique et social qui fait qu'au bout de trois ans, il n'est pas question qu'on garde la même chose. Et là c'est une éducation.

La question importante c'est, est-ce-que la majorité de nos concitoyens est d'accord pour garder quinze ans de suite  le même vêtement, parce que tout simplement il n’est pas troué, parce qu'on a pu le raccommoder. 

Alors évidemment, nous c'est ce qu'on dit, la rénovation thermique des bâtiments c'est simple, vous achetez un pull ou deux ou trois ou quatre. Ça va coûter, disons cent vingt euros, et vous fermez le chauffage. Alors voilà, la rénovation thermique, elle est faite. Mais est-ce que c'est acceptable, et on voit que ça télescope deux grands champs politiques qui sont sans doute dans plusieurs pays. C'est que le champ politique de Droite, il dit que chacun doit être à sa place, et que les riches, si ils gagnent de l'argent c'est parce qu'ils travaillent beaucoup, et les pauvres si ils ne gagnent pas d'argent c'est parce qu'ils  ne font pas grand-chose. Et les riches, quand ils vont gagner de l'argent, ils vont envoyer des miettes aux pauvres. Et donc tout le monde est content. Et la Gauche nous dit, non, il faut redistribuer. Et il n'est pas normal que les pauvres n'aient pas accès à ce avec quoi jouent les riches depuis maintenant cent cinquante ans. Donc la gauche est coincée. Et la droite n'est pas coincée. Enfin, encore une fois, mesurez bien les termes. Donc la Droite nous dit, eh bien voilà, ça continue comme ça et la Gauche dit eh bien non, il faudrait que les pauvres ou ceux qui ont un petit peu moins accèdent aussi à ce qu'ont les riches. Et par exemple, nous on a une image pour ça. Alors les gens de la Décroissance, si on devait se placer sur un échiquier, on dirait qu'on est ni de gauche ni de droite. Mais en fait, comme on est humanistes et qu'on est partageurs, on serait plutôt à gauche. Mais par exemple dans ma ville, à Saint-Étienne, ça ne me dérange pas s’il y a que quatre Ferrari qui veulent rouler dans la ville. Et s’il n'y a aucune autre voiture. Du moment que les quatre Ferrari roulent au pas du piéton dans la zone piétonne. Moi ça ne dérange pas, même si il y en a cent cinquante. Moi c'est les douze mille Renault Clio que je veux éliminer. Parce que c'est elle qui ont l'impact de l'empreinte écologique. Alors évidemment ce discours-là est inacceptable. Les gens de Gauche disent, non on veut que les Ferrari s'arrêtent d'abord. Moi ça ne dérange pas, je veux bien que les Ferrari s'arrêtent d'abord, ça fera toujours quatre voitures de moins. Mais ce n'est pas le sujet. Ce n'est pas le sujet au sens de la Décroissance et de l'empreinte écologique.

Mais voilà où on en est, et les partis écologiques, les Verts qui sont  le plus important parti écologiste, sont à gauche, d'où le télescopage. L’incompréhension fondamentale est : les Verts ne peuvent pas aller à Droite, parce que, fondamentalement, à part quelques défenseurs de l'environnement très, comment dire, très environnementalistes, qui préféreraient la terre avec des arbres et des petits oiseaux plutôt qu'avec des êtres humains. La majorité des partis écologistes sont plutôt pour une redistribution et une diminution de l'empreinte écologique. Mais pas la Gauche. Donc on est coincés par ça. Du coup, la production de la publicité et du champ publicitaire qui impacte le plus sur l'acte d'achat, il est admis par tous. On ne voit pas bien, on ne peut pas penser qu'une mesure autoritaire même de l'ordre de la loi, peut être de nature à contrer un champ social qui n'est pas d'accord. On le voit sur les lois un peu plus emblématiques...

\vspace{1\baselineskip}

M : Ou alors quand il faut augmenter les impôts ?

\vspace{1\baselineskip}

Oui, alors il y a quand même dans la loi quelque chose qui a été introduit, c'est tout ce qui se fabrique doit pouvoir se réparer. Ça c'est une notion qui a été introduite dans la loi. Mais par exemple, il y a une loi qui pourrait interdire les objets jetables. Parce que là, on est plus que dans l'obsolescence programmée, enfin, c'est marqué dessus ! On ne trompe pas le client ! Vous achetez ça, vous vous en servez une fois, puis après vous le jetez. Donc c'est plus que programmé, et ça ce n’est pas interdit ! Bon on convient évidemment que certains objets , …

\vspace{1\baselineskip}

J : Le problème c'est qu'ils sont moins chers donc plus demandé. Les gens se disent, celui-là j'en ai pas forcément besoin, mais c'est jetable donc je peux en acheter quelques un à peu cher au lieu d'en acheter un bien que je n'utiliserais pas forcément très souvent.

\vspace{1\baselineskip}

Oui, c'est... Qu'est-ce qui a poussé M. Bic qui a commencé à inventer le stylo ? On aurait pu lui interdire ! On aurait pu lui dire, non, vous n'avez pas le droit de dire que c'est jetable. Vous fabriquez quelque chose de pas cher, les gens s'en servent autant de fois qu'ils veulent, ça les regarde. S’ils arrivent à récupérer et à bidouiller les lames, qu'il y a un petit fabricant qui se met à produire des lames différentes et qu'on peut mettre, ça les regarde. Après M. Bic il s'arrange sur la licence, les droits d'auteurs, enfin bref la liberté industrielle et autre. Mais non ! Là la société dans un ensemble et le code du commerce, droit du commerce et autre, a accepté le fait d'avoir du « jetable ». Tant qu’on n’interdit pas le jetable, on ne voit pas une grande avancée. 

Et après, interdire le jetable, ça peut poser des problèmes, notamment pour tous ce qui est du domaine de la santé. Donc ce sont des sujets qui ne sont pas bornés. On se retrouve à un moment donné de la situation, tant qu'on se trouve dans une économie de flux, il n'y a pas de régulation possible de cela. Et on ne peut pas retourner à une économie de stock, puisque on ne sait plus le faire, on n'est pas équipés pour cela. Nous ne sommes plus équipés pour cela. 

Alors enlevez la couche publicitaire et la couche marketing, c'est compliqué parce que, c'est quand même là où on trouve, pas le plus de travailleurs mais presque. C'est quand même des gens qui il y a cent cinquante ans on appelait des parasites. C'est-à-dire tous ceux qui ne travaillent pas directement à la production, c'est toutes les études de marques, et d'économistes, soit Marxistes, ou de la lignée, qui ont remarqué que, au fur et à mesure que l'on diminuait la quantité de personnes qui directement sont liées à la production, on augmente la quantité de personnes qui sont des couches intermédiaires pour faciliter la vente du produit. Alors c'est les assurances, ça concerne la fonction publique, ça concerne en fait tous les gens qui ne produisent rien, comparés à ceux qui produisent effectivement, qui restent les agriculteurs et les ouvriers. Alors on peut y mettre les ingénieurs [Rires].

Alors là, de plus en plus, on a tout une population, qu'on qualifie d'employé qui est au service de la transmission du produit fabriqué au produit consommé. Et si on enlève toute cette chaîne de flux, la moitié de ces gens-là, on en a plus besoin. Donc ce sont des champs très compliqués au sens de la réflexion globale. 

\vspace{1\baselineskip}

J : En mesures concrètes, vous disiez interdire le jetable, ce n'est pas forcément faisable à cause de la santé. Et sinon qu'est-ce qu'on pourrait faire ? Certains proposaient de mettre une garantie obligatoire sur tous les produits pour forcer les industriels à faire quelque chose de durable.

\vspace{1\baselineskip}

Oui, mais encore une fois, le vice caché, ça dure longtemps. Je crois que la prescription, c'est de l'ordre de dix ans au sens du code civile. SI vous achetez quelque chose et qu'il a un vice caché. Ou c'est peut-être que trois ans, je ne suis pas certain. Mais oui on pourrait le mettre.

\vspace{1\baselineskip}

M : Il y avait eu des discussions récemment pour passer la garantie légale de un à deux ans. Il me semble que ça a été voté d'ailleurs.

\vspace{1\baselineskip}

Par exemple quand vous achetez un ordinateur, on essaie de vous vendre une garantie de trois ou cinq ans. En plus. Donc si vous ajoutez le prix de la garantie à l'ordinateur, qui est le même ! L'objet que vous achetez est le même. Mais, avec cet objet-là, on vous vend une garantie. Et vous calculez que le vrai prix, la vrai rentabilité de l'objet pour l'entreprise, c'est le nombre de personne qui payent l'assurance, ajoutée au prix de l'objet. Donc il y a des gens qui font un prix, en disant : non l'objet fonctionne. L'entreprise fait un pari aussi, parce que elle sait à peu près combien de temps son objet va pouvoir fonctionner, ou le taux de renouvellement moyen constaté, y compris avec la prospective : dans combien de temps on en changera. On se rend compte que le vrai prix de l'objet n'est pas celui affiché sur le catalogue, il est en proportion celui du catalogue, plus une part de l'assurance complémentaire qu'on paie. 

\vspace{1\baselineskip}

Même si il y a la guerre des prix entre les fabricants, au bout d'un moment il y a la réalité derrière la production. Donc si on oblige tous les fabricants que tout ce qu’ils fabriquent soit garanti, au lieu d'être un an, soit deux ans, soit trois ans, nécessairement d'ici un ou deux ou trois ans, les prix augmenteront en proportion. Ça ne changera pas grand-chose dans la manière dont les objets sont fabriqués, sauf à ce que certains objets soient justement fabriqués parce que leur taux de renouvellement au sens de la mode, est plus beaucoup plus important que leur durée. Ça peut être typiquement le cas de certains ordinateurs qui ne sont vraiment pas chers, de certains ordinateurs qui ne sont vraiment pas chers, dont on sait de toute façon qu'au bout d'un an, il y a de fortes chances qu'il soit changé. 

Récemment j'avais récupéré le carton d'une scie sauteuse qui vient de Taiwan, qui valait douze euros. Alors elle est dans un emballage de papier kraft, bon le carton pourrait être coloré, ça coûterait le même prix. C'est volontairement présenté comme quelque chose de bas de gamme. Au lieu d'avoir une éjection automatique de la lame, il y avait encore des vis, comme on faisait quand moi j'ai commencé à me servir d'une scie sauteuse. Évidemment, celui qui faisait la démonstration était quelqu'un comme dans les foires, quelqu'un de très doué, et il coupé avec sa scie aussi bien qu'avec une scie haut de gamme. Alors après combien de temps ça dure, est-ce que les  à l'intérieur du rotor, est-ce que les fils électriques ou autres ça durait longtemps ?

\vspace{1\baselineskip}

J : Vu le prix on se dit que, on s'en fiche si elle ne dure pas très longtemps, avec le prix qu'on l'a payée.

\vspace{1\baselineskip}

Ça c'est exactement ce que vous avez dit tout à l'heure sur le fait, ....
Quand j'ai commencé la mécanique, maintenant on trouve dans les magasins de mécanique presque tous les outils des professionnels. Quelque chose qui n'existait pas, il y a vingt ou trente ans . Et donc la tentation était grande de ne pas acheter la clé haut de gamme, mais d'acheter le machin qui venait de Tchécoslovaquie. Mais on avait besoin d'une clé spécifique, on acheter en gros pour un écrou à dévisser, et au bout de la quatrième fois elle cassait. Et puis qu'est-ce qu'on fait ? Ou je change mon montage, ou j'achète la vrai clé parce que je vais systématiquement en avoir besoin. Alors il y avait bien cet arbitrage entre le pas cher, on ne sait pas si il va durer et le haut de gamme qui est garanti à vie. Mais au sens de l'industrie la question ne se pose jamais. L'artisan il va acheter du haut de gamme de toute façon. Il ne va pas se poser la question. 

Alors effectivement par exemple maintenant dans tous les magasins de bricolage comme on dit, il y a sans doute un équilibre que cherchent à trouver les fabricants entre fabriquer le moins cher possible, mais pas pour que ça casse, pour fabriquer le moins cher possible ce qui nécessairement va dans le sens on fabrique plus rapidement et on met moins de métal, et on regarde combien de temps ça dure. Mais le problème c'est que les bancs d'essais ont beau être le plus sophistiqués possible, être certains que quelque chose va casser dans trois mois ou six ans, je ne sais pas quels sont vos spécificités, de ce que vous avez travaillé en mécanique ou autre, c'est presque impossible.

\vspace{1\baselineskip}

M : Oui en plus ça dépend beaucoup de l'utilisation qui en est faite.

\vspace{1\baselineskip}

Tout à fait, tous les systèmes dont on prévoit l'usure de telle ou telle chose, 
Encore une fois, les trains qui roulent à cinq cents kilomètres par heure, il y a cent cinquante ans, des ingénieurs ont fait des calculs en disant qu’une roue en fer sur un rail en fer, ça ne marcherait pas. Et l'ingénieur des chemins de fer répond, on sait faire rouler un train à cinq cents kilomètres par heure, parce que l'année dernière on l'a fait rouler à quatre cents quatre-vingt, et il y a deux on l'a fait rouler à quatre cents cinquante, et il y a dix ans on le faisait rouler à trois cents cinquante. Et donc, à chaque fois, le gain n'est pas spectaculaire. La capacité d'oser ça, c'est qu'en fait, on est certain que ça va fonctionner. Parce que la différence elle peut être spectaculaire au sens de l'énergie ou de la caténaire qui doit rattraper son onde sonore. Donc ça c'est des choses un peu particulières, mais qui sont du domaine de l'ingénieur de base, si je peux me permettre. C'est-à-dire qu'au bout d'un moment on a vu que ça cassait, .... Et d'ailleurs vous regarderez les caténaires des TGV, ce sont les caténaires de base tels qu'on les utilisait il y a cinquante ans. Il n'y a plus tout le système de suspension qu'on a mis pour les TER. On a dû calculer que peut-être que ça durerait moins longtemps, peut-être que faudra les changer plus souvent, mais que finalement, celui qui résistait le mieux à cette onde sonore, c'était le caténaire le plus simple, puisque finalement, il y avait moins de vibrations si il y avait qu'un seul fil. Mais bon, je ne sais pas si c'est ça la raison. L'usure et l'utilisation au bout de quelques années, rendent bien mieux compte de si quelque chose dure ou pas, que des études sophistiquées.
Moi je suis comme vous, je cherche un ingénieur repenti. 

\vspace{1\baselineskip}

J : Un économiste montrait des études qui expliquaient que la durée de vie des objets n'avait pas diminué au cours des dix ou vingt dernières années, comme la plupart des gens le pensent. On a l'impression que le vieux réfrigérateur d'il y a vingt ans tient encore maintenant. 
Vous pensez que la durée de vie a vraiment changée ?

\vspace{1\baselineskip}

C'est pareil, je ne le constate pas. Ce qu'on peut constater, c'est que des réparations majeures ou mineures sur tous les objets, surtout mineures, sont très complexes. Et donc, le prix des objets ayant vraiment, énormément diminué. Je revois mes parents se demander si ils allaient acheter un lave-vaisselle, donc il y a trente ou quarante ans. C'était vraiment un objet de luxe

\vspace{1\baselineskip}

M : En terme de salaire, pour une personne, c'était, je ne sais pas, un tiers de mois ? Qu'on ait une idée à peu près.

\vspace{1\baselineskip}

C'était vraiment un arbitrage. Je ne veux pas dire c'était le lave-vaisselle ou les vacances. Ma mère était cadre, mais en argent actuel, je dirais qu'elle gagnait entre deux milles et trois mille euros. 

A deux mille cinq cents euros maintenant, vous achetez un lave-vaisselle tous les mois ! Je ne sais pas, ça vaut combien, trois cents ou quatre cents euros ? Et encore, un frigidaire, tiens on en a acheté un récemment, pour des étudiants, parce qu’on a fait un repas du marché au campus. Donc on a montré qu'on pouvait acheter des choses sur le marché, et le livrer au campus. Donc c'était un repas circuit cours. Et on nous a fait remarquer que comme on achetait le jeudi après-midi  et que le repas était servi le vendredi midi, c'était obligatoire qu'on ait un frigidaire. Ah ! On ne savait pas, donc on a acheté un frigidaire. Un gros frigidaire, on a dû le payer deux cents cinquante euros. On est allé chez Auchan, on a mis les cagoules pour pas qu'on reconnaisse les gens de la Décroissance, et on a acheté un frigidaire. C'est plus un arbitrage. 

Qu'est-ce qu'on fait si le moteur tombe en panne ? Et moi j'en ai un frigidaire, mais j'ai déjà presque tout changé sur ce frigidaire. J'ai racheté des pièces, et je les ai échangées, y compris la sonde. Donc, j'ai fait un trou, je l'ai retirée, j'ai colmaté pour que ça soit étanche, et on m'a dit, non ça ne se fait plus ça. Mais en fait les pièces détachées on peut les trouver par différents circuits. Ça m'a peut-être coûté plus cher, plus cher que de racheter un frigidaire. C'est doute là où l'arbitrage se fait, c'est-à-dire que ça peut coûter moins cher d'acheter du neuf, et puis en plus c'est du neuf, donc plus joli, c'est à la mode, et c'est plus un arbitrage. Alors encore une fois, on part bien du niveau de salaire moyen en France, tous les gens qui gagnent au-dessus de deux milles euros ou si c'est un couple, qui a entre trois et quatre mille euros de revenu par mois, c'est que quarante pourcent de la population. Il reste donc une grande partie de la population qui gagne moins. Mais même un Smicard peut se payer un frigidaire par an. Ce n'est pas le frigidaire ou les vacances. C'est une dépense qui doit s'analyser, qui doit se calculer, mais même avec la bourse, la dotation de rentrée scolaire, on peut s'acheter un frigidaire. On ne sait pas ce qui déclencherait le fait qu'on accepte finalement d'acheter des choses qui durent et de les réparer, parce qu'on pourra tout réparer, mais on ne pourra pas changer l'aspect de la chose. Ce que nous vendent la publicité et le marketing c'est avant tout l'aspect. Le robot de marque connue qui fait la purée n'a pas changé. Il y a un guide avec des ouvertures, et à l'intérieur il y a une lame qui tourne autour d'un axe. Et vous branchez un moteur électrique dessus.

\vspace{1\baselineskip}

J : Justement, on ne pourrait pas sensibiliser les gens, par exemple dans les écoles, que cela ne sert à rien de racheter quelque chose qui a la même fonctionnalité mais différente pour l'esthétisme ?

\vspace{1\baselineskip}

Si, mais on appelle ça de l'activisme politique. C'est-à-dire qu’on va vous écouter, on va être d'accord avec vous, en gros, tout le monde va être d'accord avec vous. 

\vspace{1\baselineskip}

M : Oui c'est du bon sens.

\vspace{1\baselineskip}

Sauf que la publicité EDF, c'est tout le monde court quasiment tout nu dans la maison. Et on va faire un cadeau à maman, on va lui acheter un presse-purée. Il faut vraiment être un .... Pour aller acheter un presse-purée à Emmaüs. On peut essayer d’entraîner ça, mais il n'y a pas d'ambiance à ça. C'est le bon sens, mais il n'y a pas d'ambiance

\vspace{1\baselineskip}

J : Donc vous qu'est-ce que vous faites comme actions pour contrer un peu l'effet de la publicité ?

\vspace{1\baselineskip}

On essaye de démontrer tout ce que je vous dis. Par exemple, sur la rénovation thermique, suffit d'acheter un chauffage, le presse-purée, il n'y a pas besoin d'un électrique, la voiture on n'en a pas besoin quand on est ville. On démontre même qu’on n’en a pas besoin quand on habite à la campagne.  On en a besoin à la campagne si on veut y vivre comme à la ville. Donc on va à pieds ou à vélo. Après, il y a des cas particuliers, donc on essaie systématiquement de rendre compte, d'une part qu'on peut le faire, qu'on est pas en danger, qu'on peut se débarrasser d'un très grand nombre de choses, qu'on est pas obligé de renouveler. Mais, encore une fois, ça reste un acte politique, même au sens le plus élargi ou le plus noble. Ça ne dépasse pas la proposition de dire aux gens : arrêtez de consommer ! Ça ne sert à rien mais ça contrecarre les politiques actuelles de croissance, puisque l'économie des pays est assise sur ce flux de consommation. On n’a pas d'autres arguments que de dire : ça ne sert à rien. On peut ramener à l'empreinte écologique ou autre, mais là on entre dans des domaines, même l'empreinte écologique elle est très difficile à expliquer à toute personne qui n'accepte pas la discussion.

\vspace{1\baselineskip}

M : Ça ne sert à rien de prêcher des convaincus

\vspace{1\baselineskip}

Depuis qu'on fait le journal Casseurs de pubs, vous avez le dossier Casseur de pubs presque uniquement contre la publicité, on est passé à la Décroissance parce qu'on pense que c'est plus le cœur du sujet, et nous avons actuellement environ dix mille ventes et trois mille abonnés. La plupart sont convaincus, on peut penser que, disons les vingt mille adhérents du parti écologique les Verts sont vraisemblablement convaincus. Après on peut même penser que les adhérents au sens très large, soit de Greenpeace, soit des grandes associations, de défense de l'environnement, si on monte à un million de personnes en France actuellement, de gens plus une ancienne partie de la population qui sont nos parents et grands-parents qui ont connu ce que c'était que l'économie ménagère : on ne jette pas quelque chose qui peut encore servir, bon on a peut-être trois ou quatre millions de personnes en France, c'est-à-dire moins de dix pourcents de la population de gens qui réfléchissent à la question, qui en même temps sont maîtres de la manière dont ils peuvent se conduire, parce que aller vivre à moitié nu dans les bois, c'est quand même un peu se dé-sociabiliser. 

Donc ces arbitrages qui font qu'on accepte de vivre dans cette société tout en essayant d'être économe, on s'adresse à une toute petite partie de cette population. La plupart des animateurs et des personnes qui voudraient qu'on améliore cette situation sont très peu nombreux. Il n'y a pas beaucoup d'élèves-ingénieurs qui nous contacte, ou qui contactent les autres associations pour parler de ça. Ça reste vraiment très, très faible. Parce que ce n'est pas lié justement aux problématiques majeures, soit aux problématiques climatiques, soit aux problématiques de dépression des ressources. Puisque ce n'est pas visible. En tout cas pas dans notre pays. Même la fin du pétrole, chez nous, on n'en parle pas.

La montée des prix a été un signe, et là tout d'un coup, ils baissent les prix. Alors bon, allez expliquer les problématiques, que l'Arabie Saoudite essaie de faire plier la Russie et le gaz de schiste aux États-Unis, les gens, .... Bon voilà le pétrole on ne sait plus quoi en faire, donc il n'est pas cher. C'est vraiment complexe. Parce que il faut une strate, de comprendre qu'on peut vivre sans  avoir la Rolex et la BMW, qui sont des marqueurs sociaux, même si on a les moyens. De toute façon tout le monde ne peut pas les avoir, donc ce n'est pas très sympathique comme proposition de société, et que si tout le monde en avait, de toute façon, ça s'arrêterait. On peut montrer ça, on peut le modéliser, on peut l'expliquer, on peut en parler. Mais on ne dépasse pas, on pourrait dire les structures mentales des personnes qui se sont déjà emparée du sujet, et qui font déjà la réflexion.

\vspace{1\baselineskip}

J : Quels seraient selon vous les scénarios possibles dans le futur ? Les gens commenceront à se rendre compte que certaines ressources manqueront, comme le cuivre l'or, certains disent dans dix-vingt ans ans il n'y en aura plus.

\vspace{1\baselineskip}

Oui, et ça j'ai déjà rencontré des gens dans des entreprises, qui sont des directeurs des achats en gros. C'est des discussions qui sont récurrentes au niveau des productions. Simplement, pour l'instant, ça n'est pas arrivé. C'est-à-dire que pour l'instant, aucune des ressources dont on parle n'a fait défaut. Et on n'est jamais sûrs exactement de ce qu'il y a dans le sol. Don on en est réduit à proposer aux gens de se préparer pour quelque chose qui est certain d'arriver, mais dont on ne peut pas fixer la date. Et surtout qu'on ne peut pas prouver. On peut dire qu'il n'y a pas assez de cuivre pour tout le monde, pour que tout le monde ait l'électricité. Aujourd'hui dans le monde, il n'y a pas possibilité de mettre des fils électriques pour les sept milliards d’habitants de la planète. Pour l'instant, on se contente que deux ou trois milliards n'aient pas accès à l'électricité. Ou pas régulièrement, ou en tout cas ils ont une électricité basée sur des générateurs diesel. Donc on ne se pose pas de question, pour savoir si on leur met des fils électriques ou pas. Et on a toujours, je ne suis pas Marxiste, mais les analystes marxistes, les pays ont toujours une armée de réserve. Cent millions de Chinois en armée de réserve. On a à peu près quatre à cinq cents millions d'Indien en armée de réserve et on n’a pas loin de huit cents millions d'Africains. Donc on a une armée de réserve, c'est-à-dire une quantité de personnes qui n'ont absolument pas notre niveau de vie, qui n'ont même pas le niveau de vie de nos parents, bien que certains puissent avoir de temps à autre un portable, bien que certains de temps à autre utilisent une mobylette ou une voiture, bien que certains de temps à autre puissent même prendre l'avion, mais d'une manière générale, dans le long temps, ils n'ont absolument pas accès à ce à quoi nous avons accès. Et chez nous, en contrepartie, nous avons commencé à importer de la pauvreté. Elle n'est que financière et notre collectif est capable de payer. Il y a un certain nombre de gens, entre trois et six millions de personnes qui ne vivent plus d'une activité rémunérée, ce qui aurait été le cas il y a cinquante ans. Et donc petit à petit, ils ont l'impression qu'ils peuvent accéder à certaines choses, et en fait, de plus en plus, ils ne pourront plus y accéder. Et là, il y aura un coincement à un moment, c'est-à-dire que on ne pourra plus agrandir le nombre de personnes qui ont accès à des choses merveilleuses que sont la voiture, le portable ou l'ordinateur parce qu’il n'y a plus de ressources. Et il y aura un nombre de plus en plus grand qui comprendront qu'ils n'y auront plus jamais accès non plus. Là, c'est difficile de dire si cet effet, créera un effet de révolution comme les Jacqueries quand il n'y avait plus à manger, ou quand il y avait trop d'impôts. On ne sait pas.

\vspace{1\baselineskip}

M : Bon on va s’arrêter là je pense, c'était une bonne discussion, et ça va faire bientôt une heure !

\vspace{1\baselineskip}

Si vous avez besoin de précisions vous m'envoyez un mail. Mais le petit bouquin de Serge Latouche, je pense est le plus sérieux sur la question. Souvent, il n'en parle pas trop dans ses conférences, parce que c'est plus parlant la problématique écologique ou technique. Et en fait, la problématique psycho-sociale est beaucoup plus importante. 
Et bien merci à vous et bon travail.

\vspace{1\baselineskip}

Merci à vous !

\textit{[Remercions également Jean et Amine pour avoir joué les scribes]}