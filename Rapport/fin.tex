%Résumé Français + mots clés
\vspace{-1cm}\begin{center}
\textit{L'\OP, réalité industrielle ou lubie consumériste ? }
\end{center}


\begin{Large}
\textbf{Résumé}
\end{Large}

\vspace{1\baselineskip}

L'\op est un terme au nombreux visages, apparu récemment sur le devant de la scène. Il est lié au sentiment croissant des consommateurs que nos produits ont une durée de vie plus faible qu'il y a une vingtaines d'année. 
On peut surtout référencer deux types : l'\op technique, liée à la fabrication du produit et l'\op psychologique, liée à la publicité.

Certains tentent de contrer ses effets, en proposant de l'aide pour réparer les produits de tous les jours, en proposant des produits modulables ou de nouvelles lois pour obliger les industriels à créer des objets viables. 
En effet, la production à outrance lié à notre modèle économique implique un épuisement des matières premières et de nombreux déchets dont on ne sait que faire. Et cette production, motivée par le profit, signifie pour l'entreprise que les biens doivent être achetés, sous peine de pertes. La plupart du temps, le consommateur est poussé, grâce à la publicité, au rachat et au changement de produits parfois encore fonctionnels. 

Mais l'\op technique est elle un sujet à controverse, dans le sens où, bien qu'elle nous paraisse évidente du fait de nos impressions quotidiennes, de nombreux économistes la considère comme un mythe. Alors, l'\op est-elle simplement une vision du consommateur, ou est-elle réellement appliquée à nos biens de consommation ?

\vspace{2\baselineskip}

\begin{large} \emph{Mots-clés :} \end{large}Obsolescence technique, Obsolescence psychologique, Ressources naturelles, Développement durable, Association de consommateurs, Décroissance, Consumérisme, Société de consommation

Obsolescence technique, obsolescence psychologique, ressources naturelles, développement durable, association de consommateurs, décroissance, consumérisme, 


\vfill

%English Abstract + keywords

\begin{Large}
\textbf{Abstract}
\end{Large}

\vspace{1\baselineskip}

Planned obsolescence is a term with multiple faces that only recently came to the spotlight. It is linked to the growing feeling of consumers that our products have a shorter lifetime than twenty years ago. We can mainly reference two types: technical obsolescence, linked to the product’s fabrication, and psychological obsolescence, linked to advertising.

Some try to tackle its effects by offering help fixing everyday products, by offering modular designed products, or new laws to force manufacturers into creating viable objects. Indeed, the excessive production related to our economic model implies the depletion of raw materials and the creation of waste we are unable to reuse. And this profit-driven production means, for the company, that its goods must be sold, otherwise they risk losses. Most of the time, the consumer is pushed, thanks to advertising, to rebuying or replacing products that are sometimes still functional.

But is planned technical obsolescence a controversial issue, in the sense that even though its existence seems obvious to us due to our daily impressions, numerous economists consider it is a myth. So, is planned obsolescence simply a view of the consumer, or is it really applied to our consumer goods?
\vspace{2\baselineskip}

\begin{large}\emph{Keywords:} \end{large} Technical obsolescence, Planned obsolescence, Psychological obsolescence, Natural resources, Sustainable development, Consumers association, Degrowth, Consumerism, Consumer society
