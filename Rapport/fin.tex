%Résumé Français + mots clés

\begin{Large}
\textbf{Résumé}
\end{Large}

\vspace{1\baselineskip}

L'\op est un terme au nombreux visages, apparu récemment sur le devant de la scène. Il est lié au sentiment croissant des consommateurs que nos produits ont une durée de vie plus faible qu'il y a une vingtaines d'année. 
On peut surtout référencer deux types : l'\op technique, liée à la fabrication du produit et l'\op psychologique, liée à la publicité.

Certains tentent de contrer ses effets, en proposant de l'aide pour réparer les produits de tous les jours, en proposant des produits modulables ou de nouvelles lois pour obliger les industriels à créer des objets viables. 
En effet, la production à outrance lié à notre modèle économique implique un épuisement des matières premières et de nombreux déchets dont on ne sait que faire. Et cette production, motivée par le profit, signifie pour l'entreprise que les biens doivent être achetés, sous peine de pertes. La plupart du temps, le consommateur est poussé, grâce à la publicité, au rachat et au changement de produits parfois encore fonctionnels. 

Mais l'\op technique est elle un sujet à controverse, dans le sens où, bien qu'elle nous paraisse évidente du fait de nos impressions quotidiennes, de nombreux économistes la considère comme un mythe. Alors, l'\op est-elle simplement une vision du consommateur, ou est-elle réellement appliquée à nos biens de consommation ?

\vspace{2\baselineskip}

\begin{large} \emph{Mots-clés :} \end{large}Obsolescence technique, Obsolescence psychologique, Ressources naturelles, Développement durable, Association de consommateurs, Décroissance, Consumérisme, Société de consommation


\vspace{4\baselineskip}


%English Abstract + keywords

\begin{Large}
\textbf{Abstract}
\end{Large}