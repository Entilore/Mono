\chapter{Glossaire}

\begin{description}

\item[UFC] : L'Union Fédérale des Consommateurs est une association de consommateurs, d'usagers, de contribuables et de défense de l'environnement.

\item[CEC] : Le Centre Européen de la Consommation est une association franco-allemande d’information et de conseils aux consommateurs et a pour mission de garantir et promouvoir les droits des consommateurs en Europe.

\item[Charges fixes] : Appelées aussi "charges structurelles" ou "charges de structure", ce type de charges restent indépendantes à court terme du niveau de vente ou de production de l’entreprise. Parmi les charges fixes, on trouve les loyers, les assurances, les amortissements des immobilisations et les salaires.

\item[Footprint Network] : C'est un laboratoire d’idées international qui fournit des outils pour la comptabilité de l'empreinte écologique afin de guider les décisions politiques dans un monde aux ressources limitées.

\item[GESAMP] : The Joint Group of Experts on the Scientific Aspects of Marine Environmental Protection, ou Le Groupe mixte d'experts sur les aspects scientifiques de la protection du milieu marin, c'est un organisme consultatif, créé en 1969, qui conseille le système des Nations Unies (ONU) sur les aspects scientifiques de la protection de l'environnement marin.

\item[Électrolyse] : L’électrolyse est une réaction chimique se déroulant grâce à un courant électrique. La cellule électrolytique est constituée de deux électrodes en métal inerte - la cathode (reliée au pôle moins du générateur) et l'anode (reliée au pôle positif du générateur) - immergées dans la solution dite électrolytique. L'électrolyse est principalement utilisée dans l'industrie pour la synthèse ou la séparation 
de composés.

\item[ADEME] : Agence de l’Environnement et de la Maîtrise de l’Énergie. C'est l'opérateur de l'État pour accompagner la transition écologique et énergétique. Il s'agit d'un établissement public à caractère industriel et commercial (EPIC) placé sous tutelle conjointe du ministère de l’Écologie, du Développement durable et de l’Énergie et du ministère de l’Éducation nationale, de l’Enseignement supérieur et de la Recherche.

\item[DEEE] : Déchets d’Équipements Électriques et Électroniques. Ils sont une catégorie de déchets constituée des équipements en fin de vie, fonctionnant à l'électricité ou via des champs électromagnétiques, ainsi que les équipements de production, de transfert et de mesure de ces courants et champs (ce sont surtout des ordinateurs, imprimantes, téléphones portables, appareils photos numériques, réfrigérateurs, jeux électroniques ou télévisions).

\item[LED] : Light-Emitting Diode, en français la diode électroluminescente, c'est un composant opto-électronique capable d’émettre de la lumière lorsqu’il est parcouru par un courant électrique. %Une diode électroluminescente ne laisse passer le courant électrique que dans un seul sens.

\item[SWOT] : L’analyse SWOT ou matrice SWOT est un outil de stratégie d'entreprise permettant de déterminer les options stratégiques envisageables au niveau d'un domaine d'activité stratégique (DAS ou SBU). Le terme SWOT est un acronyme issu de l'anglais : Strengths (forces), Weaknesses (faiblesses), Opportunities (opportunités), Threats (menaces).

\end{description}