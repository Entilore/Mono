\chapter{Glossaire}

\begin{description}

\item[DEEE] : Déchets d’Équipements Électriques et Électroniques sont une catégorie de déchets constituée des équipements en fin de vie, fonctionnant à l'électricité ou via des champs électromagnétiques, ainsi que les équipements de production, de transfert et de mesure de ces courants et champs (ce sont surtout des ordinateurs, imprimantes, téléphones portables, appareils photos numériques, réfrigérateurs, jeux électroniques ou télévisions) 

\item[ADEME] : Agence de l’Environnement et de la Maîtrise de l’Énergie. C'est est l'opérateur de l'État pour accompagner la transition écologique et énergétique. C'est un établissement public à caractère industriel et commercial (EPIC) placé sous tutelle conjointe du ministère de l’Écologie, du Développement durable et de l’Énergie et du ministère de l’Éducation nationale, de l’Enseignement supérieur et de la Recherche.

\item[UFC] : L'Union fédérale des consommateurs est une association de consommateurs, d'usagers, de contribuables et de défense de l'environnement.

\item[CEC] : Le Centre Européen de la Consommation est une association franco-allemande d’information et de conseils aux consommateurs et a pour mission  de garantir et promouvoir les droits des consommateurs en Europe.

\item[Footprint Network] : C'est un laboratoire d’idées international qui fournit des outils de comptabilité d'empreinte écologique afin de guider les décisions politiques appropriées dans un monde aux ressources limitées.

\item[GESAMP] : The Joint Group of Experts on the Scientific Aspects of Marine Environmental Protection (GESAMP)ou Le Groupe mixte d'experts sur les aspects scientifiques de la protection du milieu marin, c'est est un organe consultatif, créé en 1969, qui conseille le système des Nations Unies (ONU) sur les aspects scientifiques de la protection de l'environnement marin.

\item[Électrolyse] : L’électrolyse de l'eau est un procédé électrolytique qui décompose l'eau en dioxygène et dihydrogène gazeux avec l'aide d'un courant électrique. La cellule électrolytique est constituée de deux électrodes en métal inerte - la cathode (reliée au pôle – du générateur) et l'anode (reliée au pôle + du générateur) - immergées dans l'eau et connectées aux pôles opposés de la source de courant continu.

\item[Charges fixes] : ...


\end{description}