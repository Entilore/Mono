
\subsection{La contribution au réchauffement climatique}


Après leur extraction, les matières premières sont rassemblées, transportées, puis transformées en d'autres composés, qui seront eux-même possiblement traités ou consommés dans des processus divers et variés. Tout cela a malheureusement un coût parfois important pour l'environnement.

\medbreak Le transport est un élément important pour le modèle de l'\op. En effet, il s'agit d'avoir un apport régulier de biens pour la vente, sinon, produire en continu ne servirait pas. On distingue ainsi plusieurs phases d'acheminement des produits.
\\
La première est le transport des matières premières depuis le lieu d'extraction jusqu'aux usines de transformation. Les principaux pays exportateurs sont situés en Amérique du Sud et en Afrique (----Ref----). En effet les pays du nord ont consommé la plupart des leurs (charbon, mines de métaux), ou certaines normes de sécurité de l'environnement ne leur permettent pas d'extraire pour peu cher, et il devient plus avantageux pour eux d'acheter à l'étranger, puis d'amener ces ressources aux lieux de production. La destination phare est l'Asie de l'est, "L'Atelier du Monde", où la plupart des produits sont manufacturés.(----Ref----).

Ces produits sont ensuite envoyés essentiellement dans les pays du nord, Europe et Amérique du Nord. Ils seront stockés dans des entrepôts, puis de nouveau transporté vers les lieux d'achats (supermarchés, boutiques). 

Et finalement, les produits consommés et mis à la poubelle sont emmenés vers une décharge et parfois malheureusement ceux-ci finissent en décharge sauvage en Afrique (----Ref----). Ils restent ainsi sans traitement et polluent durant de nombreuses années.

On a ainsi des bateaux, trains, avions, camions qui se déplacent à travers le monde. 
-----> BCP DE POLLUTION


Pillage des matières premières / arrosages de dictateurs

Transport : 
 - matières 1ères --> usines
 - usines --> stockage
 - stockage --> grandes surfaces
 - déchetteries --> Afrique

Transformation
--> double utilisation matières premières


