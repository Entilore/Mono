\subsection{Autres conséquences et dégâts collatéraux}


Toute ces productions créent beaucoup de déchets, qui finissent le plus souvent en décharge et sont même parfois envoyés dans d'autre pays, où ils sont abandonnés, au grand dam des populations locales.

\bigbreak En France, selon l'ADEME (Agence de l’Environnement et de la Maîtrise de l’Énergie), un français produisait 590 kg de déchet par an, ce qui équivaut à 38.5 millions pour la France entière \cite{ademeStat} . Toujours selon l'ADEME \cite{ADEMEprodDechetFR}, le coût moyen de gestion et d'élimination des déchets par an et par habitant est de 85 euros. Cela représente 5,1 milliards d'euros chaque année pour l'ensemble de la France.

\medbreak Devant le coût du recyclage des déchets, certaines entreprises n’hésitent pas à les envoyer dans les pays en voie de développement. \textit{Greenpeace} dénonce ainsi dans un reportage \cite{GreenPeaceGhana} l'envoi de produits au Ghana, mais les principales destinations sont la Chine et l'Inde. L'association explique que les entreprises ne sont légalement pas autorisées, mais réussissent à contourner la loi. En effet, elles peuvent envoyer des produits dits de seconde main afin qu'ils soient réutilisés ou revendus là-bas. Cependant, comme l'explique un responsable ghanéen interviewé lors du reportage, les biens, principalement des téléviseurs ou des ordinateurs, ne fonctionnent déjà plus à l'arrivée. Les entreprises profitent donc du système pour expédier les déchets dont ils doivent s'occuper, au lieu de les recycler convenablement. Les populations locales récupèrent ce qu'il est possible ou incinèrent les déchets, constitués de métaux et de plastiques, parfois très dangereux, comme le cadmium ou le mercure et respirent donc des fumées particulièrement toxiques. Un membre de \textit{Greenpeace} explique ainsi que les produits chimiques présents sont particulièrement dangereux pour le système nerveux et le développement des enfants qui récoltent ces métaux.

\bigbreak En Chine, la décharge de Guiyu, surnommé le "Cimetière électronique", est devenue tristement célèbre. En effet, elle est la plus grande décharge de \textit{DEEE} au monde (Déchets d’Équipements Électriques et Électroniques), déchets en provenance des États-Unis, du Japon, du Canada, ... Là-bas les déchets sont dits "recyclés". Cependant, un groupe de scientifiques européens\cite{Deeechine} montrait que	le recyclage effectué était inutile. Les techniques utilisées sont sommaires, les \textit{DEEE} brûlés de façon brute, sans aucun tri. Ainsi, pour récupérer les métaux, tout est incinéré de la même façon. Cela engendre des fumées toxiques, composées par exemple de dioxines ou de plomb. Les rivières et les sols sont également pollués par les résidus, comme l'étain ou le chrome. Les populations sont particulièrement touchées, à cause de l'eau, de l'air et du sol. D'après les scientifiques, le taux moyen de plomb dans le sang des enfants vivant à Guiyu est de 15.3 $\mu$g/dL alors qu'en France, le taux de plomb dans le sang ne doit pas être supérieur à 10 $\mu$g/dL. 

\bigbreak De nombreux déchets sont donc produits dans le but de produire toujours plus et toujours plus vite. On en arrive à de telles quantités de déchets, que ceux-ci sont envoyés à l'étranger, pour y polluer les sols et les habitants. 

On pourrait ainsi se dire que le recyclage est une bonne solution à cette problématique. Nous allons voir cependant que cela n'est pas toujours aussi simple.

