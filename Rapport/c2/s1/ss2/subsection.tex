\subsection{Un modèle sans garantie de viabilité}


Comme l’explique Alexandre Delaigue, il n’est pas forcément plus rentable de vendre de gros volumes.
En effet, il faut différencier charges fixes et charges variables comme il faut différencier chiffre d’affaire et bénéfice. On fait effectivement un chiffre d’affaire plus importants si l’on vend plus, cependant la marge peut être inférieure. Par exemple, les coûts de distribution peuvent radicalement changer pour la production d’une seule unité supplémentaire, si notre camion ne peut transporter que seulement  1000 unités,  la production de la 1001e unité impliquera des coûts de stockage ou de transport bien plus importants que le bénéfice attendu pour une seule unité.
\medbreak
Ensuite, une entreprise a-t-elle intérêt à vendre des produits ayant une durée de vie de de 6 mois ou des produits deux fois plus chers ayant une durée de vie d’un an ? Si le consommateur reste fidèle à cette entreprise, le chiffre d’affaire sera effectivement le même. Par contre l’entreprise n’a pas forcément envie de passer deux fois par les processus de productions et de distributions, qui ont un coût proportionnel au nombre d’unités vendues, et non au chiffre d’affaire. On peut cependant nuancer le propos en prenant en compte des compétences, matières premières  et des phases de recherche et développement nécessaires pour la fabrication de produits plus performants. Dans ce cas, la viabilité même de l’obsolescence programmée fonctionnelle est mise à mal.
\medbreak
Par contre si le consommateur est incité à remplacer ce produit plus cher au bout de 6 mois, avec une campagne marketing efficace, on peut parler d’obsolescence psychologique, ce qui peut être grandement  profitable à l’entreprise. En effet, elle vend alors des produits fiables, appréciés par le consommateur et plus chers, à la cadence d’une entreprise bridant les performances du produit pour le rendre moins cher.

\bigbreak
Enfin, ce sont le plus souvent les entreprises les plus innovantes qui sont la cible de critiques (Apple, Samsung), d’une part car l’innovation rend obsolètes les anciens produits, ce qui n’est en soi pas de l’obsolescence programmée. Et enfin, l’implémentation de nouvelles technologies peut être sujette à des erreurs d’ingénierie. On peut évidemment rappeler le cas des condensateurs placés trop près d’une pièce émettrice de chaleur sur des téléviseurs Samsung, ou la batterie défectueuse et irremplaçable des iPod . Pour ce dernier cas, l’erreur ne vient pas du problème de remplacement, mais plus d'un choix de design, or ce point n’aurait sûrement même pas été relevé si les batteries n’étaient pas défectueuses.

\newpage
