\subsection{L'\OP, complémentaire de notre système économique ?}

Serge Latouche, économiste français connu et reconnu, dit de notre société de consommation que la croissance économique est basée sur la publicité, le crédit et l'\op. La croissance serait un des moteurs de notre modèle économique, qui lui permettrait de continuer à exister et à avancer. Des liens existent entre la société de consommation dans laquelle nous vivons et l'\op. Celle-ci est particulièrement liée à la consommation et donc à la croissance. C'est en tout cas ce qu'il présente dans son livre \textit{Bon Pour la Casse}. Nous allons essayer d'expliciter son point de vue, qu'il décrit dans son essai.

\bigbreak
Dans notre modèle de société de consommation, le profit est au cœur de la vie d'une entreprise. En vendant suffisamment ses produits ou services, une société engrangera des bénéfices et pourra investir dans la bonne marche de son activité. Celle-ci pourra acheter de nouvelles machines, engager de nouveaux employés afin de s'agrandir, ce qui finalement lui permettra de se développer, et ainsi de suite. Cela lui permet par exemple de contrer plus facilement certains coups durs.

Bien entendu certaines entreprises vendent des produits qui se veulent de très bonne qualité, comme Lamborghini ou Rolex, et n'espèrent pas forcément un nombre de vente important, mais plutôt régulier, afin d'engranger tout de même des profits. Ainsi, la vente de l'un de leur produit ayant une forte valeur ajoutée leur permet de couvrir le faible taux de vente.

D'un autre coté, d'autres entreprises nécessitent de vendre en grande quantité afin d'amortir les coûts de production (ex : les charges fixes). Les revenus de l'entreprise sont donc fortement liées à la demande des consommateurs. Le but peut donc être pour une entreprise de pousser le consommateur à l'achat ou au rachat. Il existe plusieurs façons de faire :  le renouvellement des produits proposés, fidéliser les clients par la qualité des services ou des produits, attirer les clients par la publicité. Ou encore par l'obsolescence programmée. Par l'incompatibilité logicielle ou par la technique. La publicité à outrance et uniquement dans le but de faire racheter aux clients des produits, peut également être considérée comme telle. Tout en sachant que l'\op technique n'est pas prouvée et nécessite de la retenue.

Ainsi, le produit courant est jeté au profit d'un nouveau. Il pourra être plus joli ou à la mode, comme présenté dans la dernière publicité vue à la télévision ou sur tout autre média, ou bien le produit ne sera  tout simplement plus fonctionnel.


%\subsection{Obsolescence et consommation}
%
%Dans notre modèle de société de consommation, le profit est au cœur de la vie d'une entreprise. En vendant suffisamment ses produits ou services, une société engrangera des bénéfices et pourra investir dans la bonne marche de son activité. Celle-ci pourra acheter de nouvelles machines, engager de nouveaux employés afin de s'agrandir, ce qui finalement lui permettra de s'agrandir, et ainsi de suite.
%
%% Pour cela, il lui faut donc réussir à écouler ses stocks, afin de vendre le plus possible, et donc susciter la consommation. Cela se fait souvent au travers du marketing et plus particulièrement de la publicité, qui nous incite à acheter le produit d'une marque X. L'idée est que certains produits ou marques nouvelles sur le marché doivent être connus du grand public, sans quoi celui-ci risque de ne pas vouloir du produit.
%
%\bigbreak
%% Cependant, si le produit acheté est durable, du fait de sa bonne qualité ou de l'inutilité de son rachat, l'entreprise ne gagnera qu'une seule fois par acheteur, et finit par ne plus avoir de clients.
%%Il faut donc de nouveau pousser le consommateur à l'achat par la publicité, le renouvellement des produits proposés, ou par l'obsolescence programmée. 
%Ainsi, le produit courant est jeté au profit d'un nouveau. Il pourra être plus joli ou à la mode, comme présenté dans la dernière publicité vue à la télévision ou sur tout autre média, ou bien le produit ne sera  tout simplement plus fonctionnel. Bien entendu certaines entreprises vendent des produits qui se veulent de très bonne qualité réussissent à tirer leur épingle du jeu et à  
%
%%La dernière raison peut être totalement involontaire, 
%alors le produit devra être racheté si le consommateur en a l'utilité. Inversement, le dysfonctionnement serait prévu et recherché par le fabricant. Ici apparaît le concept de l'obsolescence programmée : le produit est prévu dans sa construction pour ne durer qu'un certain temps, comme une année par exemple. Au bout de cette période, le consommateur serait donc poussé au rachat s'il souhaite continuer à bénéficier de l'avantage procuré par le produit.
%
%Tout cela reste une hypothèse, étant donné que seuls quelques exemples d'\op avérée ont filtré. Cela ne signifie pas qu'elle n'existe pas. Simplement, comme le disait M. Clémentin (----ref interview----), on ne peut affirmer qu'elle existe.
%
%\bigbreak
%Dans le même temps, il est vrai que certaines entreprises ne permettent pas la réparation de leur produits. Plusieurs exemples existent, comme présentés dans le film \textit{Prêt à jeter} \cite{PretAjeter}. Certains portables sont impossibles à démonter, et quand bien même on tenterait de forcer nos appareils électroniques, le plus souvent de petites pièces qui permettaient à l'objet de tenir se cassent, rendant la réparation impossible. Pour des objets nécessitant des pièces détachées, celles-ci ne sont parfois plus disponibles à l'achat après une certaine période. Sauf que ces pièces détachées sont le plus souvent utilisées quelques années après l'achat, plutôt que dans l'année.
%
%Dans les années trente, des économistes proposaient de sortir de la crise en obligeant les consommateurs à jeter leur biens après une certaine date, afin d'en racheter de nouveaux, et ainsi relancer la consommation. Ce qui aurait permis selon eux de sortir de la crise. On retrouve ici l'idée d'une \op qui permet de contrebalancer la production de masse.
%
%%CREDITS !!!!!!!!!!!!!!!!!!!!!!!!!!!!!!!!!!!
%%consommation --> emplois !
%%publicité --> emplois !
%%de plus en plus de croissance --> on veut toujours plus --> toujours plus de gens à rechercher profits --> plus de production --> OP