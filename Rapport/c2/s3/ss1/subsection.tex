\subsection{Le recyclage, une obligation}

Le recyclage est défini par le Larousse  comme étant l'\og{}\textit{ensemble des techniques ayant pour objectif de récupérer des déchets et de les réintroduire dans le cycle de production dont ils sont issus}\fg{}. L'un des objectifs de l'Europe est de réduire les déchets. Une série de lois a donc été adoptée par l'union européenne, puis retranscrite par la France. 

La directive européenne 2012/19/UE du 4 juillet 2012, retranscrite dans le droit français par le décret 2014-928 du 19 août 2014 oblige  les magasins dont la surface est supérieure à $400m^2$ à accepter n'importe quel \textit{DEEE} de petite taille, et ce sans obligation d'achat pour le consommateur. 

Depuis 2005, un décret (articles R. 543-172 à R. 543-206 du code de l'environnement) mis en place pour transposer % terme officiel
 une directive européenne, oblige le consommateur à recycler un objet électronique lorsqu'il en achète un nouveau. Pour ce faire, il peut le ramener dans le lieu d'achat du nouveau produit, ou bien en déchetterie, ou encore dans les réseaux solidaires comme Emmaüs.

\begin{wrapfigure}{l}{0.3\linewidth}
~\includegraphics[scale=0.33]{Rsc/pictopoubellebarree.jpg} 
\caption{Pictogramme instauré par la loi de 2006}
\label{fig::picPoubelleBarree}
\end{wrapfigure} 
La loi a également mise en place le pictogramme représentant une poubelle barrée (figure \ref{fig::picPoubelleBarree}), présent sur chaque objet électronique, pour rappeler au consommateur qu'il lui est nécessaire de recycler le nouveau produit à la fin de son utilisation.

Plus flagrant peut-être, la loi a instauré la taxe d'éco-participation. Celle ci rend le consommateur contribuable au recyclage. Elle est calculée à partir du coup de collecte du produit (donc de son poids et de sa taille) et du coup de traitement lors du recyclage. Ainsi, les prix varient d'un centime d'euro pour un GPS à 13\euro~pour un congélateur. 

La taxe est directement payée par le consommateur. Elle est collectée par le distributeur (\textit{But}, \textit{Carrefour}, \dots). Une partie est donnée aux constructeurs (ceux qui acceptent les \textit{DEEE}), le reste est distribué aux différents éco-organismes. 

Les éco-organismes sont divisés en trois entités différentes : les organismes de collecte, les organismes de traitement et les organismes de remise en état. Ces trois institutions différentes vont toucher une division correspondant au service fourni lors du recyclage. 

\medbreak


\textit{Envie 44}  est une réseau solidaire basé à Nantes, dont le directeur adjoint, interviewé par \textit{Télénantes}, affirme toucher 20\% de ses recettes grâce aux subventions de l'état\cite{7andco}.

L'association récupère les \textit{DEEE} auprès des producteurs de déchets électroniques (principalement des administrations : mairies, hôpitaux) et se charge du recyclage. 

Pour \textit{Envie 44}, le recyclage se divise en deux parties. 
\begin{itemize}
\item 12\% des DEEE étaient rénovés en 2008. Il estimait alors que ce taux pourrait monter jusqu'à 25\%, mais pas plus : 3 machines sont nécessaires pour en réparer une. En 2014, l'entreprise a vendu plus de 7000 articles issus du recyclage. 
\item le reste est envoyé dans des centres de traitement, à Rennes et à Angers.
\end{itemize}

Pour l'entreprise, l'éco-participation a été un véritable moteur, qui lui a permis de se développer. Elle compte aujourd'hui 92 employés. Le filon a attiré de nombreuses nouvelles entreprises de recyclage, fondées à la suite du décret. 

\bigbreak

\begin{figure}[h]
\begin{minipage}{0.5\linewidth}
\begin{center}
\includegraphics[scale=0.3]{Rsc/ecoembalage.png} 
\end{center}
\end{minipage}
\begin{minipage}{0.5\linewidth}
\begin{center}
\includegraphics[scale=0.9]{Rsc/triman.jpg} 
\end{center}
\end{minipage}
\caption{\scriptsize{A gauche, l'ancien logo certifiant que l'entreprise verse une quotisation à Éco-embalage. A droite le nouveau logo signifiant que l'objet est recyclable, qui sera mis en place en 2015.}}
\end{figure}

\bigbreak

Ainsi certaines associations ou entreprises profitent des \textit{DEEE} pour créer de nouveaux produits fonctionnels. Cependant, moins d'un quart des déchets que récupère \textit{Envie 44} est rénové. Nous disions plus tôt que les 75\% restant étaient envoyés vers  des centres de traitements. 

D'après \textit{ADEME}\cite{ADEME_unite_traitement}, \og \textit{le terme " traitement " correspond à la transformation des fractions valorisables des déchets}  \fg{}. Concrètement, si une partie des déchets est triée et écologiquement revalorisée, nos \textit{DEEE} sont principalement incinérés, avec ou sans valorisation énergétique. 

Autre point surprenant : les déchets de Nantes sont envoyés à Angers ou à Rennes (90 ou 110 km).   Les déchets sont donc déplacés, puis brulés. La production de \textit{CO2} n'est donc pas négligeable. 

\bigbreak

Ainsi, le recyclage des \textit{DEEE} est obligatoire. Une partie des déchets est réinjecté dans le marché suite à une rénovation, cependant, une partie des objets n'ayant pas pu être sauvée sera brûlée, contribuant ainsi au réchauffement climatique. Le recyclage reste malgré tout une bonne solution, cependant des progrès restent à faire. 
\medbreak
Une autre solution pour émettre moins de déchets consisterait à repenser le produit à la base pour qu'une pièce défectueuse n'engendre pas la mort de la totalité de l'objet. 
