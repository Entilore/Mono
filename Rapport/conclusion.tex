\chapter*{Conclusion}

%op technique : peu d'exemples concrets, difficile à prouver
%
%op psychologique : coeur du sujet
%rôle du consommateur
%
%mythe
%
%environnement
%
%nécessité
%
%luttes

L'idée de mythe concernant l'\op \textit{technique} serait donc plus ou moins fondée. Si le consommateur pense que la durée de vie des produits est plus faible depuis quelques dizaines d'années , il oublie aussi que le prix de ces même produits a beaucoup diminué durant cette période. Cela implique forcément une baisse de la qualité des biens de consommations. L'\op \textit{voulue} semble anecdotique, du fait du faible nombre d'exemples concrets. Cela ne signifie pas qu'elle n'existe pas. Elle reste cependant non prouvée, et s'assimile pour certains à une théorie du complot, qui se servirait des entreprises comme bouc émissaire. 

\bigbreak
En revanche, l'\op \textit{psychologique} est un phénomène connu et reconnu, conscient et recherché par beaucoup d'entreprises au travers du marketing et plus particulièrement de la publicité. Celle-ci s'est développée sur de nombreux médias et est devenue omniprésente, dans le cas de contenus gratuits. L'esthétisme ou les effets de mode amènent les consommateurs à considérer que certains produits sont obsolètes. La publicité pousse ainsi les gens à consommer et à renouveler, racheter des biens qu'ils possèdent déjà ou qui fonctionnent encore. Le consommateur est donc l’acteur principal de cette problématique.

\bigbreak
La grandes question est la nécessité de l'obsolescence programmée dans notre modèle économique. Celle-ci est particulièrement liée à la société de consommation. Le fait est que le renouvellement des produits implique de nombreuses dérives écologiques et que certains acteurs sont déjà à la recherche d'autres solutions. Ils arguent que l'on peut vivre sans société de consommation, et donc sans obsolescence, afin d'éviter de gaspiller les ressources naturelles.