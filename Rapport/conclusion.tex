\chapter*{Conclusion}

%op technique : peu d'exemples concrets, difficile à prouver
%
%op psychologique : coeur du sujet
%rôle du consommateur
%
%mythe
%
%environnement
%
%nécessité
%
%luttes

L'idée de mythe concernant l'\op technique serait donc plus ou moins fondée. Si le consommateur pense que la durée de vie des produits est plus faible depuis quelques dizaines d'années , il oublie aussi que le prix de ces même produits a beaucoup diminué durant cette période. Cela implique forcément une baisse de la qualité des biens de consommations. L'\op voulue semble anecdotique, du fait du faible nombre d'exemples concrets. Cela ne signifie pas qu'elle n'existe pas. Elle reste cependant non prouvée, et apparaît pour certains comme une théorie du complot, qui pointe un bouc émissaire. 

En revanche, l'\op psychologique est un phénomène connu, reconnu et recherché par beaucoup d'entreprises, au travers du marketing et plus particulièrement de la publicité. Celle-ci s'est développée sur de nombreux médias et est devenue omniprésente, recherchée parfois, dans le cas de contenus gratuits. L'esthétisme ou des effets de mode amènent les consommateurs à décider que certains produits sont obsolètes. La publicité pousse ainsi les gens à consommer et à renouveler, racheter des biens qu'ils possèdent déjà ou qui fonctionnent. Le consommateur est donc l'un des acteurs principaux de cette problématique.

L'une des grandes question est la nécessité de l'\op dans notre modèle économique. Celle-ci est particulièrement liée à la société de consommation. Le fait est que le renouvellement des produits implique de nombreuses dérives écologiques et donc certains acteurs ont cherché d'autres solutions. Il a ainsi été montré que l'on peut vivre sans société de consommation, et donc sans obsolescence, afin d'éviter de gaspiller les ressources naturelles.